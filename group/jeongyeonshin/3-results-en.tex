\makeatletter
\def\input@path{{../}}
\makeatother
\documentclass[../document-en.tex]{subfiles}

\begin{document}

\section{Results}
\subsection{Participants characteristics} 

Descriptive statistics for participants' demographics and covariates are presented in Table~\ref{tab:demo} and Table~\ref{tab:cov}. The sample size was 3,568 at Y0 (mean age = 9.96, \textit{SD} = 0.63) and 2,222 at Y2 (mean age = 11.94, \textit{SD} = 0.63). 



\begin{table}[]
\caption{Descriptive statistics for demographics and socioeconomic status}
\label{tab:demo}
\scriptsize
\renewcommand{\arraystretch}{0.8} % 행 높이를 더 줄임
\setlength{\tabcolsep}{4pt} % 열 간격을 더 줄임
\begin{adjustbox}{max width=\textwidth, center} % 표 중앙 정렬
\begin{tabular}{@{}lcccl@{}}
\toprule                & \multicolumn{2}{c}{Y0 (N = 3,568)} & \multicolumn{2}{c}{Y2 (N = 2,222)} \\ \cmidrule(l){2-5} 
                              & N (\%)               & \multicolumn{1}{c}{Mean (SD)}    & N (\%)               & \multicolumn{1}{c}{Mean (SD)}    \\ \midrule
Child age (years)             & \multicolumn{1}{l}{} & \multicolumn{1}{c}{9.96 (0.63)}  & \multicolumn{1}{l}{} & \multicolumn{1}{c}{11.94 (0.63)} \\
Gender                & \multicolumn{1}{l}{}       &       & \multicolumn{1}{l}{}       &       \\
\hspace{1em}Male                  & 1,823 (51\%)               &       & 1,150 (52\%)               &       \\
\hspace{1em}Female                & 1,745 (49\%)               &       & 1,072 (48\%)               &       \\
Race and Ethnicity    & \multicolumn{1}{l}{}       &       & \multicolumn{1}{l}{}       &       \\
\hspace{1em}White                 & 2,248 (63\%)               &       & 1,464 (66\%)               &       \\
\hspace{1em}Black                 & 259 (7.3\%)                &       & 142 (6.4\%)                &       \\
\hspace{1em}Hispanic              & 603 (17\%)                 &       & 364 (16\%)                 &       \\
\hspace{1em}Asian                 & 65 (1.8\%)                 &       & 36 (1.6\%)                 &       \\
\hspace{1em}Other                 & 393 (11\%)                 &       & 216 (9.7\%)                &       \\
Maternal age at child's birth & \multicolumn{1}{l}{} & \multicolumn{1}{c}{30.3 (5.8)}   & \multicolumn{1}{l}{} & \multicolumn{1}{c}{30.3 (5.8)}   \\
Parental education            & \multicolumn{1}{l}{} & \multicolumn{1}{c}{17.25 (2.35)} & \multicolumn{1}{l}{} & \multicolumn{1}{c}{17.24 (2.34)} \\
Data acquisition site & \multicolumn{1}{l}{}       &       & \multicolumn{1}{l}{}       &       \\
\hspace{1em}Site 1                & 88 (2.5\%)                 &       & 35 (1.6\%)                 &       \\
\hspace{1em}Site 2                & 190 (5.3\%)                &       & 126 (5.7\%)                &       \\
\hspace{1em}Site 3                & 159 (4.5\%)                &       & 83 (3.7\%)                 &       \\
\hspace{1em}Site 4                & 231 (6.5\%)                &       & 184 (8.3\%)                &       \\
\hspace{1em}Site 5                & 118 (3.3\%)                &       & 85 (3.8\%)                 &       \\
\hspace{1em}Site 6                & 229 (6.4\%)                &       & 155 (7.0\%)                &       \\
\hspace{1em}Site 7                & 79 (2.2\%)                 &       & 32 (1.4\%)                 &       \\
\hspace{1em}Site 8                & 110 (3.1\%)                &       & 50 (2.3\%)                 &       \\
\hspace{1em}Site 9                & 152 (4.3\%)                &       & 77 (3.5\%)                 &       \\
\hspace{1em}Site 10               & 175 (4.9\%)                &       & 114 (5.1\%)                &       \\
\hspace{1em}Site 11               & 151 (4.2\%)                &       & 76 (3.4\%)                 &       \\
\hspace{1em}Site 12               & 161 (4.5\%)                &       & 92 (4.1\%)                 &       \\
\hspace{1em}Site 13               & 223 (6.3\%)                &       & 120 (5.4\%)                &       \\
\hspace{1em}Site 14               & 191 (5.4\%)                &       & 133 (6.0\%)                &       \\
\hspace{1em}Site 15               & 65 (1.8\%)                 &       & 44 (2.0\%)                 &       \\
\hspace{1em}Site 16               & 453 (13\%)                 &       & 294 (13\%)                 &       \\
\hspace{1em}Site 17               & 203 (5.7\%)                &       & 132 (5.9\%)                &       \\
\hspace{1em}Site 18               & 146 (4.1\%)                &       & 72 (3.2\%)                 &       \\
\hspace{1em}Site 19               & 81 (2.3\%)                 &       & 51 (2.3\%)                 &       \\
\hspace{1em}Site 20               & 203 (5.7\%)                &       & 144 (6.5\%)                &       \\
\hspace{1em}Site 21               & 148 (4.1\%)                &       & 119 (5.4\%)                &       \\
\hspace{1em}Site 22               & 12 (0.3\%)                 &       & 4 (0.2\%)                  &       \\ \bottomrule
\end{tabular}

\end{adjustbox}
        \parbox{\linewidth}{
        \vspace{0.5em}
        \renewcommand{\baselinestretch}{1}
        \small
        \raggedright
        \textit{Note.} Mean age: Y0 (cross-sectional) = 9.96 years, Y2 (longitudinal) = 11.94 years
        }
\end{table}

% \begin{table}[]
%     \centering
%     \caption{Descriptive statistics for demographics and socioeconomic status}
%     \label{tab:demo}
%     \resizebox{0.9\textwidth}{!}{% 
%         \includegraphics[width=\linewidth]{demo.png}
%     }
% %    \renewcommand{\baselinestretch}{1}
% %   \small
% %    \raggedright
% %        \textit{Note. } At Y2, all variables are collected at Y0 except for child age.
% \end{table}




\begin{table}[]
\caption{Descriptive statistics for adversity before birth, parental psychopathology and child behavior}
\label{tab:cov}
\scriptsize
\renewcommand{\arraystretch}{0.8} % 행 높이를 더 줄임
\setlength{\tabcolsep}{4pt} % 열 간격을 더 줄임
\begin{adjustbox}{max width=\textwidth, center} % 표 중앙 정렬
\begin{tabular}{@{}lcccl@{}}
\toprule
& \multicolumn{2}{c}{Y0 (N = 3,568)}                                 & \multicolumn{2}{c}{Y2 (N = 2,222)}                                 \\ \cmidrule(l){2-5} 
                                       & \multicolumn{1}{c}{N (\%)} & \multicolumn{1}{c}{Mean (SD)}  & \multicolumn{1}{c}{N (\%)} & \multicolumn{1}{c}{Mean (SD)}  \\ \midrule
Birth weight (pound)         &                                  & \multicolumn{1}{c}{6.73 (1.41)} &                                  & \multicolumn{1}{c}{6.74 (1.41)} \\
Premature birth              &                                  &                                 &                                  &                                 \\
\hspace{1em}Yes                          & \multicolumn{1}{c}{607 (17\%)}   &                                 & \multicolumn{1}{c}{385 (17\%)}   &                                 \\
\hspace{1em}No                           & \multicolumn{1}{c}{2,961 (83\%)} &                                 & \multicolumn{1}{c}{1,837 (83\%)} &                                 \\
                             &                                  &                                 &                                  &                                 \\
Cyanosis (blue at birth)     &                                  &                                 &                                  &                                 \\
\hspace{1em}Yes                          & \multicolumn{1}{c}{124 (3.5\%)}  &                                 & \multicolumn{1}{c}{80 (3.6\%)}   &                                 \\
\hspace{1em}No                           & \multicolumn{1}{c}{3,444 (97\%)} &                                 & \multicolumn{1}{c}{2,142 (96\%)} &                                 \\
Pregnant alcohol use/smoking &                                  &                                 &                                  &                                 \\
\hspace{1em}Yes                          & \multicolumn{1}{c}{210 (5.9\%)}  &                                 & \multicolumn{1}{c}{114 (5.1\%)}  &                                 \\
\hspace{1em}No                           & \multicolumn{1}{c}{3,358 (94\%)} &                                 & \multicolumn{1}{c}{2,108 (95\%)} &                                 \\
Parent anxiety problems      &                                  & \multicolumn{1}{c}{53.3 (5.2)}  &                                  & \multicolumn{1}{c}{53.2 (5.0)}  \\
Parent somatic problems      &                                  & \multicolumn{1}{c}{54.3 (5.9)}  &                                  & \multicolumn{1}{c}{54.4 (6.0)}  \\
Parent depressive problems             &                            & \multicolumn{1}{c}{53.8 (5.7)} &                            & \multicolumn{1}{c}{53.8 (5.8)} \\
Parent avoidant personality problems   &                            & \multicolumn{1}{c}{53.0 (5.1)} &                            & \multicolumn{1}{c}{53.1 (5.2)} \\
Parent AD/H problems         &                                  & \multicolumn{1}{c}{53.0 (5.3)}  &                                  & \multicolumn{1}{c}{53.1 (5.4)}  \\
Parent antisocial personality problems &                            & \multicolumn{1}{c}{52.8 (4.4)} &                            & \multicolumn{1}{c}{52.8 (4.4)} \\
Father drug problem          &                                  &                                 &                                  &                                 \\
\hspace{1em}Yes                          & \multicolumn{1}{c}{524 (15\%)}   &                                 & \multicolumn{1}{c}{317 (14\%)}   &                                 \\
\hspace{1em}No                           & \multicolumn{1}{c}{3,044 (85\%)} &                                 & \multicolumn{1}{c}{1,905 (86\%)} &                                 \\
Mother drug problem          &                                  &                                 &                                  &                                 \\
\hspace{1em}Yes                          & \multicolumn{1}{c}{160 (4.5\%)}  &                                 & \multicolumn{1}{c}{93 (4.2\%)}   &                                 \\
\hspace{1em}No                           & \multicolumn{1}{c}{3,408 (96\%)} &                                 & \multicolumn{1}{c}{2,129 (96\%)} &                                 \\
Child externalizing problems &                                  & \multicolumn{1}{c}{44 (9)}      &                                  & \multicolumn{1}{c}{44 (9)}      \\
Child internalizing problems &                                  & \multicolumn{1}{c}{48 (10)}     &                                  & \multicolumn{1}{c}{48 (10)}     \\ \bottomrule
\end{tabular}
\end{adjustbox}

        \parbox{\linewidth}{
        \vspace{0.5em}
        \renewcommand{\baselinestretch}{1}
        \small
        \raggedright
        \textit{Note.} Mean age: Y0 (cross-sectional) = 9.96 years, Y2 (longitudinal) = 11.94 years
        }
\end{table}




% \begin{table}[]
%     \centering
%     \caption{Descriptive statistics for adversity before birth, parental psychopathology and child behavior}
%     \label{tab:cov}
%     \includegraphics[width=1\linewidth]{cov.png}
% %    \renewcommand{\baselinestretch}{1}
% %    \small
% %    \raggedright
% %        \textit{Note. } At Y2, all variables are collected at Y0 except for child externalizing problems, and child internalizing problems.\\
% \end{table}

Table~\ref{tab:els_distribution} presents the distribution of threat and deprivation at Y0 and Y2. There were no significant differences in the distribution of threat (\textit{p} = 1.000) or deprivation (\textit{p} = 1.000) scores between Y0 and Y2, allowing for direct comparisons between cross-sectional and longitudinal analyses. Since nearly half of the participants scored zero, categorical groups were used instead of continuous scores. At both Y0 and Y2, 13\% of participants were classified in the high-threat group, and 9.2\% were in the high-deprivation group.

% Please add the following required packages to your document preamble:
% \usepackage{booktabs}
% \usepackage{multirow}
% \usepackage{graphicx}
\begin{table}[]
\caption{Distribution of threat and deprivation}
\label{tab:els_distribution}
\resizebox{\columnwidth}{!}{%
\begin{tabular}{@{}llcclccl@{}}
\toprule&
  \multicolumn{3}{c}{Y0 (N = 3,568)} &
  \multicolumn{3}{c}{Y2 (N = 2,222)} &
  \multicolumn{1}{c}{\multirow{2}{*}{\textit{p}}} \\ \cmidrule(lr){2-7}
 &
  \multicolumn{1}{c}{N (\%)} &
  Score &
  N (\%) &
  \multicolumn{1}{c}{N (\%)} &
  Score &
  N (\%) &
  \multicolumn{1}{c}{} \\ \midrule
Threat &
   &
  \multicolumn{1}{l}{} &
  \multicolumn{1}{l}{} &
   &
  \multicolumn{1}{l}{} &
  \multicolumn{1}{l}{} &
  \multicolumn{1}{c}{1.000} \\
\hspace{1em}Low threat group &
  \multicolumn{1}{c}{3,096 (87\%)} &
  0 &
  1,670 (47\%) &
  \multicolumn{1}{c}{1,927 (87\%)} &
  0 &
  1,044 (47\%) &
   \\
 &
   &
  1 &
  1,426 (40\%) &
   &
  1 &
  883 (40\%) &
   \\
\hspace{1em}High threat group &
  \multicolumn{1}{c}{472 (13\%)} &
  2 &
  410 (11\%) &
  \multicolumn{1}{c}{295 (13\%)} &
  2 &
  251 (11\%) &
   \\
 &
   &
  3 &
  49 (1.4\%) &
   &
  3 &
  35 (1.6\%) &
   \\
 &
   &
  4 &
  9 (0.3\%) &
   &
  4 &
  4 (0.2\%) &
   \\
 &
   &
  5 &
  4 (0.1\%) &
   &
  5 &
  5 (0.2\%) &
   \\
Deprivation &  & \multicolumn{1}{l}{} & \multicolumn{1}{l}{} &  & \multicolumn{1}{l}{} & \multicolumn{1}{l}{} & \multicolumn{1}{c}{1.000} \\
\hspace{1em}Low deprivation group &
  \multicolumn{1}{c}{3,240 (91\%)} &
  0 &
  1,805 (51\%) &
  \multicolumn{1}{c}{2,017 (91\%)} &
  0 &
  1,107 (50\%) &
   \\
 &
   &
  1 &
  955 (27\%) &
   &
  1 &
  576 (26\%) &
   \\
 &
   &
  2 &
  480 (13\%) &
   &
  2 &
  334 (15\%) &
   \\
\hspace{1em}High deprivation group &
  \multicolumn{1}{c}{328 (9.2\%)} &
  3 &
  206 (5.8\%) &
  \multicolumn{1}{c}{205 (9.2\%)} &
  3 &
  121 (5.4\%) &
   \\
 &
   &
  4 &
  83 (2.3\%) &
   &
  4 &
  61 (2.7\%) &
   \\
 &
   &
  5 &
  31 (0.9\%) &
   &
  5 &
  17 (0.8\%) &
   \\
 &
   &
  6 &
  8 (0.2\%) &
   &
  6 &
  6 (0.3\%) &
   \\ \bottomrule
\end{tabular}%
}

        \parbox{\linewidth}{
        \vspace{0.5em}
        \renewcommand{\baselinestretch}{1}
        \small
        \raggedright
        \textit{Note.} A Mann-Whitney test was performed to compare the scores of Y0 and Y2. Mean age: Y0 (cross-sectional) = 9.96 years, Y2 (longitudinal) = 11.94 years
        }
\end{table}


The overlap between high threat and high deprivation groups is an important consideration, as it reflects individuals with mixed exposure. As shown in Table~\ref{tab:overlap}, at Y0, 2.92\% of participants (N = 104) belonged to both high-threat and high-deprivation groups, a proportion that remained similar at Y2 (N = 56, 2.51\%). Conversely, the majority of participants were classified as low threat and low deprivation, comprising 80.49\% (N = 2,872) at Y0 and 80.02\% (N = 1,778) at Y2.

To address the overlap, statistical controls were applied to account for the effects of each dimension while retaining participants with mixed exposure. This approach ensured a sufficient number of participants for analysis while disentangling the unique contributions of threat and deprivation.

\begin{table}[h]
\caption{Overlap of Threat and Deprivation Groups}
\label{tab:overlap}
\resizebox{\columnwidth}{!}{%
\begin{tabular}{@{}lcccccc@{}}
\toprule
 & \multicolumn{3}{c}{Y0 {[}N (\%){]}}        & \multicolumn{3}{c}{Y2 {[}N (\%){]}}        \\ \cmidrule(l){2-7} 
 & Low deprivation & High deprivation & Total & Low deprivation & High deprivation & Total \\ \midrule
Low threat  & 2,872 (80.49\%) & 224 (6.28\%) & 3,096 (86.77\%) & 1,778 (80.02\%) & 149 (6.71\%) & 1,927 (86.73\%) \\
High threat & 368 (10.31\%)   & 104 (2.92\%) & 472 (13.23\%)   & 239 (10.76\%)   & 56 (2.51\%)  & 295 (13.27\%)   \\
Total       & 3,240 (90.80\%) & 328 (9.20\%) & 3,568 (100\%)   & 2,017 (90.78)   & 205 (9.22\%) & 2,222 (100\%)   \\ \bottomrule
\end{tabular}%
}

        \parbox{\linewidth}{
        \vspace{0.5em}
        \renewcommand{\baselinestretch}{1}
        \small
        \raggedright
        \textit{Note.} Mean age: Y0 (cross-sectional) = 9.96 years, Y2 (longitudinal) = 11.94 years
        }
\end{table}



% \begin{table}[]
%     \centering
%     \caption{Distribution of threat and deprivation}
%     \label{tab:els_distribution}
%     \includegraphics[width=1\linewidth]{els_distribution.png}

%         \parbox{\linewidth}{
%         \renewcommand{\baselinestretch}{1}
%         \small
%         \raggedright
%         \textit{Note.} A Mann-Whitney test was performed to compare the scores of Y0 and Y2.
%         }
% \end{table}

Table~\ref{tab:y0_threat} and Table~\ref{tab:y0_depriv} summarize the demographics and covariates for high-threat and low-threat groups and for high-deprivation and low-deprivation groups at Y0. Age and gender did not significantly differ between high-threat and low-threat groups (\textit{t}(624.24) = -1.92, \textit{p} = .056 for age; \textit{p} = .882 for gender) or between high-deprivation and low-deprivation groups (\textit{t}(394.11) = -1.97, \textit{p} = .050 for age; \textit{p} = 1.000 for gender).



\begin{table}[]
\caption{Descriptive statistics and group comparison (high vs. low threat) at Y0 (mean age = 9.96)}
\label{tab:y0_threat}
\scriptsize
\renewcommand{\arraystretch}{0.8} % 행 높이를 더 줄임
\setlength{\tabcolsep}{4pt} % 열 간격을 더 줄임
\begin{adjustbox}{max width=\textwidth, center} % 표 중앙 정렬
% \begin{tabular}{@{}lcllllll@{}}

\begin{tabular}{@{}lcllll>{\raggedleft\arraybackslash}ll@{}}

\toprule 
 &
  \multicolumn{4}{c}{Y0 (N = 3,568)} &
   &
   &
   \\ \cmidrule(l){2-8} 
 &
  \multicolumn{2}{c}{High threat group (N = 472)} &
  \multicolumn{2}{c}{Low threat group (N = 3,096)} &
  \multicolumn{1}{c}{\multirow{2}{*}{t}} &
  \multicolumn{1}{r}{\multirow{2}{*}{p}} &
   \\ \cmidrule(lr){2-5}
 &
  N (\%) &
  \multicolumn{1}{c}{Mean (SD)} &
  \multicolumn{1}{c}{N (\%)} &
  \multicolumn{1}{c}{Mean (SD)} &
  \multicolumn{1}{c}{} &
  \multicolumn{1}{r}{} &
   \\ \midrule
Child age (years) &
  \multicolumn{1}{l}{} &
  \multicolumn{1}{c}{9.91	(0.62)} &
   &
  \multicolumn{1}{c}{9.97	(0.63)} &
  \multicolumn{1}{c}{-1.92} &
  \multicolumn{1}{r}{.056} &
   \\
Gender &
  \multicolumn{1}{l}{} &
   &
   &
   &
   &
  \multicolumn{1}{r}{.882} &
   \\
\hspace{1em}Male &
  243 (51\%) &
   &
  \multicolumn{1}{c}{1,580 (51\%)} &
   &
   &
   &
   \\
\hspace{1em}Female &
  229 (49\%) &
   &
  \multicolumn{1}{c}{1,516 (49\%)} &
   &
   &
   &
   \\
Race and Ethnicity &
  \multicolumn{1}{l}{} &
   &
   &
   &
   &
  \textless{}.001 &
  *** \\
\hspace{1em}White &
  255 (54\%) &
   &
  \multicolumn{1}{c}{1,993 (64\%)} &
   &
   &
   &
   \\
\hspace{1em}Black &
  70 (15\%) &
   &
  \multicolumn{1}{c}{189 (6.1\%)} &
   &
   &
   &
   \\
\hspace{1em}Hispanic &
  88 (19\%) &
   &
  \multicolumn{1}{c}{515 (17\%)} &
   &
   &
   &
   \\
\hspace{1em}Asian &
  7 (1.5\%) &
   &
  \multicolumn{1}{c}{58 (1.9\%)} &
   &
   &
   &
   \\
\hspace{1em}Other &
  52 (11\%) &
   &
  \multicolumn{1}{c}{341 (11\%)} &
   &
   &
   &
   \\
Maternal age at child's birth &
  \multicolumn{1}{l}{} &
  \multicolumn{1}{c}{28.49 (5.97)} &
   &
  \multicolumn{1}{c}{30.60 (5.69)} &
  \multicolumn{1}{c}{−7.22} &
  \multicolumn{1}{r}{\textless{}.001} &
  *** \\
Parental education &
  \multicolumn{1}{l}{} &
  \multicolumn{1}{c}{16.42 (2.71)} &
   &
  \multicolumn{1}{c}{17.38 (2.27)} &
  \multicolumn{1}{c}{−7.32} &
  \multicolumn{1}{r}{\textless{}.001} &
  *** \\
Birth weight (pound) &
  \multicolumn{1}{l}{} &
  \multicolumn{1}{c}{6.69 (1.40)} &
   &
  \multicolumn{1}{c}{6.74 (1.41)} &
  \multicolumn{1}{c}{−0.72} &
  \multicolumn{1}{r}{.047} & *
   \\
Premature birth &
  \multicolumn{1}{l}{} &
   &
   &
   &
   &
  \multicolumn{1}{r}{.599} &
   \\
\hspace{1em}Yes &
  76 (16\%) &
   &
  \multicolumn{1}{c}{531 (17\%)} &
   &
   &
   &
   \\
\hspace{1em}No &
  396 (84\%) &
   &
  \multicolumn{1}{c}{2,565 (83\%)} &
   &
   &
   &
   \\
Cyanosis (blue at birth) &
  \multicolumn{1}{l}{} &
   &
   &
   &
   &
  \multicolumn{1}{r}{.892} &
   \\
\hspace{1em}Yes &
  17 (3.6\%) &
   &
  \multicolumn{1}{c}{107 (3.5\%)} &
   &
   &
   &
   \\
\hspace{1em}No &
  455 (96\%) &
   &
  \multicolumn{1}{c}{2,989 (97\%)} &
   &
   &
   &
   \\
Pregnant alcohol use/smoking &
  \multicolumn{1}{l}{} &
   &
   &
   &
   &
  \textless{}.001 &
  *** \\
\hspace{1em}Yes &
  51 (11\%) &
   &
  \multicolumn{1}{c}{159 (5.1\%)} &
   &
   &
   &
   \\
\hspace{1em}No &
  421 (89\%) &
   &
  \multicolumn{1}{c}{2,937 (95\%)} &
   &
   &
   &
   \\
Parent anxiety problems &
  \multicolumn{1}{l}{} &
  \multicolumn{1}{c}{54.72 (6.04)} &
   &
  \multicolumn{1}{c}{53.07 (5.01)} &
  \multicolumn{1}{c}{5.62} &
  \multicolumn{1}{r}{\textless{}.001} &
  *** \\
Parent somatic problems &
  \multicolumn{1}{l}{} &
  \multicolumn{1}{c}{56.97 (7.57)} &
   &
  \multicolumn{1}{c}{53.91 (5.52)} &
  \multicolumn{1}{c}{8.43} &
  \multicolumn{1}{r}{\textless{}.001} &
  *** \\
Parent depressive problems &
  \multicolumn{1}{l}{} &
  \multicolumn{1}{c}{56.26 (7.22)} &
   &
  \multicolumn{1}{c}{53.43 (5.33)} &
  \multicolumn{1}{c}{8.18} &
  \multicolumn{1}{r}{\textless{}.001} &
  *** \\
Parent avoidant personality problems &
  \multicolumn{1}{l}{} &
  \multicolumn{1}{c}{54.79 (6.92)} &
   &
  \multicolumn{1}{c}{52.72 (4.71)} &
  \multicolumn{1}{c}{6.26} &
  \multicolumn{1}{r}{\textless{}.001} &
  *** \\
Parent AD/H problems &
  \multicolumn{1}{l}{} &
  \multicolumn{1}{c}{54.88 (6.90)} &
   &
  \multicolumn{1}{c}{52.69 (4.90)} &
  \multicolumn{1}{c}{6.64} &
  \multicolumn{1}{r}{\textless{}.001} &
  *** \\
Parent antisocial personality problems &
  \multicolumn{1}{l}{} &
  \multicolumn{1}{c}{54.64 (5.80)} &
   &
  \multicolumn{1}{c}{52.55 (4.05)} &
  \multicolumn{1}{c}{7.54} &
  \multicolumn{1}{r}{\textless{}.001} &
  *** \\
Father drug problem &
  \multicolumn{1}{l}{} &
   &
   &
   &
   &
  \textless{}.001 &
  *** \\
\hspace{1em}Yes &
  114 (24\%) &
   &
  \multicolumn{1}{c}{410 (13\%)} &
   &
   &
   &
   \\
\hspace{1em}No &
  358 (76\%) &
   &
  \multicolumn{1}{c}{2,686 (87\%)} &
   &
   &
   &
   \\
Mother drug problem &
  \multicolumn{1}{l}{} &
   &
   &
   &
   &
  \textless{}.001 &
  *** \\
\hspace{1em}Yes &
  42 (8.9\%) &
   &
  \multicolumn{1}{c}{118 (3.8\%)} &
   &
   &
   &
   \\
\hspace{1em}No &
  430 (91\%) &
   &
  \multicolumn{1}{c}{2,978 (96\%)} &
   &
   &
   &
   \\
Child externalizing problems &
  \multicolumn{1}{l}{} &
  \multicolumn{1}{c}{48.99 (10.87)} &
   &
  \multicolumn{1}{c}{43.54 (9.00)} &
  \multicolumn{1}{c}{10.37} &
  \multicolumn{1}{r}{\textless{}.001} &
  *** \\
Child internalizing problems &
  \multicolumn{1}{l}{} &
  \multicolumn{1}{c}{51.79 (11.26)} &
   &
  \multicolumn{1}{c}{47.31 (9.80)} &
  \multicolumn{1}{c}{8.20} &
  \multicolumn{1}{r}{\textless{}.001} &
  *** \\ \bottomrule
\end{tabular}
\end{adjustbox}
    \parbox{\linewidth}{
            \vspace{0.5em}
        \renewcommand{\baselinestretch}{1}
        \small
        \raggedright
    \textit{Note.} * p < .05, ** p < .01, *** p < .001 \\
    A Fisher's exact test was performed to assess group differences in categorical variables. A Welch's t-test was used for continuous variables. 
    }
\end{table}

% \begin{table}[]
%     \centering
%     \caption{Descriptive statistics and group comparison (high vs. low threat) at Y0}
%     \label{tab:y0_threat}
%     \includegraphics[width=1\linewidth]{y0_threat.png}
%     \renewcommand{\baselinestretch}{1}
%     \small
%     \raggedright
%     \parbox{\linewidth}{
%     \vspace{0.5em}
%     \textit{Note.} * p < .05, ** p < .01, *** p < .001 \\
%     A Fisher's exact test was performed to assess group differences in categorical variables. A Welch's t-test was used for continuous variables. 
%     }
% \end{table}


\begin{table}[]
\caption{Descriptive statistics and group comparison (high vs. low deprivation) at Y0 (mean age = 9.96)}
\label{tab:y0_depriv}
\scriptsize
\renewcommand{\arraystretch}{0.8} % 행 높이를 더 줄임
\setlength{\tabcolsep}{4pt} % 열 간격을 더 줄임
\begin{adjustbox}{max width=\textwidth, center} % 표 중앙 정렬
% \begin{tabular}{@{}lcllllll@{}}

\begin{tabular}{@{}lcllll>{\raggedleft\arraybackslash}ll@{}}
\toprule
 &
  \multicolumn{4}{c}{Y0 (N = 3,568)} &
   &
   &
   \\ \cmidrule(l){2-8} 
 &
  \multicolumn{2}{c}{High deprivation group (N = 328)} &
  \multicolumn{2}{c}{Low deprivation group (N = 3,240)} &
  \multicolumn{1}{c}{\multirow{2}{*}{t}} &
  \multicolumn{1}{r}{\multirow{2}{*}{p}} &
   \\ \cmidrule(lr){2-5}
 &
  N (\%) &
  \multicolumn{1}{c}{Mean (SD)} &
  \multicolumn{1}{c}{N (\%)} &
  \multicolumn{1}{c}{Mean (SD)} &
  \multicolumn{1}{c}{} &
  \multicolumn{1}{r}{} &
   \\ \midrule
Child age (years) &
  \multicolumn{1}{l}{} &
  \multicolumn{1}{c}{9.90 (0.63)} &
   &
  \multicolumn{1}{c}{9.97 (0.62)} &
  \multicolumn{1}{c}{−1.97} &
  \multicolumn{1}{r}{.050} &
   \\
Gender &
  \multicolumn{1}{l}{} &
   &
   &
   &
   &
  \multicolumn{1}{r}{1.000} &
   \\
\hspace{1em}Male &
  168 (51\%) &
   &
  \multicolumn{1}{c}{1,655 (51\%)} &
   &
   &
   &
   \\
\hspace{1em}Female &
  160 (49\%) &
   &
  \multicolumn{1}{c}{1,585 (49\%)} &
   &
   &
   &
   \\
Race and Ethnicity &
  \multicolumn{1}{l}{} &
   &
   &
   &
   &
  \textless{}.001 &
  *** \\
\hspace{1em}White &
  119 (36\%) &
   &
  \multicolumn{1}{c}{2,129 (66\%)} &
   &
   &
   &
   \\
\hspace{1em}Black &
  71 (22\%) &
   &
  \multicolumn{1}{c}{188 (5.8\%)} &
   &
   &
   &
   \\
\hspace{1em}Hispanic &
  89 (27\%) &
   &
  \multicolumn{1}{c}{514 (16\%)} &
   &
   &
   &
   \\
\hspace{1em}Asian &
  2 (0.6\%) &
   &
  \multicolumn{1}{c}{63 (1.9\%)} &
   &
   &
   &
   \\
\hspace{1em}Other &
  47 (14\%) &
   &
  \multicolumn{1}{c}{346 (11\%)} &
   &
   &
   &
   \\
Maternal age at child's birth &
  \multicolumn{1}{l}{} &
  \multicolumn{1}{c}{26.72 (5.96)} &
   &
  \multicolumn{1}{c}{30.69 (5.63)} &
  \multicolumn{1}{c}{−11.56} &
  \multicolumn{1}{r}{\textless{}.001} &
  *** \\
Parental education &
  \multicolumn{1}{l}{} &
  \multicolumn{1}{c}{15.13 (2.77)} &
   &
  \multicolumn{1}{c}{17.46 (2.20)} &
  \multicolumn{1}{c}{−14.76} &
  \multicolumn{1}{r}{\textless{}.001} &
  *** \\
Birth weight (pound) &
  \multicolumn{1}{l}{} &
  \multicolumn{1}{c}{6.56 (1.38)} &
   &
  \multicolumn{1}{c}{6.75 (1.41)} &
  \multicolumn{1}{c}{−2.33} &
  \multicolumn{1}{r}{.020} &
  * \\
Premature birth &
  \multicolumn{1}{l}{} &
   &
   &
   &
   &
  \multicolumn{1}{r}{.397} &
   \\
\hspace{1em}Yes &
  50 (15\%) &
   &
  \multicolumn{1}{c}{557 (17\%)} &
   &
   &
   &
   \\
\hspace{1em}No &
  278 (85\%) &
   &
  \multicolumn{1}{c}{2,683 (83\%)} &
   &
   &
   &
   \\
Cyanosis (blue at birth) &
  \multicolumn{1}{l}{} &
   &
   &
   &
   &
  \multicolumn{1}{r}{.204} &
   \\
\hspace{1em}Yes &
  7 (2.1\%) &
   &
  \multicolumn{1}{c}{117 (3.6\%)} &
   &
   &
   &
   \\
\hspace{1em}No &
  321 (98\%) &
   &
  \multicolumn{1}{c}{3,123 (96\%)} &
   &
   &
   &
   \\
Pregnant alcohol use/smoking &
  \multicolumn{1}{l}{} &
   &
   &
   &
   &
  \textless{}.001 &
  *** \\
\hspace{1em}Yes &
  42 (13\%) &
   &
  \multicolumn{1}{c}{168 (5.2\%)} &
   &
   &
   &
   \\
\hspace{1em}No &
  286 (87\%) &
   &
  \multicolumn{1}{c}{3,072 (95\%)} &
   &
   &
   &
   \\
Parent anxiety problems &
  \multicolumn{1}{l}{} &
  \multicolumn{1}{c}{55.25 (6.91)} &
   &
  \multicolumn{1}{c}{53.09 (4.93)} &
  \multicolumn{1}{c}{5.51} &
  \multicolumn{1}{r}{\textless{}.001} &
  *** \\
Parent somatic problems &
  \multicolumn{1}{l}{} &
  \multicolumn{1}{c}{57.20 (8.23)} &
   &
  \multicolumn{1}{c}{54.03 (5.56)} &
  \multicolumn{1}{c}{6.82} &
  \multicolumn{1}{r}{\textless{}.001} &
  *** \\
Parent depressive problems &
  \multicolumn{1}{l}{} &
  \multicolumn{1}{c}{56.15 (7.76)} &
   &
  \multicolumn{1}{c}{53.57 (5.38)} &
  \multicolumn{1}{c}{5.89} &
  \multicolumn{1}{r}{\textless{}.001} &
  *** \\
Parent avoidant personality problems &
  \multicolumn{1}{l}{} &
  \multicolumn{1}{c}{54.14 (6.57)} &
   &
  \multicolumn{1}{c}{52.88 (4.92)} &
  \multicolumn{1}{c}{3.38} &
  \multicolumn{1}{r}{\textless{}.001} &
  *** \\
Parent AD/H problems &
  \multicolumn{1}{l}{} &
  \multicolumn{1}{c}{54.73 (7.57)} &
   &
  \multicolumn{1}{c}{52.80 (4.94)} &
  \multicolumn{1}{c}{4.52} &
  \multicolumn{1}{r}{\textless{}.001} &
  *** \\
Parent antisocial personality problems &
  \multicolumn{1}{l}{} &
  \multicolumn{1}{c}{54.61 (5.74)} &
   &
  \multicolumn{1}{c}{52.64 (4.18)} &
  \multicolumn{1}{c}{6.04} &
  \multicolumn{1}{r}{\textless{}.001} &
  *** \\
Father drug problem &
  \multicolumn{1}{l}{} &
   &
   &
   &
   &
  \textless{}.001 &
  *** \\
\hspace{1em}Yes &
  121 (37\%) &
   &
  \multicolumn{1}{c}{403 (12\%)} &
   &
   &
   &
   \\
\hspace{1em}No &
  207 (63\%) &
   &
  \multicolumn{1}{c}{2,837 (88\%)} &
   &
   &
   &
   \\
Mother drug problem &
  \multicolumn{1}{l}{} &
   &
   &
   &
   &
  \textless{}.001 &
  *** \\
\hspace{1em}Yes &
  39 (12\%) &
   &
  \multicolumn{1}{c}{121 (3.7\%)} &
   &
   &
   &
   \\
\hspace{1em}No &
  289 (88\%) &
   &
  \multicolumn{1}{c}{3,119 (96\%)} &
   &
   &
   &
   \\
Child externalizing problems &
  \multicolumn{1}{l}{} &
  \multicolumn{1}{c}{48.53 (11.83)} &
   &
  \multicolumn{1}{c}{43.83 (9.07)} &
  \multicolumn{1}{c}{6.99} &
  \multicolumn{1}{r}{\textless{}.001} &
  *** \\
Child internalizing problems &
  \multicolumn{1}{l}{} &
  \multicolumn{1}{c}{50.15 (11.16)} &
   &
  \multicolumn{1}{c}{47.67 (9.98)} &
  \multicolumn{1}{c}{3.86} &
  \multicolumn{1}{r}{\textless{}.001} &
  *** \\ \bottomrule
\end{tabular}
\end{adjustbox}
    \parbox{\linewidth}{
            \vspace{0.5em}
        \renewcommand{\baselinestretch}{1}
        \small
        \raggedright
    \textit{Note.} * p < .05, ** p < .01, *** p < .001 \\
    A Fisher's exact test was performed to assess group differences in categorical variables. A Welch's t-test was used for continuous variables. 
    }
\end{table}

% \begin{table}[]
%     \centering
%     \caption{Descriptive statistics and group comparison (high vs. low deprivation) at Y0}
%     \label{tab:y0_depriv}
%     \includegraphics[width=1\linewidth]{y0_depriv.png}

%     \parbox{\linewidth}{
%         \renewcommand{\baselinestretch}{1}
%         \small
%         \raggedright
%         \textit{Note.} * p < .05, ** p < .01, *** p < .001 \\
%         A Fisher's exact test was performed to assess group differences in categorical variables. A Welch's t-test was used for continuous variables. 
%         }
% \end{table}




\clearpage
For Y2, refer to Table~\ref{tab:y2_threat} and Table~\ref{tab:y2_depriv}. Age and gender did not significantly differ between the high-threat and low-threat groups (\textit{t}(394.86) = -1.47, \textit{p} = .435 for age; \textit{p} = .348 for gender). However, a significant age difference was observed between high-deprivation and low-deprivation groups (\textit{t}(242.65) = -2.21, \textit{p} = .028), although gender differences were not significant (\textit{p} = .077).

Adversity before birth and parental psychopathologies significantly differed across both threat and deprivation groups, indicating that these factors cannot be easily categorized as exclusively threat- or deprivation-related. Similarly, child externalizing and internalizing problems showed significant differences across both groups. These factors were therefore included as covariates to control for potential confounding effects and to isolate the unique effects of threat and deprivation on neurocognitive functions.


% Please add the following required packages to your document preamble:
% \usepackage{booktabs}
% \usepackage{multirow}
% \usepackage{graphicx}
\begin{table}[]
\caption{Descriptive statistics and group comparison (high vs. low threat) at Y2 (mean age = 11.94)}
\label{tab:y2_threat}
\scriptsize
\renewcommand{\arraystretch}{0.8} % 행 높이를 더 줄임
\setlength{\tabcolsep}{4pt} % 열 간격을 더 줄임
\begin{adjustbox}{max width=\textwidth, center} % 표 중앙 정렬
% \begin{tabular}{@{}lcllllll@{}}

\begin{tabular}{@{}lcllll>{\raggedleft\arraybackslash}ll@{}}
\toprule
 &
  \multicolumn{4}{c}{Y2 (N = 2,222)} &
   &
   &
   \\ \cmidrule(l){2-8} 
 &
  \multicolumn{2}{c}{High threat group (N = 295)} &
  \multicolumn{2}{c}{Low threat group (N = 1,927)} &
  \multicolumn{1}{c}{\multirow{2}{*}{t}} &
  \multicolumn{1}{r}{\multirow{2}{*}{p}} &
  \multirow{2}{*}{} \\ \cmidrule(lr){2-5}
 &
  \multicolumn{1}{c}{N (\%)} &
  \multicolumn{1}{c}{Mean (SD)} &
  \multicolumn{1}{c}{N (\%)} &
  \multicolumn{1}{c}{Mean (SD)} &
  \multicolumn{1}{c}{} &
  \multicolumn{1}{r}{} &
   \\ \midrule
Child age (years) &
   &
  \multicolumn{1}{c}{11.89 (0.62)} &
   &
  \multicolumn{1}{c}{11.95 (0.63)} &
  \multicolumn{1}{c}{−1.47} &
  \multicolumn{1}{r}{1.435} &
   \\
Gender &
   &
   &
   &
   &
   &
  \multicolumn{1}{r}{.348} &
   \\
\hspace{1em}Male &
  \multicolumn{1}{c}{145 (49\%)} &
   &
  \multicolumn{1}{c}{1,005 (52\%)} &
   &
   &
   &
   \\
\hspace{1em}Female &
  \multicolumn{1}{c}{150 (51\%)} &
   &
  \multicolumn{1}{c}{922 (48\%)} &
   &
   &
   &
   \\
Race and Ethnicity &
   &
   &
   &
   &
   &
  \textless{}.001 &
  *** \\
\hspace{1em}White &
  \multicolumn{1}{c}{162 (55\%)} &
   &
  \multicolumn{1}{c}{1,302 (68\%)} &
   &
   &
   &
   \\
\hspace{1em}Black &
  \multicolumn{1}{c}{43 (15\%)} &
   &
  \multicolumn{1}{c}{99 (5.1\%)} &
   &
   &
   &
   \\
\hspace{1em}Hispanic &
  \multicolumn{1}{c}{48 (16\%)} &
   &
  \multicolumn{1}{c}{316 (16\%)} &
   &
   &
   &
   \\
\hspace{1em}Asian &
  \multicolumn{1}{c}{4 (1.4\%)} &
   &
  \multicolumn{1}{c}{32 (1.7\%)} &
   &
   &
   &
   \\
\hspace{1em}Other &
  \multicolumn{1}{c}{38 (13\%)} &
   &
  \multicolumn{1}{c}{178 (9.2\%)} &
   &
   &
   &
   \\
Maternal age at child's birth &
   &
  \multicolumn{1}{c}{28.54 (6.03)} &
   &
  \multicolumn{1}{c}{30.61 (5.66)} &
  \multicolumn{1}{c}{−5.55} &
  \multicolumn{1}{r}{\textless{}.001} &
  *** \\
Parental education &
   &
  \multicolumn{1}{c}{16.47 (2.65)} &
   &
  \multicolumn{1}{c}{17.35 (2.27)} &
  \multicolumn{1}{c}{−5.39} &
  \multicolumn{1}{r}{\textless{}.001} &
  *** \\
Birth weight (pound) &
   &
  \multicolumn{1}{c}{6.68 (1.39)} &
   &
  \multicolumn{1}{c}{6.75 (1.42)} &
  \multicolumn{1}{c}{−0.75} &
  \multicolumn{1}{r}{.452} &
   \\
Premature birth &
   &
   &
   &
   &
   &
  \multicolumn{1}{r}{.187} &
   \\
\hspace{1em}Yes &
  \multicolumn{1}{c}{43 (15\%)} &
   &
  \multicolumn{1}{c}{342 (18\%)} &
   &
   &
   &
   \\
\hspace{1em}No &
  \multicolumn{1}{c}{252 (85\%)} &
   &
  \multicolumn{1}{c}{1,585 (82\%)} &
   &
   &
   &
   \\
Cyanosis (blue at birth) &
   &
   &
   &
   &
   &
  \multicolumn{1}{r}{.737} &
   \\
\hspace{1em}Yes &
  \multicolumn{1}{c}{9 (3.1\%)} &
   &
  \multicolumn{1}{c}{71 (3.7\%)} &
   &
   &
   &
   \\
\hspace{1em}No &
  \multicolumn{1}{c}{286 (97\%)} &
   &
  \multicolumn{1}{c}{1,856 (96\%)} &
   &
   &
   &
   \\
Pregnant alcohol use/smoking &
   &
   &
   &
   &
   &
  \multicolumn{1}{r}{.004} &
  ** \\
\hspace{1em}Yes &
  26 (8.8\%) &
   &
  88 (4.6\%) &
   &
   &
   &
   \\
\hspace{1em}No &
  \multicolumn{1}{c}{269 (91\%)} &
   &
  \multicolumn{1}{c}{1,839 (95\%)} &
   &
   &
   &
   \\
Parent anxiety problems &
   &
  \multicolumn{1}{c}{54.86 (5.80)} &
   &
  \multicolumn{1}{c}{52.92 (4.87)} &
  \multicolumn{1}{c}{5.46} &
  \multicolumn{1}{r}{\textless{}.001} &
  *** \\
Parent somatic problems &
   &
  \multicolumn{1}{c}{56.55 (7.55)} &
   &
  \multicolumn{1}{c}{54.09 (5.71)} &
  \multicolumn{1}{c}{5.36} &
  \multicolumn{1}{r}{\textless{}.001} &
  *** \\
Parent depressive problems &
   &
  \multicolumn{1}{c}{56.34 (7.23)} &
   &
  \multicolumn{1}{c}{53.44 (5.40)} &
  \multicolumn{1}{c}{6.6} &
  \multicolumn{1}{r}{\textless{}.001} &
  *** \\
Parent avoidant personality problems &
   &
  \multicolumn{1}{c}{55.03 (6.87)} &
   &
  \multicolumn{1}{c}{52.80 (4.79)} &
  \multicolumn{1}{c}{5.39} &
  \multicolumn{1}{r}{\textless{}.001} &
  *** \\
Parent AD/H problems &
   &
  \multicolumn{1}{c}{55.21 (6.97)} &
   &
  \multicolumn{1}{c}{52.78 (5.01)} &
  \multicolumn{1}{c}{5.77} &
  \multicolumn{1}{r}{\textless{}.001} &
  *** \\
Parent antisocial personality problems &
   &
  \multicolumn{1}{c}{54.64 (5.90)} &
   &
  \multicolumn{1}{c}{52.54 (4.01)} &
  \multicolumn{1}{c}{5.89} &
  \multicolumn{1}{r}{\textless{}.001} &
  *** \\
Father drug problem &
   &
   &
   &
   &
   &
  \textless{}.001 &
  *** \\
\hspace{1em}Yes &
  \multicolumn{1}{c}{68 (23\%)} &
   &
  \multicolumn{1}{c}{249 (13\%)} &
   &
   &
   &
   \\
\hspace{1em}No &
  \multicolumn{1}{c}{227 (77\%)} &
   &
  \multicolumn{1}{c}{1,678 (87\%)} &
   &
   &
   &
   \\
Mother drug problem &
   &
   &
   &
   &
   &
  \multicolumn{1}{r}{.027} &
  * \\
\hspace{1em}Yes &
  \multicolumn{1}{c}{20 (6.8\%)} &
   &
  \multicolumn{1}{c}{73 (3.8\%)} &
   &
   &
   &
   \\
\hspace{1em}No &
  \multicolumn{1}{c}{275 (93\%)} &
   &
  \multicolumn{1}{c}{1,854 (96\%)} &
   &
   &
   &
   \\
Child externalizing problems &
   &
  \multicolumn{1}{c}{46.49 (10.32)} &
   &
  \multicolumn{1}{c}{43.18 (8.84)} &
  \multicolumn{1}{c}{5.22} &
  \multicolumn{1}{r}{\textless{}.001} &
  *** \\
Child internalizing problems &
   &
  \multicolumn{1}{c}{50.91 (11.16)} &
   &
  \multicolumn{1}{c}{47.01 (9.98)} &
  \multicolumn{1}{c}{5.66} &
  \multicolumn{1}{r}{\textless{}.001} &
  *** \\ \bottomrule
\end{tabular}
\end{adjustbox}
    \parbox{\linewidth}{
            \vspace{0.5em}
        \renewcommand{\baselinestretch}{1}
        \small
        \raggedright
    \textit{Note.} * p < .05, ** p < .01, *** p < .001 \\
    A Fisher's exact test was performed to assess group differences in categorical variables. A Welch's t-test was used for continuous variables. 
    }
\end{table}


% \begin{table}[]
%     \centering
%     \caption{Descriptive statistics and group comparison (high vs. low threat) at Y2}
%     \label{tab:y2_threat}
%     \includegraphics[width=1\linewidth]{y2_threat.png}

%     \parbox{\linewidth}{    
%         \renewcommand{\baselinestretch}{1}
%         \small
%         \raggedright
%         \textit{Note.} * p < .05, ** p < .01, *** p < .001 \\
%         A Fisher's exact test was performed to assess group differences in categorical variables. A Welch's t-test was used for continuous variables. 
%         }
% \end{table}

% Please add the following required packages to your document preamble:
% \usepackage{booktabs}
% \usepackage{multirow}
% \usepackage{graphicx}
\begin{table}[]
\caption{Descriptive statistics and group comparison (high vs. low deprivation) at Y2 (mean age = 11.94)}
\label{tab:y2_depriv}
\scriptsize
\renewcommand{\arraystretch}{0.8} % 행 높이를 더 줄임
\setlength{\tabcolsep}{4pt} % 열 간격을 더 줄임
\begin{adjustbox}{max width=\textwidth, center} % 표 중앙 정렬
% \begin{tabular}{@{}lcllllll@{}}

\begin{tabular}{@{}lcllll>{\raggedleft\arraybackslash}ll@{}}
\toprule
 &
  \multicolumn{4}{c}{Y2 (N = 2,222)} &
   &
   &
   \\ \cmidrule(l){2-8} 
 &
  \multicolumn{2}{c}{High deprivation group (N = 205)} &
  \multicolumn{2}{c}{Low deprivation group (N = 2,017)} &
  \multicolumn{1}{c}{\multirow{2}{*}{t}} &
  \multicolumn{1}{r}{\multirow{2}{*}{p}} &
   \\ \cmidrule(lr){2-5}
 &
  \multicolumn{1}{c}{N (\%)} &
  \multicolumn{1}{c}{Mean (SD)} &
  \multicolumn{1}{c}{N (\%)} &
  \multicolumn{1}{c}{Mean (SD)} &
  \multicolumn{1}{c}{} &
  \multicolumn{1}{r}{} &
   \\ \midrule
Child age (years) &
   &
  \multicolumn{1}{c}{11.84 (0.66)} &
   &
  \multicolumn{1}{c}{11.95 (0.63)} &
  \multicolumn{1}{c}{−2.21} &
  \multicolumn{1}{r}{.028} &
  * \\
Gender &
   &
   &
   &
   &
   &
  \multicolumn{1}{r}{.077} &
   \\
\hspace{1em}Male &
  \multicolumn{1}{c}{104 (51\%)} &
   &
  \multicolumn{1}{c}{1,046 (52\%)} &
   &
   &
   &
   \\
\hspace{1em}Female &
   &
   &
   &
   &
   &
   &
   \\
Race and Ethnicity &
   &
   &
   &
   &
   &
  \textless{}.001 &
  *** \\
\hspace{1em}White &
  \multicolumn{1}{c}{84 (41\%)} &
   &
  \multicolumn{1}{c}{1,380 (68\%)} &
   &
   &
   &
   \\
\hspace{1em}Black &
  \multicolumn{1}{c}{42 (20\%)} &
   &
  \multicolumn{1}{c}{100 (5.0\%)} &
   &
   &
   &
   \\
\hspace{1em}Hispanic &
  \multicolumn{1}{c}{48 (23\%)} &
   &
  \multicolumn{1}{c}{316 (16\%)} &
   &
   &
   &
   \\
\hspace{1em}Asian &
  \multicolumn{1}{c}{0 (0\%)} &
   &
  \multicolumn{1}{c}{36 (1.8\%)} &
   &
   &
   &
   \\
\hspace{1em}Other &
  \multicolumn{1}{c}{31 (15\%)} &
   &
  \multicolumn{1}{c}{185 (9.2\%)} &
   &
   &
   &
   \\
Maternal age at child's birth &
   &
  \multicolumn{1}{c}{26.88 (6.42)} &
   &
  \multicolumn{1}{c}{30.69 (5.56)} &
  \multicolumn{1}{c}{−8.19} &
   &
   \\
Parental education &
   &
  \multicolumn{1}{c}{15.22 (3.07)} &
   &
  \multicolumn{1}{c}{17.44 (2.15)} &
  \multicolumn{1}{c}{−10.12} &
   &
   \\
Birth weight (pound) &
   &
  \multicolumn{1}{c}{6.67 (1.42)} &
   &
  \multicolumn{1}{c}{6.74 (1.41)} &
  \multicolumn{1}{c}{−0.69} &
  \multicolumn{1}{r}{.492} &
   \\
Premature birth &
   &
   &
   &
   &
   &
  \multicolumn{1}{r}{.561} &
   \\
\hspace{1em}Yes &
  32 (16\%) &
   &
  353 (18\%) &
   &
   &
   &
   \\
\hspace{1em}No &
  \multicolumn{1}{c}{173 (84\%)} &
   &
  \multicolumn{1}{c}{1,664 (82\%)} &
   &
   &
   &
   \\
Cyanosis (blue at birth) &
   &
   &
   &
   &
   &
  \multicolumn{1}{r}{.237} &
   \\
\hspace{1em}Yes &
  \multicolumn{1}{c}{4 (2.0\%)} &
   &
  \multicolumn{1}{c}{76 (3.8\%)} &
   &
   &
   &
   \\
\hspace{1em}No &
  \multicolumn{1}{c}{201 (98\%)} &
   &
  \multicolumn{1}{c}{1,941 (96\%)} &
   &
   &
   &
   \\
Pregnant alcohol use/smoking &
   &
   &
   &
   &
   &
  \textless{}.001 &
  *** \\
\hspace{1em}Yes &
  \multicolumn{1}{c}{28 (14\%)} &
   &
  \multicolumn{1}{c}{86 (4.3\%)} &
   &
   &
   &
   \\
\hspace{1em}No &
  \multicolumn{1}{c}{177 (86\%)} &
   &
  \multicolumn{1}{c}{1,931 (96\%)} &
   &
   &
   &
   \\
Parent anxiety problems &
   &
  \multicolumn{1}{c}{54.73 (6.22)} &
   &
  \multicolumn{1}{c}{53.02 (4.89)} &
  \multicolumn{1}{c}{3.82} &
  \multicolumn{1}{r}{\textless{}.001} &
  *** \\
Parent somatic problems &
   &
  \multicolumn{1}{c}{57.03 (8.15)} &
   &
  \multicolumn{1}{c}{54.15 (5.73)} &
  \multicolumn{1}{c}{4.93} &
  \multicolumn{1}{r}{\textless{}.001} &
  *** \\
Parent depressive problems &
   &
  \multicolumn{1}{c}{55.94 (7.57)} &
   &
  \multicolumn{1}{c}{53.61 (5.50)} &
  \multicolumn{1}{c}{4.29} &
  \multicolumn{1}{r}{\textless{}.001} &
  *** \\
Parent avoidant personality problems &
   &
  \multicolumn{1}{c}{54.18 (6.18)} &
   &
  \multicolumn{1}{c}{52.98 (5.04)} &
  \multicolumn{1}{c}{2.68} &
  \multicolumn{1}{r}{.008} &
  ** \\
Parent AD/H problems &
   &
  \multicolumn{1}{c}{54.62 (7.35)} &
   &
  \multicolumn{1}{c}{52.94 (5.11)} &
  \multicolumn{1}{c}{3.19} &
  \multicolumn{1}{r}{.002} &
  ** \\
Parent antisocial personality problems &
   &
  \multicolumn{1}{c}{54.32 (5.59)} &
   &
  \multicolumn{1}{c}{52.67 (4.20)} &
  \multicolumn{1}{c}{4.11} &
  \multicolumn{1}{r}{\textless{}.001} &
  *** \\
Father drug problem &
   &
   &
   &
   &
   &
  \textless{}.001 &
  *** \\
\hspace{1em}Yes &
  \multicolumn{1}{c}{79 (39\%)} &
   &
  \multicolumn{1}{c}{238 (12\%)} &
   &
   &
   &
   \\
\hspace{1em}No &
  \multicolumn{1}{c}{126 (61\%)} &
   &
  \multicolumn{1}{c}{1,779 (88\%)} &
   &
   &
   &
   \\
Mother drug problem, n (\%) &
   &
   &
   &
   &
   &
  \textless{}.001 &
  *** \\
\hspace{1em}Yes &
  \multicolumn{1}{c}{22 (11\%)} &
   &
  \multicolumn{1}{c}{71 (3.5\%)} &
   &
   &
   &
   \\
\hspace{1em}No &
  \multicolumn{1}{c}{183 (89\%)} &
   &
  \multicolumn{1}{c}{1,946 (96\%)} &
   &
   &
   &
   \\
Child externalizing problems &
   &
  \multicolumn{1}{c}{47.25 (10.67)} &
   &
  \multicolumn{1}{c}{43.25 (8.86)} &
  \multicolumn{1}{c}{5.19} &
  \multicolumn{1}{r}{\textless{}.001} &
  *** \\
Child internalizing problems &
   &
  \multicolumn{1}{c}{49.74 (10.75)} &
   &
  \multicolumn{1}{c}{47.30 (10.15)} &
  \multicolumn{1}{c}{3.11} &
  \multicolumn{1}{r}{.002} &
  ** \\ \bottomrule
\end{tabular}
\end{adjustbox}
    \parbox{\linewidth}{
            \vspace{0.5em}
        \renewcommand{\baselinestretch}{1}
        \small
        \raggedright
    \textit{Note.} * p < .05, ** p < .01, *** p < .001 \\
    A Fisher's exact test was performed to assess group differences in categorical variables. A Welch's t-test was used for continuous variables. 
    }
\end{table}



% \begin{table}[]
%     \centering
%     \caption{Descriptive statistics and group comparison (high vs. low deprivation) at Y2}
%     \label{tab:y2_depriv}
%     \includegraphics[width=1\linewidth]{y2_depriv.png}

%     \parbox{\linewidth}{
%         \renewcommand{\baselinestretch}{1}
%         \small
%         \raggedright
%         \textit{Note.} * p < .05, ** p < .01, *** p < .001 \\
%         A Fisher's exact test was performed to assess group differences in categorical variables. A Welch's t-test was used for continuous variables.
%         }
% \end{table}


\clearpage
\subsection{Generalized linear model results}
\subsubsection{Reward processing}
\textbf{Threat} Cross-sectional analyses revealed no significant neural activation during either reward anticipation or reward feedback. However, in the longitudinal analyses, threat was associated with deactivation in the left caudate (\textit{b} = -0.030, \textit{SE} = 0.014, \textit{p} = .029) and the left rACC (\textit{b} = -0.028, \textit{SE} = 0.013, \textit{p} = .031) during reward anticipation. For reward feedback, threat was positively related to activation in the left putamen (\textit{b} = 0.027, \textit{SE} = 0.013, \textit{p} = .046), right insula (\textit{b} = 0.028, \textit{SE} = 0.012, \textit{p} = .019), and left cACC (\textit{b} = 0.029, \textit{SE} = 0.014, \textit{p} = .032). None of these results, however, remained significant after FDR correction. 

Additionally, no behavioral or survey measures showed significant associations with threat.

\textbf{Deprivation} No significant brain activation was observed during reward anticipation. However, during the reward feedback phase, deprivation was cross-sectionally associated with deactivation in several striatal and cortical regions, including the bilateral caudate (left: \textit{b} = -0.051, \textit{SE} = 0.017, \textit{p} = .003; right: \textit{b} = -0.039, \textit{SE} = 0.017, \textit{p} = .024), bilateral putamen (left: \textit{b} = -0.034, \textit{SE} = 0.015, \textit{p} = .021; right: \textit{b} = -0.042, \textit{SE} = 0.015, \textit{p} = .004), left NAc (\textit{b} = -0.066, \textit{SE} = 0.020, \textit{p} = .043), left insula (\textit{b} = -0.030, \textit{SE} = 0.013, \textit{p} = .022), left rACC (\textit{b} = -0.039, \textit{SE} = 0.018, \textit{p} = .028), and bilateral cACC (left: \textit{b} = -0.049, \textit{SE} = 0.016, \textit{p} = .002; right: \textit{b} = -0.043, \textit{SE} = 0.015, \textit{p} = .005). After FDR correction, only deactivation in the left NAc remained significant (\textit{p}-FDR = .043).

Longitudinally, deprivation was linked to hyperactivation in the bilateral NAc (left: \textit{b} = 0.085, \textit{SE} = 0.029, \textit{p} = .004; right: \textit{b} = 0.078, \textit{SE} = 0.030, \textit{p} = .009), right pallidum (\textit{b} = 0.032, \textit{SE} = 0.015, \textit{p} = .033), and left cACC (\textit{b} = 0.041, \textit{SE} = 0.017, \textit{p} = .014) during reward feedback.

For behavioral and survey measures, only the BAS Drive subscale was positively associated with deprivation (\textit{b} = 0.369, \textit{SE} = 0.175, \textit{p} = .034) in the cross-sectional analysis.

\textbf{Summary of findings} Reduced striatal activation during reward anticipation, as hypothesized, was observed for threat only in longitudinal analyses. During reward feedback, threat was associated with hyperactivation in cortical regions (insula, cACC) and the striatum (putamen). In contrast, deprivation showed no significant association with reward anticipation. Cross-sectionally, during reward feedback, deprivation was linked to hypoactivation in striatal and cortical regions, particularly in the ventral striatum (NAc), with this finding remaining significant after FDR correction. Longitudinally, deprivation was associated with increased activation in the striatum (NAc, pallidum) and cACC, demonstrating a pattern opposite to the cross-sectional findings. Among behavioral measures, only the BAS Drive subscale showed a significant positive association with deprivation. A summary of results for reward processing is provided in Table~\ref{tab:rew_summary_glm}.

\begin{table}[H]
\caption{Summary of GLM results for reward processing}
\label{tab:rew_summary_glm}
\scriptsize
\renewcommand{\arraystretch}{0.8} % 행 높이를 더 줄임
\setlength{\tabcolsep}{4pt} % 열 간격을 더 줄임
\begin{adjustbox}{max width=\textwidth, center} % 표 중앙 정렬
\begin{tabular}{@{}lcllcccc@{}}
\toprule
\multicolumn{1}{c}{\multirow{2}{*}{Domain}} &
  \multicolumn{2}{c}{\multirow{2}{*}{Task or measurement}} &
  \multicolumn{1}{c}{\multirow{2}{*}{ROI}} &
  \multicolumn{2}{c}{Threat} &
  \multicolumn{2}{c}{Deprivation} \\ \cmidrule(l){5-8} 
\multicolumn{1}{c}{} &
  \multicolumn{2}{c}{} &
  \multicolumn{1}{c}{} &
  Y0 &
  Y2 &
  Y0 &
  Y2 \\ \hline
\multirow{11}{*}{Reward processing} &
 \multirow[t]{2}{*}{\raisebox{1.5ex}{MID fMRI}} &
    \raisebox{1.5ex}{\begin{tabular}[t]{@{}l@{}}Reward anticipation \\ (reward vs. neutral)\end{tabular}} &
  \raisebox{1.5ex}{\multirow[t]{7}{*}{\begin{tabular}[t]{@{}l@{}}ACC (rACC/cACC) \\ Insula\\ Striatum \\ (caudate/putamen/NAc) \\ Pallidum\end{tabular}}} &
  \raisebox{1.5ex}{N.S.} &
  \cellcolor[HTML]{CFE2F3}\begin{tabular}[c]{@{}c@{}}left caudate ↓\\ left rACC ↓\end{tabular} &

  \raisebox{1.5ex}{N.S.} &
  \raisebox{1.5ex}{N.S.} \\
 &
   &
  \begin{tabular}[t]{@{}l@{}}Reward feedback \\ positive vs. negative\end{tabular} &
   &

  N.S. &
  \cellcolor[HTML]{F4CCCC}\begin{tabular}[t]{@{}c@{}}left putamen ↑\\ right insula ↑\\ left cACC ↑\end{tabular} &
  \cellcolor[HTML]{CFE2F3}\begin{tabular}[t]{@{}c@{}}bilateral caudate ↓\\ bilateral putamen ↓\\ \textbf{left NAc ↓} \\ left insula ↓\\ left rACC ↓\\ bilateral cACC ↓\end{tabular} &
  \cellcolor[HTML]{F4CCCC}\begin{tabular}[t]{@{}c@{}}bilateral NAc ↑\\ right pallidum ↑\\ left cACC ↑\end{tabular} \\
 &
\raisebox{0.5ex}{\multirow{2}{*}{%
\begingroup
\renewcommand{\arraystretch}{0.7} % 줄 바꿈 간격 조정
\begin{tabular}[t]{@{}c@{}}MID fMRI\\ Behavior\end{tabular}%
\endgroup}} &
  Accuracy (reward) &
   &
  N.S. &
  N.S. &
  N.S. &
  N.S. \\
 &
   &
  RT (Successful reward) &
   &
  N.S. &
  N.S. &
  N.S. &
  N.S. \\
 &
  \multirow[t]{3}{*}{BAS} &
  Reward responsiveness &
   &
  N.S. &
  N.S. &
  N.S. &
  N.S. \\
 &
   &
  Drive &
   &
  N.S. &
  N.S. &
 \cellcolor[HTML]{F4CCCC} ↑ &
  N.S. \\
 &
   &
  Fun seeking &
   &
  N.S. &
  N.S. &
  N.S. &
  N.S. \\ \bottomrule
\end{tabular}
\end{adjustbox}
    \parbox{\linewidth}{
            \vspace{0.5em}
        \renewcommand{\baselinestretch}{1}
        \small
        % \raggedright
        \textit{Note.} Written in bold means significance after FDR correction. Mean age: Y0 (cross-sectional) = 9.96 years, Y2 (longitudinal) = 11.94 years. \\
        \textit{Abbreviations.} N.S., Not significant; ROI, Regions of Interest; MID, Monetary Incentive Delay task; BAS, Behavioral Activation System; RT, Reaction Time; ACC, anterior cingulate cortex; rACC, rostral anterior cingulate cortex; cACC, caudal anterior cingulate cortex; NAc, nucleus accumbens
    }
\end{table}

% \begin{table}[H]
%     \centering
%     \caption{Summary of GLM results for reward processing}
%     \label{tab:rew_summary_glm}
%     \includegraphics[width=1\linewidth]{rew_summary_glm.png}
%         \renewcommand{\baselinestretch}{1}
%         \small
%     \parbox{\linewidth}{
%         \textit{Note.} Written in bold means significance after FDR correction \\
%         \textit{Abbreviations.} N.S., Not significant; ROI, Regions of Interest; MID, Monetary Incentive Delay task; BAS, Behavioral Activation System; RT, Reaction Time; ACC, anterior cinculate cortex; rACC, rostral anterior cingulate cortex; cACC, caudal anterior cingulate cortex; NAc, nucleus accumbens
%         }
% \end{table}

\subsubsection{Emotion processing}
\textbf{Threat} Cross-sectional analyses indicated that threat was associated with deactivation in the left putamen (\textit{b} = -0.040, \textit{SE} = 0.017, \textit{p} = .020) and bilateral hippocampus (left: \textit{b} = -0.037, \textit{SE} = 0.016, \textit{p} = .026; right: \textit{b} = -0.038, \textit{SE} = 0.017, \textit{p} = .023) in the positive vs. neutral contrast. Longitudinally, threat was associated with deactivation in the left putamen (\textit{b} = -0.056, \textit{SE} = 0.027, \textit{p} = .035).

For the negative vs. neutral contrast, no significant activation was observed cross-sectionally. Longitudinal analyses revealed hyperactivation in the right NAc related to threat (\textit{b} = 0.094, \textit{SE} = 0.046, \textit{p} = .043). However, none of these findings remained significant after FDR correction.

Reaction time (RT) and accuracy showed no significant associations with threat in either cross-sectional or longitudinal analyses.

\textbf{Deprivation} Cross-sectionally, deprivation was associated with deactivation in the left pallidum for positive faces (\textit{b} = -0.060, \textit{SE} = 0.021, \textit{p} = .005). For negative faces, deprivation was linked to deactivation in the left pallidum (\textit{b} = -0.044, \textit{SE} = 0.021, \textit{p} = .035) and right hippocampus (\textit{b} = -0.041, \textit{SE} = 0.020, \textit{p} = .041). Longitudinally, no significant brain activation was observed.

For behavioral measures, deprivation was related to lower accuracy for positive faces (\textit{b} = -0.017, \textit{SE} = 0.005, \textit{p} = .001) and negative faces (\textit{b} = -0.017, \textit{SE} = 0.005, \textit{p} = .001). Longitudinal analyses revealed similar results, with lower accuracy for positive faces (\textit{b} = -0.022, \textit{SE} = 0.006, \textit{p} < .001) and negative faces (\textit{b} = -0.024, \textit{SE} = 0.006, \textit{p} < .001) in the deprivation group.

Furthermore, longer reaction times were observed for positive faces (\textit{b} = 27.191, \textit{SE} = 9.322, \textit{p} = .004) and negative faces (\textit{b} = 28.501, \textit{SE} = 9.553, \textit{p} = .003) in the longitudinal analysis. After FDR correction, only longitudinal associations with positive and negative accuracy remained significant.

\textbf{Summary of findings} 
For positive emotional stimuli, threat was related to reduced brain activation, contrary to the hypothesis. No significant associations were found between threat and behavioral measures such as reaction time or accuracy for positive stimuli. Deprivation was linked to cross-sectional deactivation in the pallidum, consistent with the hypothesis.

For negative emotional stimuli, the hypothesized amygdala hyperactivation associated with threat and deprivation was not observed. Instead, longitudinal analyses indicated that threat was associated with hyperactivation in the ventral striatum (NAc), while no significant behavioral associations (reaction time or accuracy) were found for threat. Deprivation was related with deactivation in the pallidum and hippocampus cross-sectionally. See Table~\ref{tab:emo_summary_glm} for a summary of results for emotion processing.





% Please add the following required packages to your document preamble:
% \usepackage{booktabs}
% \usepackage{multirow}
% \usepackage{graphicx}
% \usepackage[table,xcdraw]{xcolor}
% Beamer presentation requires \usepackage{colortbl} instead of \usepackage[table,xcdraw]{xcolor}
\begin{table}[]
\caption{Summary of GLM results for emotion processing}
\label{tab:emo_summary_glm}
\scriptsize
\renewcommand{\arraystretch}{0.8} % 행 높이를 더 줄임
\setlength{\tabcolsep}{4pt} % 열 간격을 더 줄임
\begin{adjustbox}{max width=\textwidth, center} % 표 중앙 정렬
\begin{tabular}{@{}lcllcccc@{}}
\toprule
\multicolumn{1}{c}{\multirow{2}{*}{Domain}} &
  \multicolumn{2}{c}{\multirow{2}{*}{Task or measurement}} &
  \multicolumn{1}{c}{\multirow{2}{*}{ROI}} &
  \multicolumn{2}{c}{Threat} &
  \multicolumn{2}{c}{Deprivation} \\ \cmidrule(l){5-8} 
\multicolumn{1}{c}{} &
  \multicolumn{2}{c}{} &
  \multicolumn{1}{c}{} &
  Y0 &
  Y2 &
  Y0 &
  Y2 \\ \hline
\multirow{7}{*}{Emotion processing} &
  \multirow[t]{2}{*}{\raisebox{3.5ex}{EN-Back fMRI}} &
  {\raisebox{3.5ex}{Positive vs neutral}} &
  \raisebox{3ex}{\multirow{2}{*}{\tiny \renewcommand{\arraystretch}{0.75}%
  \begin{tabular}[c]{@{}l@{}} Amygdala\\ ACC (rACC/cACC)\\ Insula \\ Striatum (caudate/putamen/NAc) \\ Pallidum\\ Thalamic regions \\(thalamus/ventral diencephalon/\\hippocampus)\end{tabular}}} &
  \cellcolor[HTML]{CFE2F3}\begin{tabular}[c]{@{}c@{}}left putamen ↓\\ bilateral hippocampus ↓\\ right insula ↓\end{tabular} &
  \raisebox{3.5ex}{\cellcolor[HTML]{CFE2F3}left putamen ↓} &
  \raisebox{3.5ex}{\cellcolor[HTML]{CFE2F3}left pallidum ↓} &
  \raisebox{3.5ex}{N.S.} \\
 &
   &
  \raisebox{1.5ex}{Negative vs neutral} &
   &
  \raisebox{1.5ex}{N.S.} &
  \raisebox{1.5ex}{\cellcolor[HTML]{F4CCCC} right NAc ↑} &
  \cellcolor[HTML]{CFE2F3}\begin{tabular}[c]{@{}c@{}}left pallidum ↓\\ right hippocampus ↓\end{tabular} &
  \raisebox{1.5ex}{N.S.} \\
 &
  \raisebox{3.5ex}{\multirow{4}{*}{\begin{tabular}[c]{@{}l@{}}EN-Back fMRI\\ Behavior\end{tabular}}} &
  Accuracy (positive) &
   &
  N.S. &
  N.S. &
  \cellcolor[HTML]{CFE2F3}↓ &
  \cellcolor[HTML]{CFE2F3}\textbf{↓} \\
 &
   &
  Accuracy (negative) &
   &
  N.S. &
  N.S. &
  \cellcolor[HTML]{CFE2F3}↓ &
  \cellcolor[HTML]{CFE2F3}\textbf{↓} \\
 &
   &
  RT (positive) &
   &
  N.S. &
  N.S. &
  N.S. &
  \cellcolor[HTML]{F4CCCC}↑ \\
 &
   &
  RT (negative) &
   &
  N.S. &
  N.S. &
  N.S. &
  \cellcolor[HTML]{F4CCCC}↑ \\ \bottomrule
\end{tabular}
\end{adjustbox}
    \parbox{\linewidth}{
            \vspace{0.5em}
        \renewcommand{\baselinestretch}{1}
        \small
        % \raggedright
        \textit{Note.} Written in bold means significance after FDR correction. Mean age: Y0 (cross-sectional) = 9.96 years, Y2 (longitudinal) = 11.94 years.  \\
        \textit{Abbreviations.} N.S., Not significant; ROI, Regions of Interest; EN-Back, emotional N-back task; RT, Reaction Time; ACC, anterior cingulate cortex; rACC, rostral anterior cingulate cortex; cACC, caudal anterior cingulate cortex; NAc, nucleus accumbens
    }
\end{table}


% \begin{table}[H]
%     \centering
%     \caption{Summary of GLM results for emotion processing}
%     \label{tab:emo_summary_glm}
%     \includegraphics[width=1\linewidth]{emo_summary_glm.png}
%         \renewcommand{\baselinestretch}{1}
%         \small
%     \parbox{\linewidth}{
%         \textit{Note.} Written in bold means significance after FDR correction \\
%         \textit{Abbreviations.} N.S., Not significant; ROI: Regions of Interest; EN-Back, emotional N-back task; RT, Reaction Time; ACC, anterior cinculate cortex; rACC, rostral anterior cingulate cortex; cACC, caudal anterior cingulate cortex; NAc, nucleus accumbens
%         }
% \end{table}

\subsubsection{Working memory}
\textbf{Threat} 

No significant relationships were observed between threat and any working memory measures in either cross-sectional or longitudinal analyses.

\textbf{Deprivation} Deprivation was related to decreased activation in CEN regions only in cross-sectional analyses (left rMFG: \textit{b} = -0.047, \textit{SE} = 0.020, \textit{p} = .017; right IPL: \textit{b} = -0.032, \textit{SE} = 0.016, \textit{p} = .048). However, these associations did not survive after FDR correction.
For behavioral measures, deprivation was related to lower 2-back accuracy (\textit{b} = -0.017, \textit{SE} = 0.005, \textit{p} = .001), list sorting test score (\textit{b} = -1.786, \textit{SE} = 0.870, \textit{p} = .040), and card sort test score (\textit{b} = -2.264, \textit{SE} = 0.949, \textit{p} = .017) cross-sectionally. Longitudinal analyses revealed that only lower 2-back accuracy was related to deprivation (\textit{b} = -0.022, \textit{SE} = 0.006, \textit{p} < .001). After FDR correction, only the 2-back accuracy across years remained significant (cross-sectional: \textit{p-fdr} = .043, longitudinal: \textit{p-fdr} = .009).

\textbf{Summary of findings} As hypothesized, deprivation was consistently linked to poorer working memory performance, particularly in 2-back accuracy across cross-sectional and longitudinal analyses. While cross-sectional analyses also suggested reduced activation in CEN regions, such as the left rMFG and right IPL, these findings did not survive FDR correction. In contrast, no significant associations were found between threat and any working memory-related measures. See Table~\ref{tab:wm_summary_glm} for a summary of results for working memory.


\begin{table}[H]
\caption{Summary of GLM results for working memory}
\label{tab:wm_summary_glm}
\scriptsize
\renewcommand{\arraystretch}{0.8} % 행 높이를 더 줄임
\setlength{\tabcolsep}{4pt} % 열 간격을 더 줄임
\begin{adjustbox}{max width=\textwidth, center} % 표 중앙 정렬
\begin{tabular}{@{}llllcccc@{}}
\toprule
\multicolumn{1}{c}{\multirow{2}{*}{Domain}} &
  \multicolumn{2}{c}{\multirow{2}{*}{Task or measurement}} &
  \multicolumn{1}{c}{\multirow{2}{*}{ROI}} &
  \multicolumn{2}{c}{Threat} &
  \multicolumn{2}{c}{Deprivation} \\ \cmidrule(l){5-8} 
\multicolumn{1}{c}{} & \multicolumn{2}{c}{}                                       & \multicolumn{1}{c}{} & Y0   & Y2   & Y0         & Y2         \\ \hline
\multirow{5}{*}{\begin{tabular}[c]{@{}l@{}}Working \\ memory\end{tabular}} &
  \raisebox{9.2ex}{EN-Back fMRI} &
  \raisebox{9.2ex}{2-back vs. 0-back} &
  \begin{tabular}[c]{@{}l@{}}DMN regions \\ (PCC, precuneus, \\ IPL, mPFC: rACC/mOFC)\\ CEN regions \\(dlPFC: rMFG/cMFG, \\ PPC: SPL/IPL)\end{tabular} &
  \raisebox{9.2ex}{N.S.} &
  \raisebox{9.2ex}{N.S.} &
  \raisebox{7.5ex}{\cellcolor[HTML]{CFE2F3}\begin{tabular}[c]{@{}c@{}}left rMFG ↓\\ right IPL ↓\end{tabular}} &
  \raisebox{9.2ex}{N.S.} \\
                     & \raisebox{2ex}{\multirow{2}{*}{EN-Back fMRI Behavior}} & Accuracy (2-back) &                      & N.S. & N.S. & \cellcolor[HTML]{CFE2F3}↓          & \cellcolor[HTML]{CFE2F3}\textbf{↓} \\
                     &                                        & RT (2-back)       &                      & N.S. & N.S. & N.S.       & N.S.       \\
                     & List sorting test                      & Total score       &                      & N.S. & -    & \cellcolor[HTML]{CFE2F3}\textbf{↓} & -          \\
                     & Card sort test                         & Total score       &                      & N.S. & -    & \cellcolor[HTML]{CFE2F3}\textbf{↓}          & -          \\ \bottomrule
\end{tabular}
\end{adjustbox}
    \parbox{\linewidth}{
            \vspace{0.5em}
        \renewcommand{\baselinestretch}{1}
        \small
        % \raggedright
        \textit{Note.} Written in bold means significance after FDR correction. Mean age: Y0 (cross-sectional) = 9.96 years, Y2 (longitudinal) = 11.94 years.  \\
        \textit{Abbreviations.} N.S., Not significant; ROI, Regions of Interest; EN-Back, emotional N-back task; RT, Reaction Time; DMN, default mode network; PCC, posterior cingulate cortex; IPL, inferior parietal lobule,; mPFC, medial prefrontal cortex; rACC, rostral anterior cinculate cortex; mOFC, medial orbitofrontal cortex; CEN, central executive network; dlPFC, dorsolateral prefrontal cortex; rMFG, rostral middle frontal gyrus; cMFG, caudal middle frontal gyrus; PPC, posterior parietal cortex; SPL, superior parietal lobule
        }
\end{table}


% \begin{table}[H]
%     \centering
%     \caption{Summary of GLM results for working memory}
%     \label{tab:wm_summary_glm}
%     \includegraphics[width=1\linewidth]{wm_summary_glm.png}
%     \renewcommand{\baselinestretch}{1}
%     \small
%     \parbox{\linewidth}{
%         \textit{Note.} N.S. Not significant; Written in bold means significance after FDR correction \\
%         \textit{Abbreviations.} ROI: Regions of Interest; EN-Back, emotional N-back task; RT, Reaction Time; DMN, default mode network; PCC, posterior cingulate cortex; IPL, inferior parietal lobule,; mPFC, medial prefrontal cortex; rACC, rostral anterior cinculate cortex; mOFC, medial orbitofrontal cortex; CEN, central excutive network; dlPFC, dorsolateral prefrontal cortex; rMFG, rostral middle frontal gyrus; cMFG, caudal middle frontal gyrus; PPC, posterior pariatal cortex; SPL, superior parietal lobule
%         }
% \end{table}

\subsubsection{Impulsivity}
\textbf{Threat} No significant brain activation was associated with threat in the correct stop versus correct go contrast in cross-sectional analyses. However, stop accuracy was negatively associated with threat in cross-sectional analyses (\textit{b} = -0.006, \textit{SE} = 0.003, \textit{p} = .038). Longitudinal analyses showed that threat was associated with hyperactivation in the left NAc (\textit{b} = 0.028, \textit{SE} = 0.013, \textit{p} = .037). However, these results did not survive after FDR correction. No significant findings were observed in the cash choice task or UPPS-P for threat.

\textbf{Deprivation} Deprivation showed no association with brain activation in either cross-sectional or longitudinal analyses. Additionally, no significant associations were found in the cash choice task. However, higher UPPS-P positive urgency (\textit{b} = 0.792, \textit{SE} = 0.175, \textit{p} < .001, \textit{p-fdr} = .001), lack of perseverance (\textit{b} = 0.459, \textit{SE} = 0.133, \textit{p} = .001, \textit{p-fdr} = .043), and lack of planning (\textit{b} = 0.607, \textit{SE} = 0.003, \textit{p} = .038, \textit{p-fdr} = .001) were associated with deprivation in cross-sectional analyses. These associations were not consistently found in longitudinal analyses, although higher lack of perseverance was modestly associated with deprivation (\textit{b} = 0.394, \textit{SE} = 0.170, \textit{p} = .021, \textit{p-fdr} = .499).

\textbf{Summary of findings} 
The hypothesized associations with UPPS-P negative urgency and increased impulsive choice were not observed for either threat or deprivation. Instead, deprivation was linked to higher positive urgency, lack of perseverance, and lack of planning in cross-sectional analyses. However, these associations were not consistent longitudinally and did not extend to brain activation or impulsive choice. Threat was associated with hyperactivation in the left ventral striatum in longitudinal analyses. However, this result did not survive FDR correction, and no significant findings were observed in impulsive choice or trait impulsivity. See Table~\ref{tab:imp_summary_glm} for a summary of results for impulsivity.

% Please add the following required packages to your document preamble:
% \usepackage{booktabs}
% \usepackage{multirow}
% \usepackage{graphicx}
\begin{table}[H]
\caption{Summary of GLM results for impulsivity}
\label{tab:imp_summary_glm}
\scriptsize
\renewcommand{\arraystretch}{0.8} % 행 높이를 더 줄임
\setlength{\tabcolsep}{4pt} % 열 간격을 더 줄임
\begin{adjustbox}{max width=\textwidth, center} % 표 중앙 정렬
\begin{tabular}{@{}llllcccc@{}}
\toprule
\multicolumn{1}{c}{\multirow{2}{*}{Domain}} &
  \multicolumn{2}{c}{\multirow{2}{*}{Task or measurement}} &
  \multicolumn{1}{c}{\multirow{2}{*}{ROI}} &
  \multicolumn{2}{c}{Threat} &
  \multicolumn{2}{c}{Deprivation} \\ \cmidrule(l){5-8} 
\multicolumn{1}{c}{} & \multicolumn{2}{c}{}                                      & \multicolumn{1}{c}{} & Y0   & Y2   & Y0         & Y2   \\ \hline
\multirow{9}{*}{Impulsivity} &
  \raisebox{7.5ex}{SST fMRI} &
  \raisebox{7.5ex}{Correct stop vs correct go} &
  \begin{tabular}[c]{@{}l@{}}IFG (pars triangularis/\\ pars orbitalis)\\ ACC (rACC/cACC)\\ Striatum \\(caudate/putamen/NAc)\end{tabular} &
  \raisebox{7.5ex}{N.S.} &
  \raisebox{7.5ex}{\cellcolor[HTML]{F4CCCC} left NAc ↑} &
  \raisebox{7.5ex}{N.S.} &
  \raisebox{7.5ex}{N.S.} \\
                     & \raisebox{2ex}{\multirow{2}{*}{SST fMRI Behavior}} & SSRT                 &                      & N.S. & N.S. & N.S.       & N.S. \\
                     &                                    & Stop Accuracy        &                      & \cellcolor[HTML]{CFE2F3}↓    & N.S. & N.S.       & N.S. \\
                     & Cash Choice Task                   &                      &                      & N.S. & N.S. & -          & -    \\
                     & UPPS-P                             & Positive urgency     &                      & N.S. & N.S. & \cellcolor[HTML]{F4CCCC} \textbf{↑} & N.S. \\
                     &                                    & Negative urgency     &                      & N.S. & N.S. & N.S.       & N.S. \\
                     &                                    & Lack of perseverance &                      & N.S. & N.S. & \cellcolor[HTML]{F4CCCC} \textbf{↑} & \cellcolor[HTML]{F4CCCC}↑    \\
                     &                                    & Sensation seeking    &                      & N.S. & N.S. & N.S.       & N.S. \\
                     &                                    & Lack of planning     &                      & N.S. & N.S. & \cellcolor[HTML]{F4CCCC} \textbf{↑} & N.S. \\ \bottomrule
\end{tabular}
\end{adjustbox}
    \parbox{\linewidth}{
            \vspace{0.5em}
        \renewcommand{\baselinestretch}{1}
        \small
        % \raggedright
        \textit{Note.} N.S. Not significant; Written in bold means significance after FDR correction. Mean age: Y0 (cross-sectional) = 9.96 years, Y2 (longitudinal) = 11.94 years. \\
        \textit{Abbreviations.} N.S., Not significant; ROI, Regions of Interest; SST, Stop Signal Task; SSRT, Stop Signal Reaction Time; UPPS-P, Urgency-Premeditation-Perseverance-Sensation Seeking-Positive Urgency; ACC, anterior cingulate cortex; rACC, rostral anterior cingulate cortex; cACC, caudal anterior cingulate cortex; NAc, nucleus accumbens}
\end{table}



% \begin{table}[H]
%     \centering
%     \caption{Summary of GLM results for impulsivity}
%     \label{tab:imp_summary_glm}
%     \includegraphics[width=1\linewidth]{imp_summary_glm.png}
%        \renewcommand{\baselinestretch}{1}
%         \small  
%         \parbox{\linewidth}{
%         \textit{Note.} N.S. Not significant; Written in bold means significance after FDR correction \\
%         \textit{Abbreviations.} ROI: Regions of Interest; SST, Stop Signal Task; SSRT, Stop Signal Reaction Time; UPPS-P, Urgency-Premeditation-Perseverance-Sensation Seeking-Positive Urgency; ACC, anterior cinculate cortex; rACC, rostral anterior cinculate cortex; cACC, caudal anterior cingulate cortex; NAc, nucleus accumbens}
        
% \end{table}

\clearpage
\subsection{LASSO results}
\subsubsection{Threat}
\textbf{Cross-sectional results}

Figure~\ref{fig:threat_cross}a presents a receiver-operating characteristic (ROC) curve and mean area under the curve (AUC) for classifying high threat and low threat groups: 0.73 for the training set and 0.71 for the test set.

Figure~\ref{fig:significant_coeffs_threat_cross} shows the significant variables for classifying the high and low threat groups. Among the neurocognitive functions, only emotion processing was significantly associated with threat. Specifically, lower accuracy to negative faces (standardized coefficient \textit{$\beta$} = -0.021, 95\% confidence interval (CI) = [-0.038, -0.004]). For neural measures, lower left putamen activation to positive faces was associated with high threat, which is consistent with the GLM results (\textit{$\beta$} = -0.035, 95\% CI = [-0.045, -0.025]; Figure~\ref{fig:threat_cross}b).

\begin{figure}[]
    \centering
    \caption{Results of the cross-sectional binomial LASSO regression (low threat vs. high threat)}
    \includegraphics[width=1\linewidth]{threat_cross.png}
    \renewcommand{\baselinestretch}{1}
    \small
    \parbox{\linewidth}{
        \textit{Note.} (a) a representative receiver-operation curve (ROC) (left) and the distribution of the area under the curve (AUC) scores (right) for the training and test datasets. (b) Regions of Interest (ROIs) with significant coefficients during the emotional N-back task, consistently significant in LASSO and GLM
        }
    \label{fig:threat_cross}
\end{figure}


\begin{figure}
    \centering
    \caption{Results of the cross-sectional binomial LASSO regression (low threat vs. high threat): significant standardized coefficients}
    \includegraphics[width=1\linewidth]{significant_coeffs_threat_baseline.png}
    \label{fig:significant_coeffs_threat_cross}    
    \renewcommand{\baselinestretch}{1}
    \small
    \parbox{\linewidth}{
    \textit{Note.} Data acquisition sites were removed for clarification. The error bar represents the 95\% confidence interval.\\
    \textit{Abbreviations.} LASSO, Least Absolute Shrinkage and Selection Operator; EN-Back, emotional N-Back task
}

\end{figure}

Participants who identified as Black (\textit{$\beta$} = 0.435, 95\% CI = [0.362, 0.507]), and those with higher deprivation scores (\textit{$\beta$} = 0.266, 95\% CI = [0.264, 0.507]) were strongly related to the high threat group. Child externalizing problems (\textit{$\beta$} = 0.259, 95\% CI = [0.255, 0.263]), and internalizing problems (\textit{$\beta$} = 0.043, 95\% CI = [0.030, 0.056]) also showed positive associations with the high threat group.

Parent-related variables were also significantly associated with threat: maternal smoking/alcohol use during pregnancy (\textit{$\beta$} = 0.146, 95\% CI = [0.079, 0.214]), father’s drug problem (\textit{$\beta$} = 0.119, 95\% CI = [0.075, 0.163]) were positively related to high threat. Additionally, high parent somatic problems (\textit{$\beta$} = 0.106, 95\% CI = [0.104, 0.109]), antisocial personality problems (\textit{$\beta$} = 0.078, 95\% CI = [0.072, 0.084]), depressive problems (\textit{$\beta$} = 0.053, 95\% CI = [0.052, 0.084]), and avoidant personality problems (\textit{$\beta$} = 0.040, 95\% CI = [0.031, 0.049]) showed positive associations with threat. Furthermore, younger maternal age at childbirth (\textit{$\beta$} = -0.077, 95\% CI = [-0.089, -0.066]) and lower parental education level (\textit{$\beta$} = -0.078, 95\% CI = [-0.084, -0.072]) were related to higher threat levels. The mean significant coefficients and 95\% CI are presented in Table~\ref{tab:threat_cross_coeffs}.

\clearpage
\textbf{Longitudinal results}
\begin{figure}
    \centering
    \caption{Results of the longitudinal binomial LASSO regression (low threat vs. high threat): significant standardized coefficients}
    \includegraphics[width=1\linewidth]{lasso_threat_long.png}
    \label{fig:lasso_threat_long}
    \parbox{\linewidth}{
    \renewcommand{\baselinestretch}{1}
    \small
        \textit{Note.} Data acquisition sites were removed for clarification. The error bar represents the 95\% confidence interval.\\
        \textit{Abbreviations.} LASSO, Least Absolute Shrinkage and Selection Operator; MID, Monetary Incentive Delay task; EN-Back, emotional N-Back task; SST, Stop Signal Task; cACC, caudal anterior cingulate cortex; rACC, rostral anterior cingulate cortex; NAc, nucleus accumbens; mOFC, medial orbitofrontal cortex
        }
\end{figure}

Figure~\ref{fig:lasso_threat_long} shows the significant variables used to classify the high threat and low threat groups. Deprivation score was positively related to the high threat group (\textit{$\beta$} = 0.173, 95\% CI = [0.171, 0.174]). Demographic information was significantly associated with the high threat group: younger age when the child reported ELS was associated with high threat (\textit{$\beta$} = -0.039, 95\% CI = [-0.065, -0.013]). Participants who identified as Black (\textit{$\beta$} = 0.732, 95\% CI = [0.679, 0.785]) and "Other" (\textit{$\beta$} = 0.037, 95\% CI = [0.004, 0.071]) were more likely to belong to the high threat group compared to White participants. High child internalizing problems (\textit{$\beta$} = 0.133, 95\% CI = [0.122, 0.145]) and externalizing problems (\textit{$\beta$} = 0.042, 95\% CI = [0.032, 0.051]) were also associated with the high threat group. Younger maternal age at childbirth (\textit{$\beta$} = -0.131, 95\% CI = [-0.146, -0.116]) and lower parental education level (\textit{$\beta$} = -0.095, 95\% CI = [-0.109, -0.081]) were also related to high threat.

High parental psychopathology scores were also linked to the high threat group, including parent antisocial personality problems (\textit{$\beta$} = 0.164, 95\% CI = [0.146, 0.181]), parent depressive problems (\textit{$\beta$} = 0.106, 95\% CI = [0.103, 0.109]), parent avoidant personality problems (\textit{$\beta$} = 0.079, 95\% CI = [0.074, 0.084]), and parent AD/H problems (\textit{$\beta$} = 0.035, 95\% CI = [0.032, 0.038]).

Figure~\ref{fig:threat_long}a presents a ROC curve and mean AUC for classifying high threat and low threat groups: 0.75 for the training set and 0.68 for the test set.

Unlike the cross-sectional results, threat was related to all four neurocognitive functions longitudinally. Regarding reward processing, hyperactivation in the right insula (\textit{$\beta$} = 0.035, 95\% CI = [0.032, 0.038]) and left cACC (\textit{$\beta$} = 0.035, 95\% CI = [0.032, 0.038]) during reward feedback was related to the high threat group, while reduced activation in left ACC (\textit{$\beta$} = -0.035, 95\% CI = [-0.045, -0.025]) and left caudate (\textit{$\beta$} = -0.035, 95\% CI = [-0.045, -0.025]) during reward anticipation were associated with threat. The brain regions consistently related to threat in GLM and LASSO results are presented in Figure~\ref{fig:threat_long}b.

For emotion processing, consistent with cross-sectional results, deactivation in the left putamen to positive faces was related to the high threat group (\textit{$\beta$} = -0.242, 95\% CI = [-0.364, -0.121]). Hyperactivation in the right rACC (\textit{$\beta$} = 0.107, 95\% CI = [0.055, 0.158]), left thalamus (\textit{$\beta$} = 0.084, 95\% CI = [0.032, 0.136]), and left NAc (\textit{$\beta$} = 0.028, 95\% CI = [0.013, 0.044]) to positive faces were associated with the high threat group. For negative faces, increased activation in the right NAc (\textit{$\beta$} = 0.094, 95\% CI = [0.052, 0.137]) and lower accuracy (\textit{$\beta$} = -0.011, 95\% CI = [-0.020, -0.003]) were related to the high threat group. The brain regions consistently related to threat in GLM and LASSO results are presented in Figure~\ref{fig:threat_long}c.

For working memory, deactivation in the mOFC, which is part of the DMN, was related to the high threat group (\textit{$\beta$} = -0.063, 95\% CI = [-0.094, -0.033]). However, this result was not significant in the previous GLM analysis.

Regarding impulsivity, increased left NAc activation (\textit{$\beta$} = 0.069, 95\% CI = [0.041, 0.096]) was associated with the high threat group, consistent with previous GLM analysis (Figure~\ref{fig:threat_long}d). LASSO results exclusively revealed decreased activation in the pars triangularis of the IFG (\textit{$\beta$} = -0.035, 95\% CI = [-0.069, -0.002]) during response inhibition. Additionally, a positive association was found between the UPPS-P positive urgency subscale and high threat (\textit{$\beta$} = 0.036, 95\% CI = [0.020, 0.052]).

\begin{figure}
    \centering
    \caption{Results of the longitudinal binomial LASSO regression (low threat vs. high threat)}
    \includegraphics[width=1\linewidth]{threat_long.png}
    \label{fig:threat_long}
    \parbox{\linewidth}{
    \renewcommand{\baselinestretch}{1}
    \small
    \textit{Note.} (a) a representative receiver-operation curve (ROC) (left) and the distribution of the area under the curve (AUC) scores (right) for the training and test datasets. (b) Regions of Interest (ROIs) with significant coefficients during the Monatery Incentive Delay task, consistently significant in LASSO and GLM (c) Regions of Interest (ROIs) with significant coefficients during the emotional N-back task, consistently significant in LASSO and GLM (d) Regions of Interest (ROIs) with significant coefficients during the Stop Signal Task, consistently significant in LASSO and GLM
        }
\end{figure}

\textbf{Summary of findings}
The cross-sectional analysis revealed that high threat was associated with reduced striatal activation (left putamen) to positive faces and lower accuracy for negative faces. In the longitudinal analysis, significant differences between high-threat and low-threat groups emerged in brain activation related to reward processing, emotion processing, working memory, and impulsivity.


\clearpage
\subsubsection{Deprivation}
\textbf{Cross-sectional results}
Deprivation was related to more variables than threat (Figure~\ref{fig:significant_coeffs_deprivation_cross}). The threat score was positively related to the high deprivation group (\textit{$\beta$} = 0.268, 95\% CI = [0.266, 0.270]). Father drug problem (\textit{$\beta$} = 0.908, 95\% CI = [0.895, 0.921]), mother drug problem (\textit{$\beta$} = 0.085, 95\% CI = [0.074, 0.097]), and pregnant smoking/alcohol use (\textit{$\beta$} = 0.127, 95\% CI = [0.116, 0.138]) were positively associated with high deprivation. Race/ethnicity was associated with high deprivation: Black (\textit{$\beta$} = 0.796, 95\% CI = [0.774, 0.818]), Hispanic (\textit{$\beta$} = 0.178, 95\% CI = [0.149, 0.208]), Other (\textit{$\beta$} = 0.177, 95\% CI = [0.142, 0.211]). Also, child externalizing behavior problems (\textit{$\beta$} = 0.080, 95\% CI = [0.078, 0.082]) and parental psychopathologies were related to high deprivation (anxiety problem: \textit{$\beta$} = 0.098, 95\% CI = [0.096, 0.101]; antisocial personality problem: \textit{$\beta$} = 0.051, 95\% CI = [0.051, 0.051]; depressive problem: \textit{$\beta$} = 0.034, 95\% CI = [0.031, 0.037]; somatic problem: \textit{$\beta$} = 0.022, 95\% CI = [0.021, 0.022]).

Conversely, lower parental education level was related to high deprivation (\textit{$\beta$} = -0.473, 95\% CI = [-0.473, -0.473]). Child age (\textit{$\beta$} = -0.031, 95\% CI = [-0.039, -0.022]) and maternal age at childbirth (\textit{$\beta$} = -0.234, 95\% CI = [-0.238, -0.231]) were negatively related to high deprivation. Also, children who had cyanosis (blue at birth) were more likely to be classified in the low deprivation group (\textit{$\beta$} = -0.200, 95\% CI = [-0.262, -0.231]).

\begin{figure}
    \centering
    \caption{Results of the cross-sectional binomial LASSO regression (low deprivation vs. high deprivation): significant standardized coefficients}
    \includegraphics[width=1\linewidth]{significant_coeffs_deprivation_cross.png}
    \parbox{\linewidth}{
        \renewcommand{\baselinestretch}{1}
        \small
        \textit{Note.} Data acquisition sites were removed for clarification. The error bar represents the 95\% confidence interval.\\
        \textit{Abbreviations.} LASSO, Least Absolute Shrinkage and Selection Operator; BAS, Behavioral Activation System; MID, Monetary Incentive Delay task; EN-Back, emotional N-Back task; SST, Stop Signal Task; UPPS-P, Urgency-Premeditation-Perseverance-Sensation Seeking-Positive Urgency; NAc, nucleus accumbens; cACC, caudal anterior cingulate cortex; rACC, rostral anterior cingulate cortex; IPL, inferior parietal lobule; rMFG, rostral middle frontal gyrus
        }
\label{fig:significant_coeffs_deprivation_cross}
\end{figure}

Figure~\ref{fig:deprivation_cross}a presents a ROC curve and mean AUC for classifying high deprivation and low deprivation groups. The mean AUC scores were 0.87, 0.84 for the training set and test set, respectively. 

For the behavioral measures of reward processing, higher BAS Drive subscale was associated with high deprivation (\textit{$\beta$} = 0.044, 95\% CI = [0.039, 0.048]). Lower accuracy in reward trials in the MID task was related to high deprivation (\textit{$\beta$} = -0.019, 95\% CI = [-0.026, -0.012]). During reward anticipation, increased activation in the right NAc of the ventral striatum (\textit{$\beta$} = 0.015, 95\% CI = [0.010, 0.020]) and cACC (\textit{$\beta$} = 0.011, 95\% CI = [0.002, 0.020]) were associated with high deprivation. During reward feedback, deactivation in striatum (left NAc: \textit{$\beta$} = -0.053, 95\% CI = [-0.057, -0.049]; left caudate: \textit{$\beta$} = -0.031, 95\% CI = [-0.035, -0.027]), and left cACC (\textit{$\beta$} = -0.038, 95\% CI = [-0.040, -0.035]) was related to high deprivation. The brain regions consistently related to threat in GLM and LASSO results are presented in Figure~\ref{fig:deprivation_cross}b.

\begin{figure}
    \centering
    \caption{Results of the cross-sectional binomial LASSO regression (low deprivation vs. high deprivation)}
    \resizebox{0.9\textwidth}{!}{% 
    \includegraphics[width=1\linewidth]{deprivation_cross.png}}
    \label{fig:deprivation_cross}
    \parbox{\linewidth}{
    \renewcommand{\baselinestretch}{1}
    \small
    \textit{Note.} (a) a representative receiver-operation curve (ROC) (left) and the distribution of the area under the curve (AUC) scores (right) for the training and test datasets. (b) Regions of Interest (ROIs) with significant coefficients during the Monatery Incentive Delay task, consistently significant in LASSO and GLM (c) Regions of Interest (ROIs) associated with emotion processing during the emotional N-back task, consistently significant in LASSO and GLM. (d) Regions of Interest (ROIs) associated with working memory during the emotional N-back task, consistently significant in LASSO and GLM}
\end{figure}

Regarding emotion processing, lower accuracy to positive (\textit{$\beta$} = -0.025, 95\% CI = [-0.025, -0.024]) and negative (\textit{$\beta$} = -0.005, 95\% CI = [-0.006, -0.004]) faces were related to high deprivation. Deactivation in the left pallidum to positive faces was related to high deprivation (\textit{$\beta$} = -0.004, 95\% CI = [-0.007, -0.002]). For negative faces, deactivation in the left pallidum (\textit{$\beta$} = -0.072, 95\% CI = [-0.079, -0.065]), and right hippocampus (\textit{$\beta$} = -0.033, 95\% CI = [-0.038, -0.028]) were associated with high deprivation. The brain regions consistently related to deprivation in GLM and LASSO results are presented in Figure~\ref{fig:deprivation_cross}c.

For working memory, poor behavioral working memory performance was related to high deprivation (list sorting test: \textit{$\beta$} = -0.015, 95\% CI = [-0.017, -0.013]; card sort test: \textit{$\beta$} = -0.079, 95\% CI = [-0.085, -0.073]; 2-back accuracy: \textit{$\beta$} = -0.144, 95\% CI = [-0.145, -0.143]). For neural measures, deactivation in the left rMFG (\textit{$\beta$} = -0.022, 95\% CI = [-0.026, -0.018]) and right IPL (\textit{$\beta$} = -0.022, 95\% CI = [-0.026, -0.018]), which are part of CEN, were associated with high deprivation (Figure~\ref{fig:deprivation_cross}d), consistent with previous GLM analysis.

In impulsivity, only trait impulsivity was significantly associated with high deprivation: high positive urgency (\textit{$\beta$} = 0.156, 95\% CI = [0.154, 0.159]), high lack of planning (\textit{$\beta$} = 0.130, 95\% CI = [0.124, 0.136]), high lack of perseverance (\textit{$\beta$} = 0.048, 95\% CI = [0.042, 0.053]).

\textbf{Longitudinal results}

\begin{figure}
    \centering
    \caption{Results of the longitudinal binomial LASSO regression (low deprivation vs. high deprivation): significant standardized coefficients}
    \includegraphics[width=1\linewidth]{significant_coeffs_deprivation_long.png}
        \parbox{\linewidth}{
        \renewcommand{\baselinestretch}{1}
        \small
        \textit{Note.} Data acquisition sites were removed for clarification. The error bar represents the 95\% confidence interval.\\
        \textit{Abbreviations.} LASSO, Least Absolute Shrinkage and Selection Operator; BAS, Behavioral Activation System; MID, Monetary Incentive Delay task; EN-Back, emotional N-Back task; UPPS-P, Urgency-Premeditation-Perseverance-Sensation Seeking-Positive Urgency; RT, Reaction Time; NAc, nucleus accumbens; cACC, caudal anterior cingulate cortex
        }
\label{fig:significant_coeffs_deprivation_long}
\end{figure}

Figure~\ref{fig:significant_coeffs_deprivation_long} shows the results from the longitudinal analysis to find multivariate patterns related to the high/low deprivation group. The threat score was positively related to the deprivation group (\textit{$\beta$} = 0.150, 95\% CI = [0.142, 0.158]). Race/ethnicity was related to the high deprivation group: Black (\textit{$\beta$} = 0.927, 95\% CI = [0.918, 0.937]), Other (\textit{$\beta$} = 0.351, 95\% CI = [0.338, 0.364]). Father drug problem (\textit{$\beta$} = 0.973, 95\% CI = [0.958, 0.988]) and pregnant smoking/alcohol use (\textit{$\beta$} = 0.351, 95\% CI = [0.338, 0.364]) were positively associated with high deprivation. Also, higher child externalizing behavior problems (\textit{$\beta$} = 0.128, 95\% CI = [0.118, 0.137]) and parental psychopathologies were related to high deprivation: antisocial personality problem (\textit{$\beta$} = 0.072, 95\% CI = [0.067, 0.077]), somatic problem (\textit{$\beta$} = 0.071, 95\% CI = [0.068, 0.073]), anxiety problem (\textit{$\beta$} = 0.009, 95\% CI = [0.002, 0.017]), depressive problem (\textit{$\beta$} = 0.008, 95\% CI = [0.005, 0.011]).

Conversely, lower parental education level was related to high deprivation (\textit{$\beta$} = -0.412, 95\% CI = [-0.416, -0.409]). Younger child age (\textit{$\beta$} = -0.061, 95\% CI = [-0.080, -0.041]) and maternal age at childbirth (\textit{$\beta$} = -0.234, 95\% CI = [-0.244, -0.255]) were related to high deprivation.

Reversing the findings from the cross-sectional analysis, increased brain activation during reward feedback was observed. Hyperactivation in the right pallidum (\textit{$\beta$} = 0.050, 95\% CI = [0.038, 0.062]), bilateral NAc (left: \textit{$\beta$} = 0.019, 95\% CI = [0.011, 0.027]; right: \textit{$\beta$} = 0.033, 95\% CI = [0.026, 0.040]), and left cACC (\textit{$\beta$} = 0.022, 95\% CI = [0.011, 0.033]) were associated with the high deprivation group in the reward feedback positive vs. negative contrast. However, brain activation during reward anticipation and behavioral measures show no significant association with deprivation in the longitudinal analysis.

For emotion processing, hyperactivation in the amygdala to negative stimuli was associated with high deprivation (\textit{$\beta$} = 0.080, 95\% CI = [0.073, 0.086]), consistent with the hypothesis. Slower RT to positive (\textit{$\beta$} = 0.080, 95\% CI = [0.073, 0.086]) and negative (\textit{$\beta$} = 0.016, 95\% CI = [0.007, 0.024]) faces, and lower accuracy to positive (\textit{$\beta$} = -0.058, 95\% CI = [-0.061, -0.056]) and negative faces (\textit{$\beta$} = -0.114, 95\% CI = [-0.118, -0.111]) were also significantly related to high deprivation.

Regarding working memory, lower 2-back accuracy was consistently associated with high deprivation (\textit{$\beta$} = -0.098, 95\% CI = [-0.104, -0.092]), while no significant association in brain activation.

For impulsivity, no significant associations regarding brain activation. However, consistent with the cross-sectional analysis, high trait impulsivity was related to high deprivation: UPPS-P lack of perseverance (\textit{$\beta$} = 0.083, 95\% CI = [0.074, 0.092]), positive urgency (\textit{$\beta$} = 0.064, 95\% CI = [0.038, 0.089]).

Figure~\ref{fig:deprivation_long}a presents a ROC curve and mean AUC for classifying high deprivation and low deprivation groups with longitudinal neurocognitive functions. The mean AUC scores for the training set was 0.85, and 0.82 for the test set. Figure~\ref{fig:deprivation_long}b shows the brain activation consistent with previous GLM results.

\begin{figure}
    \centering
    \caption{Results of the longitudinal binomial LASSO regression (low deprivation vs. high deprivation)}
    \includegraphics[width=1\linewidth]{deprivation_long.png}
    \parbox{\linewidth}{
    \renewcommand{\baselinestretch}{1}                   
    \small
    \textit{Note.} (a) a representative receiver-operation curve (ROC) (left) and the distribution of the area under the curve (AUC) scores (right) for the training and test datasets. (b) Regions of Interest (ROIs) with significant coefficients during the Monatery Incentive Delay task, consistently significant in LASSO and GLM}
    \label{fig:deprivation_long}
\end{figure}

\textbf{Summary of findings} Cross-sectionally, reduced brain activation in reward processing, emotion processing, and working memory were significantly related to high deprivation, after controlling for covariates and the level of threat. However, longitudinally, hyperactivation in brain regions related to reward processing and emotion processing was associated with high deprivation.


% if needed 
\clearpage
\subsection{Additional analysis results}
In the main analyses, participants were categorized into low and high threat or deprivation groups based on approximately the 10th percentile (13\% for threat and 9.2\% for deprivation). While this approach facilitated group-based comparisons, the threshold was somewhat arbitrary. To address this limitation, additional analyses were conducted treating threat and deprivation as continuous variables. Generalized linear regression (GLM) was used for this purpose, as it allows for the direct analysis of continuous variables.
The results showed generally consistent patterns with the binary classification analyses, although some differences were observed in the specific brain regions associated with threat and deprivation. For example, continuous threat scores were positively associated with striatal and insula activation during reward feedback, consistent with the binary results. Similarly, continuous deprivation scores revealed significant positive relationships with self-reported impulsivity, although the specific subcategories showed some variability. Detailed summaries of the continuous threat and deprivation results, along with significant coefficients, are provided in Appendix~\ref{appendix:D}.

 % Differences between the binary and continuous approaches are further explored in the Discussion section.


\end{document}