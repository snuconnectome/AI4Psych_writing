\makeatletter
\def\input@path{{../}}
\makeatother
\documentclass[../document-en.tex]{subfiles}

\begin{document}

\section{Introduction}
Early life stress (ELS) encompasses adverse childhood experiences such as abuse, neglect, and household adversity. The impact of ELS on brain development, cognitive and affective functions, and mental health outcomes has been extensively investigated (\cite{carr2013}; \cite{chen2016}; \cite{mclaughlinreview2019}; \cite{pechtel2011}). Investigating its effects on neurocognitive functions is particularly critical for preventive interventions, as these functions represent key risk factors for psychiatric disorders.

Childhood and adolescence are critical periods for neurocognitive development, during which various systems interact and influence each other as they mature. Notably, the development of top-down and bottom-up processes progresses along distinct trajectories, and these patterns may be further influenced by heightened emotional reactivity (\cite{caseydevelop2008}). To better understand these complex dynamics, it is crucial to consider how ELS influences both bottom-up processes (e.g., reward and emotion processing) and top-down processes (e.g., impulsivity and working memory) at the same time, as these systems are interdependent rather than isolated.

The Dimensional Model of Adversity and Psychopathology (DMAP; \cite{mclauglinsheridan2016, sheridanmclauglin2014}) posits that threat and deprivation, the core dimensions of ELS, have distinct neural mechanisms. Threat involves experiences of actual or potential harm, whereas deprivation is characterized by the absence of critical developmental inputs from the environment (\cite{mclaughlinsheridanlambert2014, sheridanmclauglin2014}). Behavioral evidence suggests that threat is linked to negative emotion processing, while deprivation is associated with poor executive function (\cite{Schäfer2023, johnson2021, machlin2019}). However, findings on brain function remain inconsistent. A systematic review noted non-significant or inconclusive results, despite the differing effects of threat, deprivation, and mixed exposure (\cite{mclaughlinreview2019}).

This inconsistency may stem from small sample sizes and cross-sectional study designs with retrospective ELS measurement (\cite{mclaughlinreview2019}). Small sample sizes, in particular, contribute to the replication crisis (\cite{turnerreplication2018}). To address this, recent studies have leveraged the large-scale longitudinal Adolescent Brain Cognitive Development (ABCD) study to examine ELS and brain function (\cite{stinson2024, beck2024, jeong2024, Vedechkina2024, yangimpul2024, russellabcd2025}).

Nevertheless, many of these studies employed data-driven approaches (\cite{beck2024, jeong2024, Vedechkina2024, russellabcd2025}) or relied on cumulative Adverse Childhood Experience (ACE) scores (\cite{stinson2024}), complicating comparisons with earlier findings. One study applied the dimensional model, focusing on unpredictability and its relationship with resting-state functional connectivity and psychopathology (\cite{yangimpul2024}). 

Task-based functional Magnetic Resonance Imaging (fMRI) is particularly well-suited for identifying neural correlates of cognitive processes (\cite{chenfmri2015}) and holds promise for investigating the neural distinctions between threat and deprivation. To my knowledge, among studies utilizing large-scale data, only two have employed task-based fMRI: one examined network efficiency using graph theory with data-driven ELS (\cite{jeong2024}), and the other explored impulsivity in relation to cumulative ELS (\cite{stinson2024}). However, these studies did not differentiate between threat and deprivation, limiting their applicability to dimensional frameworks. 

Therefore, this study aims to investigate the distinct impacts of threat and deprivation on multiple neurocognitive functions during childhood, using task-based fMRI measures within a large-scale longitudinal dataset. 


\subsection{Early life stress and neurocognitive functions}
The subsequent section will explore the impact of ELS on reward processing, emotion processing, working memory, and impulsivity, focusing on inconsistent findings in the field. Given the extensive research in this field, this discussion will primarily draw on evidence from task-based fMRI studies and behavioral outcomes. Notably, much of the existing research has assessed ELS without explicitly distinguishing between its specific dimensions, such as threat and deprivation. Such studies have been categorized as mixed exposure to facilitate comparison.

\subsubsection{Reward processing}
Reward processing encompasses the cognitive and neural mechanisms involved in recognizing, responding to, and learning from stimuli associated with rewards. Key brain regions implicated in the reward system include the anterior cingulate cortex (ACC), orbital prefrontal cortex, ventral striatum, and ventral pallidum (\cite{haberreward2010}). Altered reward processing is commonly observed in psychiatric disorders such as depression (\cite{forbesreward2012, baskinreward2015, pizzagallireward2009}) and Substance Use Disorder (SUD; \cite{baskinreward2015, joynerreward2018, beckreward2009}).

Evidence suggests that early life stress (ELS) disrupts ventral striatum-related functions, including reward responsiveness and approach motivation (\cite{novickreward2018}). A meta-analysis revealed medium associations between ELS and reward learning, with small associations in reward valuation and responsiveness (\cite{olteanreward2022}). 

Studies using the Behavioral Activation System (BAS; \cite{carverbas1994}), a self-reported measure of reward processing, have shown mixed findings. While most reported no significant relationships between ELS and BAS subscales (\cite{rosenmanreward2009, deldonnoreward2019, draganreward2019, johnsonreward2016, marusakreward2015, marusakreward2017, miureward2017}), some studies found reduced reward responsiveness (\cite{babadreward2021, draganreward2019}) and decreased drive, which reflects motivation for pursuing goals (\cite{babadreward2021, miureward2017}).

Behavioral assessments of the Monetary Incentive Delay (MID) task also yield diverse results. For threat, reaction time (RT) to reward-related stimuli varies, showing increases (\cite{dillonreward2009}), decreases (\cite{dennisonreward2016}), or no significant differences (\cite{mehtareward2010}) compared to healthy controls. Similarly, for deprivation, RT has been reported to increase (\cite{muellerreward2012}) or show no significant differences (\cite{mehtareward2010}). Few studies have investigated accuracy, generally reporting no significant differences (\cite{muellerreward2012, mehtareward2010, dennisonreward2019}), except for reduced performance linked to food insecurity (\cite{dennisonreward2019}).

Systematic reviews have demonstrated that reduced striatal activation during reward processing is related to ELS, particularly deprivation (\cite{gerinreward2019, mccroryreview2017}). Specifically, during reward anticipation, ELS is consistently associated with reduced activation in the ventral striatum (\cite{boeckerreward2014, boeckerreward2016, mehtareward2010}), putamen, pallidum (\cite{boeckerreward2016, dillonreward2009}) and anterior cingulate cortex (\cite{boeckerreward2014, boeckerreward2016}). In reward feedback, however, the results are less consistent. Most studies focus on mixed exposures, reporting heightened activation in the insula, pallidum, and putamen (\cite{boeckerreward2014, boeckerreward2016}), while others observe no differences (\cite{birnreward2017, mehtareward2010}). Findings on ventral striatum activation during feedback show inconsistency (\cite{boeckerreward2016, hansonreward2015}). (Please refer to Table~\ref{tab:summary_rew} for further details of the results in reward processing.)


\begin{landscape}
\tiny
\renewcommand{\arraystretch}{0.5} % 행 높이를 더 줄임
% \setlength{\extrarowheight}{0pt}
% \setlength{\tabcolsep}{2pt}
\begin{longtable}{@{}m{1cm}>{\centering\arraybackslash}m{1.3cm}>{\centering\arraybackslash}m{2cm}>{\centering\arraybackslash}m{2cm}>{\centering\arraybackslash}m{1.5cm}>{\centering\arraybackslash}m{1.2cm}m{3.5cm}m{4cm}@{}}

% \begin{longtable}{@{}p{1.5cm}>{\centering\arraybackslash}p{1cm}>{\centering\arraybackslash}p{2cm}>{\centering\arraybackslash}>{\centering\arraybackslash}p{2cm}>{\centering\arraybackslash}p{1cm}>{\centering\arraybackslash}p{1.5cm}p{3cm}p{4.5cm}@{}}
\captionsetup{justification=raggedright}
\caption{Summary of previous results: early life stress and reward processing}
\label{tab:summary_rew}\\
\toprule
\multicolumn{1}{c}{Study} &
\begin{tabular}[c]{@{}c@{}} ELS \\ dimension\end{tabular} &
\multicolumn{1}{c}{Participants (N)} &
\multicolumn{1}{c}{Age: mean (SD)} &
\multicolumn{1}{c}{Design} &
\multicolumn{1}{c}{Measurement} &
\multicolumn{1}{c}{Behavioral results} &
\multicolumn{1}{c}{Neural results}\\* \midrule
\endfirsthead

\captionsetup{list=no}
\caption[]{(Continued)}\\
\toprule
\multicolumn{1}{c}{Study} &
\begin{tabular}[c]{@{}c@{}} ELS \\ dimension\end{tabular} &
\multicolumn{1}{c}{Participants (N)} &
\multicolumn{1}{c}{Age: mean (SD)} &
\multicolumn{1}{c}{Design} &
\multicolumn{1}{c}{Measurement} &
\multicolumn{1}{c}{Behavioral results} &
\multicolumn{1}{c}{Neural results}\\* \midrule
\endhead
%
\raggedright\setlength{\baselineskip}{0.5\baselineskip} \cite{dennisonreward2016} &
 Threat &
\begin{tabular}{c}
ELS (21) \\ HC (38)
\end{tabular}
&
16.95 (1.44) &
\setlength{\baselineskip}{0.5\baselineskip}  Cross-sectional &
\setlength{\baselineskip}{0.5\baselineskip}  MID task &
  RT ↓ (at the trend level) &
\multicolumn{1}{c}{-} \\* \midrule
\raggedright\setlength{\baselineskip}{0.5\baselineskip} \cite{dillonreward2009}&
  Threat &
  \begin{tabular}[c]{@{}c@{}}ELS (13) \\ HC (31)\end{tabular} &
  \begin{tabular}[c]{@{}c@{}}ELS 24.58 (.88)\\ HC 37.08 (13.77)\end{tabular} &
\setlength{\baselineskip}{0.5\baselineskip}   Cross-sectional &
\setlength{\baselineskip}{0.5\baselineskip}   MID task &
  RT ↑  (at the trend level) &
  \multicolumn{1}{m{4cm}}{\setlength{\baselineskip}{0.5\baselineskip} Reward anticipation: left putamen and left pallidus ↓} \\* \midrule
\raggedright\setlength{\baselineskip}{0.5\baselineskip}   \cite{dennisonreward2019} &
  \begin{tabular}[c]{@{}c@{}}Threat/\\ deprivation\end{tabular} &
  94 participants &
  13.57 (3.47) &
\setlength{\baselineskip}{0.5\baselineskip}  Cross-sectional &
\setlength{\baselineskip}{0.5\baselineskip}  Reward Processing task &
\multicolumn{1}{m{3.5cm}}{\raggedright
\setlength{\baselineskip}{0.5\baselineskip} 
\hspace{-0.04cm}- Abuse: reward performance not significant\\ 
- Food insecurity: reward performance (lower rewards) ↓ \\ 
- Neglect: reward performance not significant
} &
\multicolumn{1}{c}{-} \\* \midrule
\raggedright\setlength{\baselineskip}{0.5\baselineskip} \cite{liuzzireward2024}  &
  \begin{tabular}[c]{@{}c@{}}Threat/\\ deprivation\end{tabular} &
\setlength{\baselineskip}{0.5\baselineskip}  169 trauma survivors &
  32.2 (10.3) &
\setlength{\baselineskip}{0.5\baselineskip}   Cross-sectional &
\multicolumn{1}{m{1.4cm}}{\centering\setlength{\baselineskip}{0.5\baselineskip}   Resting state-functional connectivity} &
  \multicolumn{1}{c}{-} &
  \multicolumn{1}{m{4cm}}{\raggedright
\setlength{\baselineskip}{0.5\baselineskip} 
  \hspace{-0.04cm}- Threat: right nucleus accumbens connectivity with temporal fusiform gyrus/parahippocampal gyrus ↑, not significant amygdala \\ - Deprivation: nucleus accumbens, amygdala not significant} \\* \midrule
\raggedright
\setlength{\baselineskip}{0.5\baselineskip} \cite{hansonreward2015} &
  Deprivation &
  106 participants &
  13.67 &
  Longitudinal &
\setlength{\baselineskip}{0.5\baselineskip} Card guessing paradigm &
  \multicolumn{1}{c}{-} &
  \multicolumn{1}{m{4cm}}{\setlength{\baselineskip}{0.5\baselineskip} Reward feedback: ventral striatum development ↓} \\* \midrule
\raggedright\setlength{\baselineskip}{0.5\baselineskip} \cite{mehtareward2010} &
  Deprivation &
  \begin{tabular}[c]{@{}c@{}}ELS (12) \\ HC (11)\end{tabular} &
  \begin{tabular}[c]{@{}c@{}}ELS 16.1 (.77)\\ HC 16.0 (.85)\end{tabular} &
\setlength{\baselineskip}{0.5\baselineskip}   Cross-sectional &
  MID Task &
  Accuracy, RT: not significant &
  \multicolumn{1}{m{4cm}}{\raggedright\setlength{\baselineskip}{0.5\baselineskip} 
 Reward anticipation: ventral striatum and caudate nucleus ↓} \\* \midrule
\raggedright\setlength{\baselineskip}{0.5\baselineskip} \cite{muellerreward2012} &
  Deprivation &
  \begin{tabular}[c]{@{}c@{}}ELS (17) \\ HC (29)\end{tabular} &
  \begin{tabular}[c]{@{}c@{}}ELS 11.3 (1.9)\\ HC 11.9 (2.4)\end{tabular} &
\setlength{\baselineskip}{0.5\baselineskip}   Cross-sectional &
\setlength{\baselineskip}{0.5\baselineskip}   Monetary Incentive Saccade task &
  \multicolumn{1}{m{3.5cm}}{\raggedright
\setlength{\baselineskip}{0.5\baselineskip} \hspace{-0.04cm}- RT ↑ \\ - Error: not significant (reward \& neutral condition)} &
  \multicolumn{1}{c}{-}\\* \midrule
  \raggedright\setlength{\baselineskip}{0.5\baselineskip} 
\cite{birnreward2017} &
  Mixed &
  \begin{tabular}[c]{@{}c@{}}ELS (23) \\ HC (19)\end{tabular} &
  20.5 &
  Longitudinal &
  MID task &
  \multicolumn{1}{c}{-} &
  \multicolumn{1}{m{4cm}}{\setlength{\baselineskip}{0.5\baselineskip} \hspace{-0.04cm}- Reward anticipation: posterior cingulate/precuneus, middle temporal gyrus, lingual gyrus, right middle frontal gyrus, and cerebellum ↓} \\* \midrule
    \raggedright\setlength{\baselineskip}{0.5\baselineskip} 
 \cite{boeckerreward2014} &
  Mixed &
  162 participants &
  24.4 &
  Longitudinal &
  MID task &
  RT ↑ &
  \multicolumn{1}{m{4cm}}{  \raggedright\setlength{\baselineskip}{0.5\baselineskip} \hspace{-0.04cm}- Reward anticipation (monetary vs. verbal): left ventral striatum, putamen, left thalamus, left insula, left anterior cingulate cortex and right anterior hippocampus ↓ in ROI analysis\\ - Reward feedback (win vs. no win): bilateral insula, right pallidum, and bilater putamen ↑ in ROI analysis} \\* \midrule
  \raggedright\setlength{\baselineskip}{0.5\baselineskip}
 \cite{boeckerreward2016} &
  Mixed &
  168 participants &
  24.5 (.59) &
  Longitudinal &
  MID task &
  RT ↑ &
  \multicolumn{1}{m{4cm}}{\raggedright\setlength{\baselineskip}{0.5\baselineskip}\hspace{-0.04cm}- Reward anticipation (monetary vs. verbal): left middle frontal gyrus, left and right anterior cingulate cortex, left ventral striatum, left and right putamen, left anterior insula and left pallidum ↓ in ROI analysis \\ - Reward feedback (win vs. no win): not significant} \\* \midrule
\raggedright\setlength{\baselineskip}{0.3\baselineskip} \cite{holzemotion2017} &
  Mixed &
  171 participants &
  25 &
  Longitudinal &
\setlength{\baselineskip}{0.5\baselineskip}   Reward task &
  \multicolumn{1}{c}{-} &
  \multicolumn{1}{m{4cm}}{\raggedright\setlength{\baselineskip}{0.5\baselineskip} Reward anticipation: left caudate, putamen, dorsal striatum ↓ in ROI analysis} \\* \midrule
  \raggedright\setlength{\baselineskip}{0.5\baselineskip} 
\cite{babadreward2021} &
  Mixed &
  436 participants &
  19.73 (1.83) &
\setlength{\baselineskip}{0.5\baselineskip}  Cross-sectional &
  BAS &
  \multicolumn{1}{m{3.5cm}}{\raggedright\setlength{\baselineskip}{0.5\baselineskip}\hspace{-0.04cm}- Reward responsiveness: ↓ emotional neglect, physical neglect, domestic violence, age, and being male\\ - Drive:  ↓ (emotional neglect, substance use, mental illness, and cumulative ACEs) \\ - Fun seeking: not reported} &
  \multicolumn{1}{c}{-} \\* \midrule  
\raggedright\setlength{\baselineskip}{0.5\baselineskip}  \cite{deldonnoreward2019}&
  Mixed &
  \begin{tabular}[c]{@{}c@{}}MDD (23)\\ HC (27)\end{tabular} &
  \begin{tabular}[c]{@{}c@{}}MDD 25.09 (3.32)\\ HC 29.15 (9.00)\end{tabular} &
\setlength{\baselineskip}{0.5\baselineskip}  Cross-sectional &
  BAS &
  \multicolumn{1}{m{3.5cm}}{\raggedright\setlength{\baselineskip}{0.5\baselineskip}\hspace{-0.04cm}- Reward responsiveness: not significant \\ - Drive: not significant\\ - Fun seeking: not reported} &
  \multicolumn{1}{c}{-}\\* \midrule
\raggedright\setlength{\baselineskip}{0.5\baselineskip} 
 \cite{draganreward2019} &
\setlength{\baselineskip}{0.5\baselineskip}   Mixed (cumulative ELS) &
  48 participants &
  25.1 (5.36) &
\setlength{\baselineskip}{0.5\baselineskip}   Cross-sectional &
  BAS &
  \multicolumn{1}{m{3.5cm}}{\raggedright\setlength{\baselineskip}{0.3\baselineskip}\hspace{-0.04cm}- Reward responsiveness: ↓ (significant only in female)\\ - Drive: not significant\\ - Fun seeking: not significant} &
  \multicolumn{1}{c}{-} \\* \midrule
\raggedright\setlength{\baselineskip}{0.5\baselineskip} 
\cite{marusakreward2015} &
  Mixed &
  \begin{tabular}[c]{@{}c@{}}ELS (14)\\ HC (19)\end{tabular} &
  \begin{tabular}[c]{@{}c@{}}ELS 12.61 (2.11)\\ HC 12.06 (2.66)\end{tabular} &
\setlength{\baselineskip}{0.5\baselineskip}   Cross-sectional &
  BAS &
  \multicolumn{1}{m{3.5cm}}{\raggedright\setlength{\baselineskip}{0.5\baselineskip}\hspace{-0.04cm}- Reward responsiveness: not significant \\ - Drive: not significant\\ - Fun seeking: not significant} &
  \multicolumn{1}{c}{-}  \\* \midrule
  \raggedright\setlength{\baselineskip}{0.5\baselineskip} 
 \cite{marusakreward2017} &
 \setlength{\baselineskip}{0.5\baselineskip}  Threat (26\% reported neglect) &
  \begin{tabular}[c]{@{}c@{}}ELS (43)\\ HC (43)\end{tabular} &
  \begin{tabular}[c]{@{}c@{}}ELS 12.28 (2.65)\\ HC 11.51 (2.29)\end{tabular} &
\setlength{\baselineskip}{0.5\baselineskip}   Cross-sectional &
  BAS &
  \multicolumn{1}{m{3.5cm}}{\raggedright\setlength{\baselineskip}{0.5\baselineskip}\hspace{-0.04cm}- Reward responsiveness: not significant\\ - Drive: not reported\\ - Fun seeking: not reported} &
  \multicolumn{1}{c}{-} \\* \midrule
    \raggedright\setlength{\baselineskip}{0.5\baselineskip} 
\cite{miureward2017} &
  Mixed &
  375 participants &
  21.33 (4.40) &
\setlength{\baselineskip}{0.5\baselineskip}   Cross-sectional &
  BAS &
  \multicolumn{1}{m{3.5cm}}{\raggedright\setlength{\baselineskip}{0.5\baselineskip}\hspace{-0.04cm}- Reward responsiveness: not significant\\ - Drive: ↓  interpersonal trauma (than HC/non-interpersonal trauma)\\ - Fun seeking: not significant} &
    \multicolumn{1}{c}{-}\\* \midrule 

    \raggedright\setlength{\baselineskip}{0.5\baselineskip} 
 \cite{rosenmanreward2009} &
  Mixed &
  7485 participants &
\setlength{\baselineskip}{0.5\baselineskip}   20-24, 40-44, 60-64 &
\setlength{\baselineskip}{0.5\baselineskip}   Cross-sectional &
  BAS &
\multicolumn{1}{m{3.5cm}}{\raggedright\setlength{\baselineskip}{0.5\baselineskip}  Not significant in any subscale (reward responsiveness, drive, fun seeking) } &
   \multicolumn{1}{c}{-}
    \\* \midrule    
\raggedright\setlength{\baselineskip}{0.5\baselineskip} 
\cite{johnsonreward2016} &
  Mixed &
  343 participants &
  18.79 (1.93) &
åå\setlength{\baselineskip}{0.5\baselineskip}   Cross-sectional &
  BAS &
   \multicolumn{1}{m{3.5cm}}{\raggedright\setlength{\baselineskip}{0.5\baselineskip}\hspace{-0.04cm}- Reward responsiveness: not significant (only interaction effect of specific gene and ELS)\\ - Drive: not significant\\ - Fun seeking: not significant} &
    \multicolumn{1}{c}{-}
      \\* \bottomrule
\end{longtable}
\renewcommand{\baselinestretch}{1}
\small
\raggedright
\parbox{\linewidth}{ 
\textit{Note.} The criteria for distinguishing between cross-sectional and longitudinal designs were based on the timing of the administration of ELS and neurocognitive function measures. \\
\textit{Abbreviations.} ELS, Early Life Stress; HC, Healthy Control; ROI, Regions of Interest; MID, Monetary Incentive Delay; RT, Reaction Time; BAS, Behavioral Activation System
}
\end{landscape}
  






% \end{longtable} 
% \renewcommand{\baselinestretch}{1}
% \small
% \raggedright
% \parbox{\linewidth}{ 
% \textit{Note.} The criteria for distinguishing between cross-sectional and longitudinal designs were based on the timing of the administration of ELS and neurocognitive function measures. \\
% \textit{Abbreviations.} ELS, Early Life Stress; HC, Healthy Control; ROI, Regions of Interest; MID, Monetary Incentive Delay; RT, Reaction Time
% }
% \end{landscape}


% \begin{landscape}
% \begin{longtable}{c} % longtable 사용, 단일 열 테이블 설정
%     \caption{Summary of findings: early life stress and reward processing} \\ % 캡션 번호 유지
%     \label{tab:summary_rew} \\
%     % 첫 번째 이미지
%     \includegraphics[width=\linewidth]{summary_rew_1.png} \\
%     \caption[]{(Continued)} \\ 
%     % 두 번째 이미지
%     \includegraphics[width=\linewidth]{summary_rew_2.png} \\
% \end{longtable}

% \renewcommand{\baselinestretch}{1}
% \small
% \raggedright
% \parbox{\linewidth}{ 
% \textit{Note.} The criteria for distinguishing between cross-sectional and longitudinal designs were based on the timing of the administration of ELS and neurocognitive function measures. \\
% \textit{Abbreviations.} ELS, Early Life Stress; HC, Healthy Control; ROI, Regions of Interest; MID, Monetary Incentive Delay; RT, Reaction Time
% }
% \end{landscape}

\subsubsection{Emotion processing}
Emotion processing involves the identification of emotional stimuli and the regulation of emotional responses. The amygdala, insula, anterior cingulate cortex (ACC), occipital cortex, and medial prefrontal cortex (mPFC) are critical brain regions involved in this process (\cite{phanemotion2022}). Studies on emotion regulation and ELS reveal atypical neural activation in regulatory areas such as the ventral ACC and inferior prefrontal cortex, with altered frontal-subcortical connectivity (\cite{mccroryreview2017}). Threat is associated with fear learning (\cite{machlin2019}), biases toward angry faces (\cite{Schäfer2023}), but not deprivation.

Task-based fMRI studies commonly reveal amygdala hyperactivation to negative emotional stimuli such as fear and anger in children exposed to threats (\cite{mccroryemotion2011, mclaughlinemotion2015, suzukiemotion2014}). However, one study found reduced amygdala activation to angry faces and no significant difference to fearful faces (\cite{heinemotion2020}). In cases of mixed ELS or deprivation, a systematic review revealed that the amygdala’s responses to negative emotional stimuli yield inconsistent results (\cite{mclaughlinreview2019}). For deprivation, three studies reported amygdala hyperactivation to fearful faces (\cite{geeemotion2013, maheuemotion2010, tottenhanemotion2011}). However, studies have reported lower amygdala discrimination between parent and stranger faces (\cite{callaghanemotion2019, olsavskyemotion2013}) and between trustworthy and untrustworthy faces (\cite{greenemotion2016}). For mixed exposure, most studies reported hyperactivation (\cite{dannlowskiemotion2012, dannlowskiemotion2013, bellisemotion2012, mccroryemotion2013, harmelenemotion2013}), but some studies reported hypoactivation (\cite{holzemotion2017, tayloremotion2006}) or no significant findings (\cite{fonzoemotion2016}).

Furthermore, behavioral measures on negative emotion processing show varied results. Reaction time (RT) to negative emotional stimuli is sometimes faster (\cite{maheuemotion2010, tottenhamemotion2010}), but often, no significant difference is observed compared to controls (\cite{mccroryemotion2011, metzemotion2018, tottenhanemotion2011, tayloremotion2006, harmelenemotion2013}). Accuracy in identifying negative emotional stimuli in children with ELS varies, being reported as higher (\cite{metzemotion2018}), lower (\cite{mccroryemotion2011, tottenhamemotion2010}), or equivalent to those without ELS (\cite{dannlowskiemotion2012, fonzoemotion2016, mccroryemotion2013, tayloremotion2006}).

In contrast to negative emotional stimuli, positive emotional stimuli remain underexplored, despite their importance as social rewards. Behavioral measures, such as accuracy and RT to positive stimuli, have mostly shown no significant differences between individuals with ELS and healthy controls (\cite{goffemotion2013, maheuemotion2010, olsavskyemotion2013}).

In terms of neural activation, limited studies have found that threat is associated with hyperactivation in striatal regions, while deprivation is linked to deactivation in the ventral striatum. However, some studies have also reported no significant differences. Specifically, for threat, hyperactivation has been observed in striatal regions, including the nucleus accumbens and putamen (\cite{dennisonreward2016}), as well as in the amygdala and pallidum (\cite{suzukiemotion2014}). Conversely, other studies found no significant differences when participants viewed happy emotional stimuli (\cite{mclaughlinemotion2015, schweizeremotion2016, suzukiemotion2014}).

For deprivation, studies reported no significant differences in response to happy emotional stimuli compared to healthy controls (\cite{geeemotion2013, maheuemotion2010}). However, some findings have suggested deactivation in the ventral striatum in response to positive emotional stimuli (\cite{goffemotion2013, heinemotion2020}). In cases of mixed ELS, one study reported hyperactivation in the amygdala, thalamus, and pallidum in response to positive emotional stimuli (\cite{mccroryemotion2013}). (Please refer to Table~\ref{tab:summary_emo} for further details of the results in emotion processing.)


\begin{landscape}
\tiny
\renewcommand{\arraystretch}{0.5} % 행 높이를 더 줄임
% \setlength{\extrarowheight}{0pt}
% \setlength{\tabcolsep}{2pt}
\begin{longtable}{@{}m{1cm}>{\centering\arraybackslash}m{1.3cm}>{\centering\arraybackslash}m{2cm}>{\centering\arraybackslash}m{2cm}>{\centering\arraybackslash}m{1.5cm}>{\centering\arraybackslash}m{1.2cm}m{3.5cm}m{4cm}@{}}

\captionsetup{justification=raggedright}
\caption{Summary of previous results: early life stress and emotion processing}
\label{tab:summary_emo}\\
\toprule
\multicolumn{1}{c}{Study} &
\begin{tabular}[c]{@{}c@{}} ELS \\ dimension\end{tabular} &
\multicolumn{1}{c}{Participants (N)} &
\multicolumn{1}{c}{Age: mean (SD)} &
\multicolumn{1}{c}{Design} &
\multicolumn{1}{c}{Measurement} &
\multicolumn{1}{c}{Behavioral results} &
\multicolumn{1}{c}{Neural results}\\* \midrule
\endfirsthead

\captionsetup{list=no}
\caption{(Continued)}\\
\toprule
\multicolumn{1}{c}{Study} &
\begin{tabular}[c]{@{}c@{}} ELS \\ dimension\end{tabular} &
\multicolumn{1}{c}{Participants (N)} &
\multicolumn{1}{c}{Age: mean (SD)} &
\multicolumn{1}{c}{Design} &
\multicolumn{1}{c}{Measurement} &
\multicolumn{1}{c}{Behavioral results} &
\multicolumn{1}{c}{Neural results}\\* \midrule
\endhead
%
\raggedright\setlength{\baselineskip}{0.5\baselineskip} 
\cite{dennisonreward2016} &
  Threat &
  \begin{tabular}[c]{@{}c@{}}ELS (21) \\ HC (38)\end{tabular} &
  16.95 (1.44) &
\setlength{\baselineskip}{0.5\baselineskip}  Cross-sectional &
\setlength{\baselineskip}{0.5\baselineskip}  Emotional images viewing &
\multicolumn{1}{m{3.5cm}}{\raggedright\setlength{\baselineskip}{0.5\baselineskip} Rated images more emotionally intense independent of image type} &
\multicolumn{1}{m{4cm}}{\raggedright\setlength{\baselineskip}{0.5\baselineskip}   Positive vs. neutral stimuli: left nucleus accumbens, left putamen ↑}\\* \midrule
\raggedright\setlength{\baselineskip}{0.5\baselineskip} \cite{mccroryemotion2011} &
  Threat &
  \begin{tabular}[c]{@{}c@{}}ELS (20) \\ HC (23)\end{tabular} &
\setlength{\baselineskip}{0.5\baselineskip}  ELS 12.05 (1.41) HC 12.57 (1.18) &
\setlength{\baselineskip}{0.5\baselineskip}   Cross-sectional &
\setlength{\baselineskip}{0.5\baselineskip}   Face-Viewing  paradigm &
  \begin{tabular}[c]{@{}l@{}}- Accuracy: angry, sad, neutral ↓\\ - RT: not significant\end{tabular} &
\multicolumn{1}{m{4cm}}{\raggedright\setlength{\baselineskip}{0.5\baselineskip} \hspace{-0.04cm}- Angry vs. neutral face: right amygdala, anterior insula ↑\\ - Sad vs. neutral face: not significant}\\* \midrule
\raggedright\setlength{\baselineskip}{0.5\baselineskip} \cite{mclaughlinemotion2015} &
  Threat &
  \begin{tabular}[c]{@{}c@{}}ELS (21) \\ HC (21)\end{tabular} &
  \begin{tabular}[c]{@{}c@{}}ELS 16.26 (1.43) \\ HC 16.89 (1.35)\end{tabular} &
\setlength{\baselineskip}{0.5\baselineskip}   Cross-sectional &
\setlength{\baselineskip}{0.5\baselineskip}   Emotion Regulation task &
\setlength{\baselineskip}{0.5\baselineskip}   No differences in self-reported emotional intensity &
\multicolumn{1}{m{4cm}}{\raggedright\setlength{\baselineskip}{0.5\baselineskip} \hspace{-0.04cm}- Negative vs. neutral (look): amygdala, putamen, thalamus, anterior insula ↑\\ - Positive vs. neutral (look): not significant} \\* \midrule
\raggedright\setlength{\baselineskip}{0.5\baselineskip} \cite{schweizeremotion2016} &
  Threat &
  \begin{tabular}[c]{@{}c@{}}ELS (23) \\ HC (30)\end{tabular} &
  \begin{tabular}[c]{@{}c@{}}ELS 20.13 (.64) \\ HC 20.07 (.74)\end{tabular} &
  Longitudinal &
\setlength{\baselineskip}{0.5\baselineskip}   Emotional Regulation task &
\setlength{\baselineskip}{0.5\baselineskip}  Positive, Negative: emotional regulation ↑ &
\multicolumn{1}{m{4cm}}{\raggedright\setlength{\baselineskip}{0.5\baselineskip} \hspace{-0.04cm}- Negative emotion downregulation (regulation vs. look): middle frontal gyrus, middle temporal gyrus, amygdala ↓\\ - Positive (regulation vs. look): not significant} \\* \midrule
\raggedright\setlength{\baselineskip}{0.5\baselineskip} \cite{suzukiemotion2014} &
  Threat &
\multicolumn{1}{m{2cm}}{\centering\setlength{\baselineskip}{0.5\baselineskip} MDD (42) \\ HC (51) \\ Other psychiatric disorders (22)} &
\multicolumn{1}{m{2cm}}{\centering\setlength{\baselineskip}{0.5\baselineskip}MDD 9.81 (1.23) \\ Healthy 9.8 (1.36) \\ Other psychiatric disorders 10.18 (1.47)} &
  Longitudinal &
\setlength{\baselineskip}{0.5\baselineskip}   Facial emotion-processing task &
  \multicolumn{1}{c}{-} &
\multicolumn{1}{m{4cm}}{\raggedright\setlength{\baselineskip}{0.5\baselineskip} Traumatic events\newline - Sad face: amygdala, left posterior entorhinal cortex, subgenual anterior cingulate cortex and pregenual anterior cingulate cortex↑\newline - Fearful face \& happy face: not significant\newline\newline Stressful events\newline - Fearful face: right amygdala/posterior entorhinal cortex and lateral globus pallidus ↑\newline - sad face: left amygdala ↑\newline - happy face: right amygdala, left medial globus pallidus ↑} \\* \midrule
\raggedright\setlength{\baselineskip}{0.5\baselineskip} \cite{heinemotion2020} &
\begin{tabular}[c]{@{}c@{}}Threat/\\ deprivation\end{tabular} &
  167 participants &
  \begin{tabular}[c]{@{}c@{}}ELS assessment \\ 1, 3, 5, 9, 15\\ fMRI: 15\end{tabular} &
  Longitudinal &
\setlength{\baselineskip}{0.5\baselineskip}   Emotional Faces task &
  \multicolumn{1}{c}{-} &
\multicolumn{1}{m{4cm}}{\raggedright\setlength{\baselineskip}{0.5\baselineskip} \hspace{-0.04cm}- Angry vs. baseline: left amygdala activation ↓, right amygdala habituation ↓ (threat)\\ - Fearful vs. baseline: not significant (threat) \\ - Happy vs. baseline: ventral striatum ↓ (deprivation)} \\* \midrule
\raggedright\setlength{\baselineskip}{0.5\baselineskip} \cite{lambertemotion2017} &
\begin{tabular}[c]{@{}c@{}}Threat/\\ deprivation\end{tabular} &
  287 participants &
  16-17 &
\setlength{\baselineskip}{0.5\baselineskip}   Cross-sectional &
\setlength{\baselineskip}{0.5\baselineskip}   Emotional Stroop task &
\multicolumn{1}{m{3.5cm}}{\raggedright\setlength{\baselineskip}{0.5\baselineskip}  \hspace{-0.04cm}- Threat: adaptation to emotional conflict ↓, inhibition not significant\\ - Deprivation: adaptation to emotional conflict not significant, inhibition ↓} &
  \multicolumn{1}{c}{-} \\* \midrule
\raggedright\setlength{\baselineskip}{0.5\baselineskip} \cite{callaghanemotion2019} &
  Deprivation &
  \begin{tabular}[c]{@{}c@{}}ELS (45)\\ HC (57)\end{tabular} &
  10.25 &
  Longitudinal &
\multicolumn{1}{m{1.2cm}}{\centering\setlength{\baselineskip}{0.5\baselineskip} Parent/\newline stranger fMRI task} &
  \multicolumn{1}{c}{-} &
\multicolumn{1}{m{4cm}}{\raggedright\setlength{\baselineskip}{0.5\baselineskip}  - Lower amygdala discrimination in parent vs. stranger faces (not reduced amygdala to parent)} \\* \midrule
\raggedright\setlength{\baselineskip}{0.5\baselineskip} \cite{geeemotion2013} &
  Deprivation &
  \begin{tabular}[c]{@{}c@{}}ELS (41) \\ HC (48)\end{tabular} &
  \begin{tabular}[c]{@{}c@{}}ELS 10.8 (2.7)\\ HC 12.1 (3.3)\end{tabular} &
\setlength{\baselineskip}{0.5\baselineskip}   Cross-sectional &
\setlength{\baselineskip}{0.5\baselineskip}   Emotional Faces task &
\multicolumn{1}{m{3.5cm}}{\raggedright\setlength{\baselineskip}{0.5\baselineskip}   RT to correct hits to neutral faces in fear run: faster with increased age in HC, not significant in ELS} &
\multicolumn{1}{m{4cm}}{\raggedright\setlength{\baselineskip}{0.5\baselineskip} \hspace{-0.04cm}- Fearful face: amygdala, right superior temporal gyrus, inferior frontal gyrus ↑, not significant in medial prefrontal cortex activation  / negative amygdala-medial prefrontal cortex coupling ↑ (mature amygdala-medial prefrontal cortex  connectivity), \\ - Happy face: not significant in amygdala \& age-related change in connectivity} \\* \midrule
\raggedright\setlength{\baselineskip}{0.5\baselineskip}  \cite{greenemotion2016} &
  Deprivation &
\multicolumn{1}{m{2cm}}{\centering\setlength{\baselineskip}{0.5\baselineskip}Behavior: \\ELS (42) HC (45)\\ fMRI: ELS (33) HC (41)} &
\multicolumn{1}{m{2cm}}{\centering\setlength{\baselineskip}{0.5\baselineskip} Behavior: \\ ELS 10.26 (2.5) HC 9.43 (2.4)\\ fMRI: \\ ELS 10.6 (2.3) HC 10.51 (2.7)} &
  Longitudinal &
\setlength{\baselineskip}{0.5\baselineskip}   Face Processing task &
\setlength{\baselineskip}{0.5\baselineskip}   RT: not significant difference in untrust vs. trust &
\multicolumn{1}{m{4cm}}{\raggedright\setlength{\baselineskip}{0.5\baselineskip} Lower amygdala discrimination in trust vs. untrust faces (trust faces: amygdala ↑)} \\* \midrule
\raggedright\setlength{\baselineskip}{0.5\baselineskip} \cite{goffemotion2013} &
  Deprivation &
\multicolumn{1}{m{2cm}}{\centering\setlength{\baselineskip}{0.5\baselineskip}5-10 years old: ELS (24) HC (15)\\ 11-15 years old: ELS (14) HC (16)} &
  \begin{tabular}[c]{@{}c@{}}ELS 9.9 (2.6)\\ HC 9.7 (3.1)\end{tabular} &
  Cross-sectional &
\setlength{\baselineskip}{0.5\baselineskip}   Emotional Faces task &
\multicolumn{1}{m{3.5cm}}{\raggedright\setlength{\baselineskip}{0.5\baselineskip} \hspace{-0.04cm}- RT to correct hits to neutral faces in fear/happy run: not significant\\ - Accuracy (d-prime) in fear/happy run: not significant} &
\multicolumn{1}{m{4cm}}{\raggedright\setlength{\baselineskip}{0.5\baselineskip}   Fearful and happy face: nucleus accumbens ↓ in adolescent (no age difference in ELS)} \\* \midrule
\raggedright\setlength{\baselineskip}{0.5\baselineskip} \cite{maheuemotion2010} &
  Deprivation &
  \begin{tabular}[c]{@{}c@{}}ELS (11) \\ HC (19)\end{tabular} &
  \begin{tabular}[c]{@{}c@{}}ELS 13.75 (2.32)\\ HC 13.41 (2.70)\end{tabular} &
 \setlength{\baselineskip}{0.5\baselineskip}   Cross-sectional &
\setlength{\baselineskip}{0.5\baselineskip}   Face-Viewing  Paradigm &
\multicolumn{1}{m{3.5cm}}{\raggedright\setlength{\baselineskip}{0.5\baselineskip}  RT: angry face ↓, fear/happy/neutral face not significant} &
\multicolumn{1}{m{4cm}}{\raggedright\setlength{\baselineskip}{0.5\baselineskip} \hspace{-0.04cm}- Fearful face vs. neutral face: amygdala, left anterior hippocampus ↑\\ Angry face vs. neutral face: left amygdala, left anterior hippocampus ↑ \\ - Happy face vs. neutral face: not significant} \\* \midrule
\raggedright\setlength{\baselineskip}{0.5\baselineskip} \cite{olsavskyemotion2013} &
  Deprivation &
  \begin{tabular}[c]{@{}c@{}}ELS (33)\\ HC (34)\end{tabular} &
  \begin{tabular}[c]{@{}c@{}}ELS 11 (±4)\\ HC 10 (±3)\end{tabular} &
 \setlength{\baselineskip}{0.3\baselineskip}  Cross-sectional &
\multicolumn{1}{m{1.2cm}}{\centering\setlength{\baselineskip}{0.5\baselineskip} Parent/\newline stranger fMRI task} &
\multicolumn{1}{m{3.5cm}}{\raggedright\setlength{\baselineskip}{0.5\baselineskip}   Correct hits (to happy), errors of commission (to neutral), RT (correct trials): not significant} &
\multicolumn{1}{m{4cm}}{\raggedright\setlength{\baselineskip}{0.5\baselineskip}\hspace{-0.04cm}- Lower amygdala discrimination in parent vs. stranger faces (not higher amygdala for mother)\\ - Mother: not significant\\ - Stranger: amygdala ↑ }\\* \midrule
\raggedright\setlength{\baselineskip}{0.5\baselineskip} \cite{tottenhamemotion2010} &
  Deprivation &
  \begin{tabular}[c]{@{}c@{}}ELS (19) \\ HC (27)\end{tabular} &
  \begin{tabular}[c]{@{}c@{}}ELS 101.5 (20) \\ HC 115.3 (26.7)\\ (in months)\end{tabular} &
\setlength{\baselineskip}{0.5\baselineskip}   Cross-sectional &
\setlength{\baselineskip}{0.5\baselineskip}   Emotional GNG task &
\multicolumn{1}{m{3.5cm}}{\raggedright\setlength{\baselineskip}{0.5\baselineskip}\hspace{-0.04cm}- Accuracy ↓: negative face\newline - RT ↑: to neutral face \newline (positive distractor)\newline - RT ↓: to neutral face (negative distractor)} &
  \multicolumn{1}{c}{-} \\* \midrule
\raggedright\setlength{\baselineskip}{0.5\baselineskip} \cite{tottenhanemotion2011} &
  Deprivation &
  \begin{tabular}[c]{@{}c@{}}ELS (22) \\ HC (22)\end{tabular} &
  \begin{tabular}[c]{@{}c@{}}ELS 9.3 (2.2)\\ HC 10.9 (2.4)\end{tabular} &
\setlength{\baselineskip}{0.5\baselineskip}   Cross-sectional &
\setlength{\baselineskip}{0.5\baselineskip}   Emotional GNG task &
  Accuracy, RT: not significant &
\multicolumn{1}{m{4cm}}{\raggedright\setlength{\baselineskip}{0.5\baselineskip}Fearful face vs. baseline: right amygdala, left medial temporal gyrus, rostral anterior cingulate cortex ↑ }\\* \midrule
\raggedright\setlength{\baselineskip}{0.5\baselineskip} \cite{bellisemotion2012} &
  Mixed &
  \begin{tabular}[c]{@{}c@{}}ELS (5) \\ HC (11)\end{tabular} &
  \begin{tabular}[c]{@{}c@{}}ELS 15.5 (2.3) \\ HC 12.8 (3.3)\end{tabular} &
\setlength{\baselineskip}{0.5\baselineskip}   Cross-sectional &
\setlength{\baselineskip}{0.5\baselineskip}   Emotional Oddball task &
  Accuracy \& RT: Not significant &
\multicolumn{1}{m{4cm}}{\raggedright\setlength{\baselineskip}{0.5\baselineskip}  Sad vs. neutral (sad distractor): bilateral amygdala, left subgenual cingulate, left inferior frontal gyrus, right middle temporal gyrus ↑} \\* \midrule
\raggedright\setlength{\baselineskip}{0.5\baselineskip} \cite{dannlowskiemotion2012} &
  Mixed &
  145 participants &
  33.8 (± 10.4) &
\setlength{\baselineskip}{0.5\baselineskip}   Cross-sectional &
\setlength{\baselineskip}{0.5\baselineskip}   Face-Matching paradigm &
  Not significant &
\multicolumn{1}{m{4cm}}{\raggedright\setlength{\baselineskip}{0.5\baselineskip}\hspace{-0.04cm}- Fearful \& angry face: right amygdala ↑ \\ - Emotional abuse and emotional neglect were the strongest predictors for amygdala responsiveness} \\* \midrule
\raggedright\setlength{\baselineskip}{0.5\baselineskip} \cite{dannlowskiemotion2013} &
  Mixed &
  134 participants &
  34.5 (± 10.6) &
\setlength{\baselineskip}{0.5\baselineskip}   Cross-sectional &
\setlength{\baselineskip}{0.5\baselineskip}   Subliminal affective priming paradigm &
  \multicolumn{1}{c}{-} &
\multicolumn{1}{m{4cm}}{\raggedright\setlength{\baselineskip}{0.5\baselineskip}   Sad vs. neutral face: amygdala ↑} \\* \midrule
\raggedright\setlength{\baselineskip}{0.5\baselineskip} \cite{fonzoemotion2016} &
  Mixed &
\multicolumn{1}{m{2cm}}{\centering\setlength{\baselineskip}{0.5\baselineskip} PTSD \& ELS 22\\ PTSD 20\\ HC \& ELS 20} &
\multicolumn{1}{m{2cm}}{\centering\setlength{\baselineskip}{0.5\baselineskip}PTSD \& ELS 39.71 (10.8)\\ PTSD \\ 37.09 (10.37)\\ HC \& ELS \\ 32.48 (12.45)}&
\setlength{\baselineskip}{0.5\baselineskip}   Cross-sectional &
\setlength{\baselineskip}{0.5\baselineskip}   Emotional Conflict task &
  Accuracy \& RT: Not significant &
\multicolumn{1}{m{4cm}}{\raggedright\setlength{\baselineskip}{0.5\baselineskip}   Not significant in both ROI analysis and whole-brain analysis} \\* \midrule
\raggedright\setlength{\baselineskip}{0.5\baselineskip} \cite{harmelenemotion2013} &
  Mixed &
  \begin{tabular}[c]{@{}c@{}}ELS (60) \\ HC (75)\end{tabular} &
  \begin{tabular}[c]{@{}c@{}}ELS 38.22  (1.32) \\ HC 34.91  (1.22)\end{tabular} &
\setlength{\baselineskip}{0.5\baselineskip}   Cross-sectional &
\setlength{\baselineskip}{0.5\baselineskip}   The Faces task &
  RT: not significant &
\multicolumn{1}{m{4cm}}{\raggedright\setlength{\baselineskip}{0.5\baselineskip}   Angry,  fearful, sad, happy, neutral face: amygdala ↑ } \\* \midrule
\raggedright\setlength{\baselineskip}{0.5\baselineskip} \cite{holzemotion2017} &
  Mixed &
  171 participants &
  25 &
  Longitudinal &
\setlength{\baselineskip}{0.5\baselineskip}   Face Matching task &
  \multicolumn{1}{c}{-} &
\multicolumn{1}{m{4cm}}{\raggedright\setlength{\baselineskip}{0.5\baselineskip}   Fearful/angry faces: amygdala ↓} \\* \midrule
\raggedright\setlength{\baselineskip}{0.5\baselineskip} \cite{mccroryemotion2013} &
  Mixed &
  \begin{tabular}[c]{@{}c@{}}ELS (18) \\ HC (23)\end{tabular} &
  \begin{tabular}[c]{@{}c@{}}ELS 12.17 (1.43) \\ HC 12.55 (1.22)\end{tabular} &
\setlength{\baselineskip}{0.5\baselineskip}   Cross-sectional &
\setlength{\baselineskip}{0.5\baselineskip}   Masked dot-probe paradigm  (angry, happy and neutral) &
  Accuracy, RT: not significant &
\multicolumn{1}{m{4cm}}{\raggedright\setlength{\baselineskip}{0.5\baselineskip}\hspace{-0.04cm}- Angry vs. neutral face: right amygdala, cerebellum ↑\\ - Happy vs. neutral face: right amygdala, thalamus, pallidum ↑, temporal pole/middle temporal gyrus, temporal pole/superior temporal gyrus ↓} \\* \midrule
\raggedright\setlength{\baselineskip}{0.5\baselineskip} \cite{metzemotion2018} &
  Mixed &
  \begin{tabular}[c]{@{}c@{}}ELS (15) \\ HC (16)\end{tabular} &
  \begin{tabular}[c]{@{}c@{}}ELS 42.13 \\ (± 11.04)\\ HC 34.38 \\ (± 11.33)\end{tabular} &
\setlength{\baselineskip}{0.5\baselineskip}   Cross-sectional &
\setlength{\baselineskip}{0.5\baselineskip}   Emotional (word) N-Back task &
\multicolumn{1}{m{3.5cm}}{\raggedright\setlength{\baselineskip}{0.5\baselineskip}\hspace{-0.04cm}- Accuracy ↓ (across all targets)\\- Negative vs. positive accuracy ↑ in ELS group \\- RT: not significant} &
\multicolumn{1}{m{4cm}}{\raggedright\setlength{\baselineskip}{0.5\baselineskip}\hspace{-0.04cm}- Not significant in dorsolateral prefrontal cortex\\ - neutral than negative: left posterior cingulate cortex, precuneus ↓ only in ELS group\\ - negative than positive, negative than neutral: rostral anterior cingulate cortex ↓ only in ELS group} \\* \midrule
\raggedright\setlength{\baselineskip}{0.5\baselineskip} \cite{tayloremotion2006} &
  Mixed &
\setlength{\baselineskip}{0.5\baselineskip}   30 participants (ELS low/ high by median) &
  18 - 36 &
\multicolumn{1}{m{1.5cm}}{\centering\setlength{\baselineskip}{0.5\baselineskip}Cross-sectional\\ (2-6 weeks after)}&
\setlength{\baselineskip}{0.5\baselineskip}   Observation and labeling task &
  Accuracy, RT: not significant &
\multicolumn{1}{m{4cm}}{\raggedright\setlength{\baselineskip}{0.5\baselineskip}Fearful and angry face (observation): amygdala ↓} \\* \bottomrule
\end{longtable}
\renewcommand{\baselinestretch}{1}
\small
\raggedright
\parbox{\linewidth}{ 
\textit{Note.} The criteria for distinguishing between cross-sectional and longitudinal designs were based on the timing of the administration of ELS and neurocognitive function measures. \\
\textit{Abbreviations.} ELS, Early Life Stress; HC, Healthy Control; emotional GNG task, emotional go-nogo task; RT, Reaction Time
}
\end{landscape}

% \begin{landscape}
% \begin{longtable}{c} % longtable 사용, 단일 열 테이블 설정
%     \caption{Summary of findings: early life stress and emotion processing} \\ % 캡션 번호 유지
%     \label{tab:summary_emo} \\
%     % 첫 번째 이미지
%     \includegraphics[width=\linewidth]{summary_emo_1.png} \\
%     \caption[]{(Continued)} \\ 
%     % 두 번째 이미지
%     \includegraphics[width=\linewidth]{summary_emo_2.png} \\
%     % 세 번째 이미지
%     \caption[]{(Continued)} \\ 
%     \includegraphics[width=\linewidth]{summary_emo_3.png} \\

% \end{longtable}
% \renewcommand{\baselinestretch}{1}
% \small
% \raggedright
% \parbox{\linewidth}{ 
% \textit{Note.} The criteria for distinguishing between cross-sectional and longitudinal designs were based on the timing of the administration of ELS and neurocognitive function measures. \\
% \textit{Abbreviations.} ELS, Early Life Stress; HC, Healthy Control; emotional GNG task, emotional go-nogo task; RT, Reaction Time
% }
% \end{landscape}

\subsubsection{Working memory}
Working memory is a critical component of cognitive processes and is implicated in a range of psychiatric disorders, including attention-deficit/hyperactivity disorder (ADHD) and schizophrenia (\cite{leeparkemotion2005, martinussenemotion2005}). A substantial body of research highlights the negative effects of ELS on working memory (\cite{goodmanwm2019}). Successful working memory performance involves increased activity in the central executive network (CEN) and decreased activity in the default mode network (DMN) (\cite{satterthwaiteemotion2013}).

Most previous studies employing the N-Back task to assess working memory have utilized mixed ELS samples, yielding inconsistent neuroimaging findings. For example, one study reported greater DMN deactivation, particularly in the posterior cingulate cortex (PCC), middle temporal gyrus (MTG), and inferior parietal lobe (IPL), in individuals with ELS compared to healthy controls (\cite{philipwm2013}). However, these findings were not replicated in a subsequent study with a slightly larger sample size (\cite{philipwm2016}), and one study reported increased left IPL activation in a non-clinical ELS group (\cite{quidewm2017}). Notably, the IPL is also considered part of CEN, thus reduced IPL activation might reflect poorer working memory performance. Regarding other CEN regions, such as the dorsolateral prefrontal cortex (dlPFC) and middle frontal gyrus (MFG), one study reported reduced caudal MFG activation in mixed ELS samples (\cite{hallowellimpul2019}), while another study found no significant results in the dlPFC (\cite{metzemotion2018}).

Behavioral assessments typically reveal diminished working memory performance in individuals with ELS (\cite{danesewm2017, fugewm2014, majerwm2010, metzemotion2018, philipwm2016, quidewm2017, violawm2013}). Nonetheless, some studies report no significant differences compared to healthy controls (\cite{philipwm2013, ucokwm2015}). However, recent studies revealed that deprivation is more associated with poor working memory or executive function than threat (\cite{Schäfer2023, johnson2021}).

Findings regarding ELS and working memory remain inconsistent, with limited studies distinguishing the effects of threat and deprivation. Further research is needed to clarify how these distinct dimensions of ELS influence brain activation and performance. (Please refer to Table~\ref{tab:summary_wm} for further details on the results of working memory.)

\begin{landscape}
\tiny
\renewcommand{\arraystretch}{0.5} % 행 높이를 더 줄임
% \setlength{\extrarowheight}{0pt}
% \setlength{\tabcolsep}{2pt}
\begin{longtable}{@{}m{1cm}>{\centering\arraybackslash}m{1.3cm}>{\centering\arraybackslash}m{2cm}>{\centering\arraybackslash}m{2cm}>{\centering\arraybackslash}m{1.5cm}>{\centering\arraybackslash}m{1.2cm}m{3.5cm}m{4cm}@{}}
% \begin{longtable}{@{}p{1.5cm}>{\centering\arraybackslash}p{1cm}>{\centering\arraybackslash}p{2cm}>{\centering\arraybackslash}>{\centering\arraybackslash}p{2cm}>{\centering\arraybackslash}p{1cm}>{\centering\arraybackslash}p{1.5cm}p{3cm}p{4.5cm}@{}}
\captionsetup{justification=raggedright}
\caption{Summary of previous results: early life stress and working memory}
\label{tab:summary_wm}\\
\toprule
\multicolumn{1}{c}{Study} &
\begin{tabular}[c]{@{}c@{}} ELS \\ dimension\end{tabular} &
\multicolumn{1}{c}{Participants (N)} &
\multicolumn{1}{c}{Age: mean (SD)} &
\multicolumn{1}{c}{Design} &
\multicolumn{1}{c}{Measurement} &
\multicolumn{1}{c}{Behavioral results} &
\multicolumn{1}{c}{Neural results}\\* \midrule
\endfirsthead

\captionsetup{list=no}
\caption{(Continued)}\\
\toprule
\multicolumn{1}{c}{Study} &
\begin{tabular}[c]{@{}c@{}} ELS \\ dimension\end{tabular} &
\multicolumn{1}{c}{Participants (N)} &
\multicolumn{1}{c}{Age: mean (SD)} &
\multicolumn{1}{c}{Design} &
\multicolumn{1}{c}{Measurement} &
\multicolumn{1}{c}{Behavioral results} &
\multicolumn{1}{c}{Neural results}\\* \midrule
\endhead


%
\raggedright\setlength{\baselineskip}{0.5\baselineskip} \cite{johnson2021} &
  \begin{tabular}[c]{@{}c@{}}Threat/\\ Deprivation\end{tabular} &
  Review &
  \multicolumn{1}{c}{-} &
  \multicolumn{1}{c}{-} &
  \multicolumn{1}{c}{-} &
 \multicolumn{1}{m{3.5cm}}{\raggedright\setlength{\baselineskip}{0.5\baselineskip} Working memory ↓ (deprivation \textless threat)} &
    \multicolumn{1}{c}{-} \\* \midrule
\raggedright\setlength{\baselineskip}{0.5\baselineskip} \cite{majerwm2010} &
  \begin{tabular}[c]{@{}c@{}}Threat/\\ deprivation\end{tabular} &
  47 participants &
  51.51 (1.22) &
\setlength{\baselineskip}{0.5\baselineskip}  Cross-sectional &
\setlength{\baselineskip}{0.5\baselineskip} CANTAB Spatial WM task &
 \multicolumn{1}{m{3.5cm}}{\raggedright\setlength{\baselineskip}{0.5\baselineskip}\hspace{-0.04cm}- Threat: double errors ↑ in emotional abuse, physical abuse, not significant sexual abuse\\ - Deprivation: double errors  ↑ physical neglect, not significant  emotional neglect} &
  \multicolumn{1}{c}{-} \\* \midrule
\raggedright\setlength{\baselineskip}{0.5\baselineskip} \cite{ucokwm2015} &
  \begin{tabular}[c]{@{}c@{}}Threat/\\ Deprivation\end{tabular} &
\setlength{\baselineskip}{0.5\baselineskip}   Ultra-high risk for psychosis (53) &
  21.1 (4.8) &
\setlength{\baselineskip}{0.5\baselineskip}   Cross-sectional &
\multicolumn{1}{m{1.2cm}}{\centering\setlength{\baselineskip}{0.5\baselineskip}  N-Back task \\ (alphabet)} &
\multicolumn{1}{m{3.5cm}}{\raggedright\setlength{\baselineskip}{0.5\baselineskip}  Accuracy: not significant in all category} &
  \multicolumn{1}{c}{-} \\* \midrule
\raggedright\setlength{\baselineskip}{0.5\baselineskip} \cite{wolfwm2019} &
  \begin{tabular}[c]{@{}c@{}}Threat/\\ deprivation\end{tabular} &
  18200 participants &
  5.6 &
\setlength{\baselineskip}{0.5\baselineskip}    Cross-sectional &
\setlength{\baselineskip}{0.5\baselineskip}   Numbers reversed task &
\multicolumn{1}{m{3.5cm}}{\raggedright\setlength{\baselineskip}{0.5\baselineskip} \hspace{-0.04cm}- Threat, deprivation: not significant\\- Cumulative deprivation: ↓} &
 \multicolumn{1}{c}{-} \\* \midrule
\raggedright\setlength{\baselineskip}{0.5\baselineskip} \cite{violawm2013} &
  Deprivation &
  \multicolumn{1}{m{2cm}}{\centering\setlength{\baselineskip}{0.5\baselineskip} Female substance abuser: ELS (37)\\ nonELS (48)} &
 \multicolumn{1}{m{2cm}}{\centering\setlength{\baselineskip}{0.5\baselineskip} ELS 31.51 (7.83)\\ nonELS \\28.89 (8.72)} &
\setlength{\baselineskip}{0.5\baselineskip}    Cross-sectional &
\setlength{\baselineskip}{0.5\baselineskip}  Auditory N-Back task (number) &
\multicolumn{1}{m{3.5cm}}{\raggedright\setlength{\baselineskip}{0.5\baselineskip} Accuracy: not significant in 1back, \\ ↓ in 2back, 3back} &
 \multicolumn{1}{c}{-}  \\* \midrule
\raggedright\setlength{\baselineskip}{0.5\baselineskip}\cite{buckerwm2013} &
  Mixed &
  \multicolumn{1}{m{2cm}}{\centering\setlength{\baselineskip}{0.5\baselineskip}Bipolar-ELS (26)\\ Bipolar-nonELS (38)\\ HC-ELS (9)\\ HC-nonELS (19)} &
  \multicolumn{1}{m{2cm}}{\centering\setlength{\baselineskip}{0.5\baselineskip}Bipolar-ELS 22.57 (± 1.91)\\ Bipolar-nonELS  23.10 (± 5.05)\\ HC \\ 22.78 (± 4.90)} &
\setlength{\baselineskip}{0.5\baselineskip}  Cross-sectional &
  \multicolumn{1}{m{1.5cm}}{\centering\setlength{\baselineskip}{0.5\baselineskip} Letter-number sequencing, CANTAB Spatial WM task} &
\multicolumn{1}{m{3.5cm}}{\raggedright\setlength{\baselineskip}{0.5\baselineskip} \hspace{-0.04cm}- Letter-number sequencing: not significant\\ - Spatial WM ↓ only in bipolar group} &
  \multicolumn{1}{c}{-}\\* \midrule
\raggedright\setlength{\baselineskip}{0.5\baselineskip} \cite{danesewm2017} &
  Mixed &
  2044 participants &
  18, 38 &
  Longitudinal &
\multicolumn{1}{m{1.5cm}}{\centering\setlength{\baselineskip}{0.5\baselineskip}   CANTAB Spatial WM task, Digit span forward, Digit span backward} &
  Working memory ↓ &
 \multicolumn{1}{c}{-} \\* \midrule
\raggedright\setlength{\baselineskip}{0.5\baselineskip} \cite{fugewm2014} &
  Mixed &
  \multicolumn{1}{m{2.3cm}}{\centering\setlength{\baselineskip}{0.5\baselineskip} 541 participants\\ No ELS (96)\\ Low ELS (109)\\ Moderate ELS (128)\\ Severe ELS (118)} &
  \multicolumn{1}{m{2cm}}{\centering\setlength{\baselineskip}{0.5\baselineskip} No ELS 36 (± 17)\\ Low ELS \\41.6 (±  19.4)\\ Moderate ELS 41.3 (± 19.1)\\ Severe ELS\\43.5 (± 17.2)} &
\setlength{\baselineskip}{0.5\baselineskip}   Cross-sectional &
\setlength{\baselineskip}{0.5\baselineskip}   N-Back task (Word) &
  Accuracy↓ in 2 back &
  \multicolumn{1}{c}{-} \\* \midrule
\raggedright\setlength{\baselineskip}{0.5\baselineskip} 
 \cite{hallowellimpul2019} &
  Mixed &
  ELS (30) &
  20.63 (2.2) &
\setlength{\baselineskip}{0.5\baselineskip}   Cross-sectional &
\multicolumn{1}{m{1.2cm}}{\centering\setlength{\baselineskip}{0.5\baselineskip}  N-Back task \\ (alphabet)}  &
  Accuracy: ↓ in 2 back &
  \multicolumn{1}{m{4cm}}{\raggedright\setlength{\baselineskip}{0.5\baselineskip} 2back vs. 0back: posterior parietal lobule, left caudal middle frontal gyrus ↓} \\* \midrule
\raggedright\setlength{\baselineskip}{0.5\baselineskip} 
 \cite{metzemotion2018} &
  Mixed &
  \multicolumn{1}{m{2cm}}{\centering\setlength{\baselineskip}{0.5\baselineskip} ELS (15) \\ HC (16)} &
  \begin{tabular}[c]{@{}c@{}}ELS 42.13 \\ (± 11.04)\\ HC 34.38 \\ (± 11.33)\end{tabular} &
\setlength{\baselineskip}{0.5\baselineskip}  Cross-sectional &
\setlength{\baselineskip}{0.5\baselineskip}  Emotional (word) N-Back task &
\multicolumn{1}{m{3.5cm}}{\raggedright\setlength{\baselineskip}{0.5\baselineskip} \hspace{-0.04cm}- Accuracy ↓ (across all targets)\newline - RT: not significant} &
\multicolumn{1}{m{4cm}}{\raggedright\setlength{\baselineskip}{0.5\baselineskip} \hspace{-0.04cm}- Not significant in dorsolateral prefrontal cortex\\ - Neutral than negative: left posterior cingulate cortex, precuneus ↓ only in ELS group\\ - Negative than positive, negative than neutral: rostral anterior cingulate cortex ↓ only in ELS group} \\* \midrule
\raggedright\setlength{\baselineskip}{0.5\baselineskip} 
 \cite{philipwm2013} &
  Mixed &
    \multicolumn{1}{m{2cm}}{\centering\setlength{\baselineskip}{0.5\baselineskip} ELS (10) \\ HC (9)}&
  \begin{tabular}[c]{@{}c@{}}ELS 37.4 (12)\\ HC 39.7 (15)\end{tabular} &
\setlength{\baselineskip}{0.5\baselineskip}  Cross-sectional &
\multicolumn{1}{m{1.2cm}}{\centering\setlength{\baselineskip}{0.5\baselineskip}  N-Back task \\ (alphabet)}  &
  Accuracy. RT: not significant &
\multicolumn{1}{m{4cm}}{\raggedright\setlength{\baselineskip}{0.5\baselineskip} 2back: right posterior cingulate cortex, medial prefrontal cortex, left middle frontal gyrus, right middle temporal region ↓ (greater default mode network deactivation in ELS)} \\* \midrule
\raggedright\setlength{\baselineskip}{0.5\baselineskip}  \cite{philipwm2016} &
  Mixed &
  \multicolumn{1}{m{2cm}}{\centering\setlength{\baselineskip}{0.5\baselineskip} ELS (13) \\ HC (13)}
  &
  \begin{tabular}[c]{@{}c@{}}ELS 38 (9)\\ HC 30 (9)\end{tabular} 
  &
\setlength{\baselineskip}{0.5\baselineskip}   Cross-sectional &
\multicolumn{1}{m{1.2cm}}{\centering\setlength{\baselineskip}{0.5\baselineskip}  N-Back task \\ (alphabet)} &
\multicolumn{1}{m{3.5cm}}{\raggedright\setlength{\baselineskip}{0.5\baselineskip} \hspace{-0.04cm}- Accuracy: ↓ in 2 back, not significant in 0 back \\ - RT: not significant in 0 back, 2 back} &
\multicolumn{1}{m{4cm}}{\raggedright\setlength{\baselineskip}{0.5\baselineskip} \hspace{-0.04cm}- 2back: right inferior frontal gyrus, right inferior parietal lobule, left superior temporal gyrus/inferior parietal lobule, left posterior cingulate cortex/precuneus ↑ (ELS vs. nonELS); right inferior frontal gyrus, left middle temporal gyrus/inferior parietal lobule ↑ as ELS severity ↑ (not significant between subtype severity)} \\* \midrule
\raggedright\setlength{\baselineskip}{0.5\baselineskip} 
 \cite{quidewm2017} &
  Mixed &
    \multicolumn{1}{m{2cm}}{\centering\setlength{\baselineskip}{0.5\baselineskip} Clinical-ELS (56) \\ Clinical-nonELS (36) \\ non-clinical ELS (17) \\ HC (30)}&
    \multicolumn{1}{m{2cm}}{\centering\setlength{\baselineskip}{0.5\baselineskip} Clinical-ELS 37.4 (12.1) \\ Clinical-nonELS 38.3 (12) \\ ELS-HC 44 (12.5) \\ HC 35.5 (8.4)} &
\setlength{\baselineskip}{0.5\baselineskip}  Cross-sectional &
\setlength{\baselineskip}{0.5\baselineskip}  N-Back task (number) &
\multicolumn{1}{m{3.5cm}}{\raggedright\setlength{\baselineskip}{0.5\baselineskip} \hspace{-0.04cm}- Accuracy: ↓ in 2back as ELS severity ↑ regardless of clinical status\\ - RT: not significant} &
\multicolumn{1}{m{4cm}}{\raggedright\setlength{\baselineskip}{0.5\baselineskip} \hspace{-0.04cm}- Not significant in ROI analysis\\ - 2back vs 0back: left Inferior parietal lobule ↑ (ELS vs. nonELS), cuneus, calcarine sulcus, lingual gyrus cuneus ↑ (clinical ELS vs. clinical-nonELS), calcarine sulcus, lingual gyrus ↓ (non-clinical ELS vs. HC)} \\* \bottomrule
\end{longtable}
\renewcommand{\baselinestretch}{1}
\small
\raggedright
\parbox{\linewidth}{ 
\textit{Note.} The criteria for distinguishing between cross-sectional and longitudinal designs were based on the timing of the administration of ELS and neurocognitive function measures. \\
\textit{Abbreviations.} ELS, Early Life Stress; HC, Healthy Control; ROI, Regions of Interest; RT, Reaction Time
}
\end{landscape}
  


% \begin{landscape}
% \begin{longtable}{c} % longtable 사용, 단일 열 테이블 설정
%     \caption{Summary of findings: early life stress and working memory} \\ % 캡션 번호 유지
%     \label{tab:summary_wm} \\
%     % 첫 번째 이미지
%     \includegraphics[width=\linewidth]{summary_wm_1.png} \\
%     \caption[]{(Continued)} \\ 
%     % 두 번째 이미지
%     \includegraphics[width=\linewidth]{summary_wm_2.png} \\

% \end{longtable}
% \renewcommand{\baselinestretch}{1}
% \small
% \raggedright
% \parbox{\linewidth}{ 
% \textit{Note.} The criteria for distinguishing between cross-sectional and longitudinal designs were based on the timing of the administration of ELS and neurocognitive function measures. \\
% \textit{Abbreviations.} ELS, Early Life Stress; HC, Healthy Control; ROI, Regions of Interest; RT, Reaction Time
% }
% \end{landscape}

\subsubsection{Impulsivity}

Impulsivity has been proposed as a potential mechanism linking ELS and psychopathology, such as alcohol dependence (\cite{kimimpul2018, liuimpul2019}). Previous studies on impulsivity and ELS have primarily utilized mixed ELS measures, making it difficult to separately review the effects of threat and deprivation. Impulsivity manifests in various dimensions: impulsive action (motor impulsivity), impulsive choice (cognitive impulsivity, including delay discounting), and impulsive personality traits (\cite{mackillopimpul2016}). This section reviews the findings in terms of impulsivity dimensions. 

In the context of impulsive actions (e.g., Stop Signal Task; SST), inconsistent findings have been reported. Some studies observed longer reaction time (RT), which are indicative of compromised inhibitory control in individuals with ELS (\cite{merzimpul2013, muellerimpul2010, limimpul2015}). However, others found no significant differences in RT or stop accuracy when comparing individuals with ELS to controls (\cite{carrionimpul2008, eltonimpul2014, hartimpul2012, limimpul2015, mezzacappaimpul2001, hamiltonimpul2013}).

As for the neural correlates of impulsive action (e.g., during successful inhibition), research shows mixed results. Hyperactivation in inhibitory regions of the brain, such as the inferior frontal gyrus (IFG), anterior cingulate cortex (ACC), and striatum, has been observed during successful inhibition (\cite{muellerimpul2010}). In contrast, other studies have reported no significant neural differences (\cite{eltonimpul2014, liuimpul2019}), while some have found decreased activation in these regions (\cite{bruceimpul2013}).

Regarding choice impulsivity, a majority of studies report an elevated discounting rate in individuals with ELS (\cite{achesonimpul2019, simmenimpul2015, oshriimpul2018}). However, one study found no significant correlation between discounting rate and the severity of ELS (\cite{hamiltonimpul2013}).

Trait impulsivity, as assessed through self-report measures, consistently appears elevated in individuals with ELS across studies (\cite{hallowellimpul2019, liuimpul2019, yangimpul2024}). Among its subscales, negative urgency—defined as the tendency to act impulsively under distress—has been consistently reported as increased. Findings for other subscales, however, remain inconsistent (\cite{weissimpul2023, gagnonimpul2013, shinimpul2018, liuimpul2019}). (Please refer to Table~\ref{tab:summary_imp} for further details on the results related to impulsivity.)


\begin{landscape}
\tiny
\renewcommand{\arraystretch}{0.5} % 행 높이를 더 줄임
% \setlength{\extrarowheight}{0pt}
% \setlength{\tabcolsep}{2pt}
\begin{longtable}{@{}m{1cm}>{\centering\arraybackslash}m{1.3cm}>{\centering\arraybackslash}m{2cm}>{\centering\arraybackslash}m{2cm}>{\centering\arraybackslash}m{1.5cm}>{\centering\arraybackslash}m{1.2cm}
>{\centering\arraybackslash}m{1.5cm}m{3cm}m{3.5cm}@{}}
% \begin{longtable}{@{}p{1.5cm}>{\centering\arraybackslash}p{1cm}>{\centering\arraybackslash}p{2cm}>{\centering\arraybackslash}>{\centering\arraybackslash}p{2cm}>{\centering\arraybackslash}p{1cm}>{\centering\arraybackslash}p{1.5cm}p{3cm}p{4.5cm}@{}}
\captionsetup{justification=raggedright}
\caption{Summary of previous results: early life stress and impulsivity}
\label{tab:summary_imp}\\
\toprule
\multicolumn{1}{c}{Study} &
\begin{tabular}[c]{@{}c@{}} ELS \\ dimension\end{tabular} &
\multicolumn{1}{c}{Participants (N)} &
\multicolumn{1}{c}{Age: mean (SD)} &
\multicolumn{1}{c}{Design} &
\begin{tabular}[c]{@{}c@{}}Impulsivity \\ dimension\end{tabular} &
\multicolumn{1}{c}{Measurement} &
\multicolumn{1}{c}{Behavioral results} &
\multicolumn{1}{c}{Neural results}\\* \midrule
\endfirsthead


\captionsetup{list=no}
\caption{(Continued)}\\
\toprule
\multicolumn{1}{c}{Study} &
\begin{tabular}[c]{@{}c@{}} ELS \\ dimension\end{tabular} &
\multicolumn{1}{c}{Participants (N)} &
\multicolumn{1}{c}{Age: mean (SD)} &
\multicolumn{1}{c}{Design} &
\begin{tabular}[c]{@{}c@{}}Impulsivity \\ dimension\end{tabular} &
\multicolumn{1}{c}{Measurement} &
\multicolumn{1}{c}{Behavioral results} &
\multicolumn{1}{c}{Neural results}\\* \midrule
\endhead

%
\raggedright\setlength{\baselineskip}{0.5\baselineskip} \cite{auebachimpulsitivy2014} &
  Threat &
  194 participants &
  15.53 (1.34) &
\setlength{\baselineskip}{0.5\baselineskip}  Cross-sectional &
\setlength{\baselineskip}{0.5\baselineskip}  Impulsive action &
\setlength{\baselineskip}{0.5\baselineskip}  Continuous performance Task &
\multicolumn{1}{m{3cm}}{\raggedright\setlength{\baselineskip}{0.5\baselineskip}Commission error: not significant} &
  \multicolumn{1}{c}{-}\\* \midrule
\raggedright\setlength{\baselineskip}{0.5\baselineskip}\cite{carrionimpul2008} &
  Threat &
  \begin{tabular}[c]{@{}c@{}}ELS (16) \\ HC (14)\end{tabular} &
  13.7 &
\setlength{\baselineskip}{0.5\baselineskip}  Cross-sectional &
\setlength{\baselineskip}{0.5\baselineskip}  Impulsive action &
\setlength{\baselineskip}{0.5\baselineskip}  Go/Nogo task (alphabet) &
\multicolumn{1}{m{3cm}}{\raggedright\setlength{\baselineskip}{0.5\baselineskip}Accuracy, RT: not significant} &
\multicolumn{1}{m{3.5cm}}{\raggedright\setlength{\baselineskip}{0.5\baselineskip} Nogo vs. go (stop vs. go): left cuneus, left inferior occipital, left inferior temporal gyri, medial frontal gyrus, anterior cingulate cortex ↑} \\* \midrule
\raggedright\setlength{\baselineskip}{0.5\baselineskip} \cite{mezzacappaimpul2001} &
  Threat &
\multicolumn{1}{m{2cm}}{\centering\setlength{\baselineskip}{0.5\baselineskip}ELS (25) (therapeutic school/abused) \\ therapeutic/\\ non abused (52) \\ HC (48)} &
  10.5 (2.1) &
\setlength{\baselineskip}{0.5\baselineskip}  Cross-sectional &
\setlength{\baselineskip}{0.5\baselineskip}  Impulsive action &
\setlength{\baselineskip}{0.5\baselineskip}  Stop signal task &
\multicolumn{1}{m{3cm}}{\raggedright\setlength{\baselineskip}{0.5\baselineskip}  Stop accuracy: ↓} &
 \multicolumn{1}{c}{-} \\* \midrule
\raggedright\setlength{\baselineskip}{0.5\baselineskip}\cite{lambertemotion2017} &
  \begin{tabular}[c]{@{}c@{}}Threat/\\ deprivation\end{tabular} &
  287 participants &
  16-17 &
\setlength{\baselineskip}{0.5\baselineskip}   Cross-sectional &
\setlength{\baselineskip}{0.5\baselineskip}   Impulsive action &
\setlength{\baselineskip}{0.5\baselineskip}   The arrows task &
\multicolumn{1}{m{3cm}}{\raggedright\setlength{\baselineskip}{0.5\baselineskip}\hspace{-0.04cm}- Threat: not significant inhibition or switching\\ - Deprivation: ↓ inhibition and switching} &
 \multicolumn{1}{c}{-}  \\* \midrule
\raggedright\setlength{\baselineskip}{0.5\baselineskip}\cite{wolfwm2019}  &
  \begin{tabular}[c]{@{}c@{}}Threat/\\ deprivation\end{tabular} &
\setlength{\baselineskip}{0.5\baselineskip}  18200 participants &
  5.6 &
\setlength{\baselineskip}{0.5\baselineskip}  Cross-sectional &
\setlength{\baselineskip}{0.5\baselineskip}  inhibitory control &
\setlength{\baselineskip}{0.5\baselineskip}  teacher reported survey &
\multicolumn{1}{m{3cm}}{\raggedright\setlength{\baselineskip}{0.5\baselineskip}  Threat, deprivation: inhibitory control ↓} &
  \multicolumn{1}{c}{-}\\* \midrule
\raggedright\setlength{\baselineskip}{0.5\baselineskip}\cite{merzimpul2013} &
  Deprivation &
\multicolumn{1}{m{2cm}}{\centering\setlength{\baselineskip}{0.5\baselineskip} ELS (41 adopted ≤9 months \\ 34 adopted \\ ≥ 14 months) \\ HC (133)} &
  \begin{tabular}[c]{@{}c@{}}ELS 12.97 (3.03) \\ HC 12.26 (2.75)\end{tabular} &
\setlength{\baselineskip}{0.5\baselineskip}  Cross-sectional &
\setlength{\baselineskip}{0.5\baselineskip}  Impulsive action &
\setlength{\baselineskip}{0.5\baselineskip}  Stop signal task &
\multicolumn{1}{m{3cm}}{\raggedright\setlength{\baselineskip}{0.5\baselineskip}SSRT: ↑ (adopted after 14 months \textgreater before 9 months) }&
  \multicolumn{1}{c}{-} \\* \midrule
\raggedright\setlength{\baselineskip}{0.5\baselineskip}\cite{bruceimpul2013} &
  Mixed &
 \multicolumn{1}{m{2cm}}{\centering\setlength{\baselineskip}{0.5\baselineskip} ELS (11) \\ HC (11)} &
  10.91 (0.90) &
\setlength{\baselineskip}{0.5\baselineskip}  Cross-sectional &
\setlength{\baselineskip}{0.5\baselineskip}  Impulsive action &
\setlength{\baselineskip}{0.5\baselineskip}  Go/Nogo task &
\multicolumn{1}{m{3cm}}{\raggedright\setlength{\baselineskip}{0.5\baselineskip}  No-go accuracy, go accuracy: not significant} &
\multicolumn{1}{m{3.5cm}}{\raggedright\setlength{\baselineskip}{0.5\baselineskip}\hspace{-0.04cm}- Correct no go vs. correct go: right anterior cingulate cortex, right middle frontal, right lingual, cuneus ↓\\ - Incorrect no go vs. correct no go: left inferior parietal lobule, right precunues, right cuneus, right lingual gyrus ↓} \\* \midrule
\raggedright\setlength{\baselineskip}{0.5\baselineskip}\cite{eltonimpul2014}&
  Mixed &
  40 participants &
  29.6 (7.9) &
\setlength{\baselineskip}{0.5\baselineskip}  Cross-sectional &
\setlength{\baselineskip}{0.5\baselineskip}  Impulsive action &
\setlength{\baselineskip}{0.5\baselineskip}  Stop signal task &
\multicolumn{1}{m{3cm}}{\raggedright\setlength{\baselineskip}{0.5\baselineskip}   Stop accuracy, SSRT: not significant} &
\multicolumn{1}{m{3.5cm}}{\raggedright\setlength{\baselineskip}{0.5\baselineskip}Successful stop vs go (not following a stop trials): inferior frontal gyrus, dorsal anterior cingulate cortex, middle cingulate, supramarginal not significant} \\* \midrule
\raggedright\setlength{\baselineskip}{0.5\baselineskip}\cite{hamiltonimpul2013} &
  Mixed &
  ELS (192) &
  30.42 (9.3) &
\setlength{\baselineskip}{0.5\baselineskip}   Cross-sectional &
\setlength{\baselineskip}{0.5\baselineskip}   Impulsive action &
\setlength{\baselineskip}{0.5\baselineskip}   Go/Nogo task &
\multicolumn{1}{m{3cm}}{\raggedright\setlength{\baselineskip}{0.5\baselineskip}  Nogo accuracy: not significant} &
  \multicolumn{1}{c}{-} \\* \midrule
\raggedright\setlength{\baselineskip}{0.5\baselineskip}\cite{hartimpulsivity2018} &
  Mixed &
  \begin{tabular}[c]{@{}c@{}}ELS (23) \\ HC (27)\end{tabular} &
  \begin{tabular}[c]{@{}c@{}}ELS 17.2 (2.44)\\ HC 17.5 (1.63)\end{tabular} &
\setlength{\baselineskip}{0.5\baselineskip}   Cross-sectional &
\setlength{\baselineskip}{0.5\baselineskip}   Impulsive action &
\setlength{\baselineskip}{0.5\baselineskip}   Stop signal task &
\multicolumn{1}{m{3cm}}{\raggedright\setlength{\baselineskip}{0.5\baselineskip} \hspace{-0.04cm}- Stop accuracy: not significant\\- Stop signal reaction time (SSRT): not significant\\- Mean RT, post-error RT↑} &
\multicolumn{1}{m{3.5cm}}{\raggedright\setlength{\baselineskip}{0.5\baselineskip} Unsuccessful stop vs. go: ↓ connectivity right putamen between caudate, right anterior cingulate cortex, left putamen, ↓ connectivity right supplementary motor area between right dorsomedial prefrontal cortex, right  dorsolateral prefrontal cortex, right  inferior frontal gyrus} \\* \midrule
\raggedright\setlength{\baselineskip}{0.5\baselineskip}\cite{liimpulsivity2013} &
  Mixed &
  \begin{tabular}[c]{@{}c@{}}PVP (53)\\ PVnP (64)\\ nPV (68) \\ nV (74)\end{tabular} &
  \begin{tabular}[c]{@{}c@{}}PVP 19.66 (0.85)\\ PVnP 19.55 (0.78)\\ nPV 19.68 (0.80)\\ nV 19.74 (0.84)\end{tabular} &
\setlength{\baselineskip}{0.5\baselineskip}  Cross-sectional &
\setlength{\baselineskip}{0.5\baselineskip}  Impulsive action &
\setlength{\baselineskip}{0.5\baselineskip}  Stop signal task &
  \multicolumn{1}{m{3cm}}{\raggedright\setlength{\baselineskip}{0.5\baselineskip} \hspace{-0.04cm}- Stop accuracy (last half): ↓ (PVP \textless nV)\\ - SSRT: not significant} &
  \multicolumn{1}{c}{-}  \\* \midrule
\raggedright\setlength{\baselineskip}{0.5\baselineskip} \cite{limimpul2015} &
  Mixed &
  \begin{tabular}[c]{@{}c@{}}ELS (23) \\ HC (27) \\ Psychiatric-\\ nonELS (20)\end{tabular} &
\multicolumn{1}{m{2cm}}{\centering\setlength{\baselineskip}{0.5\baselineskip} ELS 17.2 (2.44)\\ HC 17.5 (1.63)\\ Psychiatric-nonELS \\ 17.0 (2.42)} &
\setlength{\baselineskip}{0.5\baselineskip}  Cross-sectional &
\setlength{\baselineskip}{0.5\baselineskip}  Impulsive action &
\setlength{\baselineskip}{0.5\baselineskip}  Stop signal task &
  \multicolumn{1}{m{3cm}}{\raggedright\setlength{\baselineskip}{0.5\baselineskip} \hspace{-0.04cm}- Stop accuracy: not significant\\- SSRT: ↑} &
\multicolumn{1}{m{3.5cm}}{\raggedright\setlength{\baselineskip}{0.5\baselineskip} \hspace{-0.04cm}- Correct stop vs. correct go: not significant\\- Incorrect stop vs. correct go:presupplementary motor area, supplementary motor area, dorsal anterior cingulate cortex, superior frontal gyri, left paracentral lobule  ↑} \\* \midrule
\raggedright\setlength{\baselineskip}{0.5\baselineskip} \cite{muellerimpul2010} &
  Mixed &
  \begin{tabular}[c]{@{}c@{}}ELS (12) \\ HC (21)\end{tabular} &
  13.16 (2.58) &
\setlength{\baselineskip}{0.5\baselineskip}  Cross-sectional &
\setlength{\baselineskip}{0.5\baselineskip}  Impulsive action &
\setlength{\baselineskip}{0.5\baselineskip}  Change task (variant of stop signal task) &
  \multicolumn{1}{m{3cm}}{\raggedright\setlength{\baselineskip}{0.5\baselineskip}  Change signal reaction time (CSRT) ↑} &
\multicolumn{1}{m{3.5cm}}{\raggedright\setlength{\baselineskip}{0.5\baselineskip} \hspace{-0.04cm}- Correct change vs. correct go: left inferior prefrontal gyrus, right dorsal anterior cingulate cortex, left anterior cingulate cortex, left putamen, left caudate, precentral, right post central, left claustrum/insula, primary motor and sensorimotor cortex  ↑\\ - Correct change vs. incorrect change: left inferior prefrontal gyrus, left precentral/inferior frontal gyrus, left claustrum/insula ↑} \\* \midrule
\raggedright\setlength{\baselineskip}{0.5\baselineskip} \cite{stinson2024} &
  Mixed &
  \begin{tabular}[c]{@{}c@{}}Y0 (7895)\\ Y2 (4973)\end{tabular} &
\multicolumn{1}{m{2cm}}{\centering\setlength{\baselineskip}{0.5\baselineskip} Y0 \\approximately 10\\ Y2 \\approximately 12} &
  Longitudinal &
\setlength{\baselineskip}{0.5\baselineskip}  Impulsive action &
\setlength{\baselineskip}{0.5\baselineskip}  Stop signal task &
  \multicolumn{1}{m{3cm}}{\raggedright\setlength{\baselineskip}{0.5\baselineskip} \hspace{-0.04cm}-Stop accuracy: ↓ both 0, 2year (ACE)\\ SSRT: ↑ only in 2 year (ACE)} &
\multicolumn{1}{m{3.5cm}}{\raggedright\setlength{\baselineskip}{0.5\baselineskip} Correct stop vs. correct go: pre-supplementary motor area, right inferior frontal gyrus ↓, not significant in left inferior frontal gyrus, anterior cingulate cortex, dorsolateral prefrontal cortex, anterior insula (ACE correlation) / anterior insula, left pre-supplementary motor area ↑ (family conflict) \\ (both 0, 2 year)} \\* \midrule
\raggedright\setlength{\baselineskip}{0.5\baselineskip} \cite{achesonimpul2019} &
  Mixed &
  ELS (1192) &
  23.6 (3.5) &
\setlength{\baselineskip}{0.5\baselineskip}   Cross-sectional &
\setlength{\baselineskip}{0.5\baselineskip}   Impulsive choice &
\setlength{\baselineskip}{0.5\baselineskip}   Delay discounting task &
  \multicolumn{1}{m{3cm}}{\raggedright\setlength{\baselineskip}{0.5\baselineskip}  Discounting rate  ↑} &
  \multicolumn{1}{c}{-} \\* \midrule
\raggedright\setlength{\baselineskip}{0.5\baselineskip} \cite{hamiltonimpul2013} &
  Mixed &
  ELS (192) &
  30.42 (9.3) &
\setlength{\baselineskip}{0.5\baselineskip}   Cross-sectional &
\setlength{\baselineskip}{0.5\baselineskip}   Impulsive choice &
\setlength{\baselineskip}{0.5\baselineskip}   Experiential discounting task &
  \multicolumn{1}{m{3cm}}{\raggedright\setlength{\baselineskip}{0.5\baselineskip} Discounting rate: no significant association with severity} &
  \multicolumn{1}{c}{-} \\* \midrule
\raggedright\setlength{\baselineskip}{0.5\baselineskip} \cite{oshriimpul2018} &
  Mixed &
  225 participants &
  21.56 (2.24) &
  \begin{tabular}[c]{@{}c@{}}Longitudinal \\ (9 - 12 months)\end{tabular} &
\setlength{\baselineskip}{0.5\baselineskip}  Impulsive choice &
  \begin{tabular}[c]{@{}c@{}} Monetary \\ Choice \\ Questionnaire\end{tabular}&
  \multicolumn{1}{m{3cm}}{\raggedright\setlength{\baselineskip}{0.5\baselineskip}  Discounting rate  ↑}  &
 \multicolumn{1}{c}{-}  \\* \midrule
\raggedright\setlength{\baselineskip}{0.5\baselineskip} \cite{simmenimpul2015} &
  Mixed &
  \begin{tabular}[c]{@{}c@{}}ELS (103) \\ HC (50)\end{tabular} &
  \begin{tabular}[c]{@{}c@{}}ELS 76.75 (5.23)\\ HC 75.16 (5.66)\end{tabular} &
 \setlength{\baselineskip}{0.5\baselineskip}  Cross-sectional &
\setlength{\baselineskip}{0.5\baselineskip}   Impulsive choice &
\setlength{\baselineskip}{0.5\baselineskip}   Delay discounting task &
  \multicolumn{1}{m{3cm}}{\raggedright\setlength{\baselineskip}{0.5\baselineskip} Discounting rate: no significant association with severity} &
  \multicolumn{1}{c}{-} \\* \midrule
\raggedright\setlength{\baselineskip}{0.5\baselineskip} \cite{weissimpul2023} &
  Threat &
\setlength{\baselineskip}{0.5\baselineskip}  11872 participants &
  \multicolumn{1}{m{2cm}}{\centering\setlength{\baselineskip}{0.5\baselineskip}9.83 (0.61) \\at baseline} &
  Longitudinal &
  \multicolumn{1}{m{1.2cm}}{\centering\setlength{\baselineskip}{0.5\baselineskip}  Trait\\ impulsivity} &
\setlength{\baselineskip}{0.5\baselineskip}   UPPS-P positive urgency, negative urgency &
  \multicolumn{1}{m{3.01cm}}{\raggedright\setlength{\baselineskip}{0.5\baselineskip} \hspace{-0.04cm}- UPPS positive urgency: ↑\\ - UPPS negative urgency: ↑} &
 \multicolumn{1}{c}{-} \\* \midrule
\raggedright\setlength{\baselineskip}{0.5\baselineskip} \cite{gagnonimpul2013} &
  Mixed &
  122 participants &
  23.05 (3.72) &
 \setlength{\baselineskip}{0.5\baselineskip} Cross-sectional &
  \multicolumn{1}{m{1.2cm}}{\centering\setlength{\baselineskip}{0.5\baselineskip}  Trait\\ impulsivity} &
  UPPS &
  \multicolumn{1}{m{3cm}}{\raggedright\setlength{\baselineskip}{0.5\baselineskip} \hspace{-0.04cm}- UPPS negative urgency ↑  \\ - UPPS lack of planning ↑ \\ - UPPS lack of perseverance: not significant\\ - UPPS sensation seeking: not significant} &
  \multicolumn{1}{c}{-} \\* \midrule
\raggedright\setlength{\baselineskip}{0.5\baselineskip} \cite{hallowellimpul2019} &
  Mixed &
  ELS (30) &
  20.63 (2.2) &
 \setlength{\baselineskip}{0.5\baselineskip}   Cross-sectional &
  \multicolumn{1}{m{1.2cm}}{\centering\setlength{\baselineskip}{0.5\baselineskip}  Trait\\ impulsivity}&
  UPPS-P &
  UPPS-P ↑ &
 \multicolumn{1}{c}{-} \\* \midrule
\raggedright\setlength{\baselineskip}{0.5\baselineskip}\cite{liuimpul2019} &
  Mixed &
  55 publications &
   \multicolumn{1}{c}{-} &
   \multicolumn{1}{c}{-} &
  \multicolumn{1}{m{1.2cm}}{\centering\setlength{\baselineskip}{0.5\baselineskip}  Trait\\ impulsivity} &
  \multicolumn{1}{m{1.5cm}}{\centering\setlength{\baselineskip}{0.5\baselineskip} General trait impulsivity (BIS, ICS, SNAP-IV, EIS, CCQ) \\ (21 studies), \\ UPPS \\ (9 studies)} &
  \multicolumn{1}{m{3cm}}{\raggedright\setlength{\baselineskip}{0.5\baselineskip} \hspace{-0.04cm}- General trait impulsivity ↑\\ - UPPS; lack of perseverance/ negative urgency ↑, lack of planning, sensation seeking: not significant\\ - Negative urgency ↑ - sexual abuse, physical abuse, physical neglect\\ + motor impulsivity ↑: mixed, physical abuse, emotional abuse} &
 \multicolumn{1}{c}{-} \\* \midrule
 \raggedright\setlength{\baselineskip}{0.5\baselineskip} \cite{shinimpul2018} &
  Mixed &
  335 participants &
  21.7 (2.1) &
 \setlength{\baselineskip}{0.5\baselineskip}  Cross-sectional &
  \multicolumn{1}{m{1.2cm}}{\centering\setlength{\baselineskip}{0.5\baselineskip}  Trait\\ impulsivity} &
  UPPS &
  \multicolumn{1}{m{3cm}}{\raggedright\setlength{\baselineskip}{0.5\baselineskip} \hspace{-0.04cm}- UPPS negative urgency ↑ (emotional ACEs, high/multiple ACEs \textgreater low ACEs), not significant (house/community ACEs)  \\ - UPPS sensation seeking ↑ (house/community ACEs \textgreater emotional ACEs)\\- UPPS lack of planning, lack of perseverance: not significant} &
 \multicolumn{1}{c}{-} \\* \midrule
\raggedright\setlength{\baselineskip}{0.5\baselineskip}\cite{yangimpul2024} &
  Mixed &
  \setlength{\baselineskip}{0.5\baselineskip} 2925 participants &
  9.52 (0.50) &
  Longitudinal &
  \multicolumn{1}{m{1.2cm}}{\centering\setlength{\baselineskip}{0.5\baselineskip}  Trait\\ impulsivity}&
  UPPS-P &
   \multicolumn{1}{m{3cm}}{\raggedright\setlength{\baselineskip}{0.5\baselineskip} \hspace{-0.04cm} UPPS: ↑ (neighborhood violence, adverse life events, SES disadvantage), not significant (area poverty) (correlation)} &
  - \\* \bottomrule


\end{longtable}
\renewcommand{\baselinestretch}{1}
\small
\raggedright
\parbox{\linewidth}{ 
\textit{Note.} The criteria for distinguishing between cross-sectional and longitudinal designs were based on the timing of the administration of ELS and neurocognitive function measures. \\
\textit{Abbreviations.} ELS, Early Life Stress; HC, Healthy Control; PVP, Poly-victimization with PTSD; PVnP, Poly-victimization without PTSD; nPV, Non-PV nV, Non-victimization; ROI, Regions of Interest; RT, Reaction Time; UPPS, Urgency-Premeditation-Perseverance-Sensation Seeking; UPPS-P, Urgency-Premeditation-Perseverance-Sensation Seeking-Positive Urgency; ACE, Adverse Childhood Experience; SES, Socio-economic Status; BIS, Barratt Impulsiveness Scale; ICS, Impulsivity Control Scale; SNAP-IV, Swanson, Nolan, and Pelham Rating Scale; EIS, Eysenck Impulsivity Scale; CCQ, California Child Q-Set
}
\end{landscape}









% \begin{landscape}
% \begin{longtable}{c} % longtable 사용, 단일 열 테이블 설정
%     \caption{Summary of findings: early life stress and impulsivity} \\ % 캡션 번호 유지
%     \label{tab:summary_imp} \\
%     % 첫 번째 이미지
%     \includegraphics[width=\linewidth]{summary_imp_1.png} \\
%     \caption[]{(Continued)} \\ 
%     % 두 번째 이미지
%     \includegraphics[width=\linewidth]{summary_imp_2.png} \\
%     \caption[]{(Continued)} \\ 
%     % 세 번째 이미지
%     \includegraphics[width=\linewidth]{summary_imp_3.png} \\

% \end{longtable}
% \renewcommand{\baselinestretch}{1}
% \small
% \raggedright
% \parbox{\linewidth}{ 
% \textit{Note.} The criteria for distinguishing between cross-sectional and longitudinal designs were based on the timing of the administration of ELS and neurocognitive function measures. \\
% \textit{Abbreviations.} ELS, Early Life Stress; HC, Healthy Control; PVP, Poly-victimization with PTSD; PVnP, Poly-victimization without PTSD; nPV, Non-PV nV, Non-victimization; ROI, Regions of Interest; RT, Reaction Time; UPPS, Urgency-Premeditation-Perseverance-Sensation Seeking; UPPS-P, Urgency-Premeditation-Perseverance-Sensation Seeking-Positive Urgency
% }
% \end{landscape}

\subsection{Research gap and objectives}
Previous studies on ELS and neurocognitive functions using task-based fMRI have yielded mixed findings. Such inconsistencies may stem from varied definitions of ELS, with limited studies distinguishing between threat and deprivation, and small sample sizes, often fewer than 100 participants. Additionally, many studies have focused on adults with retrospective measures of ELS or have been cross-sectional when involving children, limiting insights into developmental impacts.

To address these gaps, this thesis leverages a large-scale longitudinal dataset to examine how threat and deprivation uniquely affect neurocognitive functions. This approach aims to clarify the distinct pathways associated with each dimension and provide deeper insights into their developmental impacts. Furthermore, previous studies have rarely accounted for interactions among neurocognitive functions. By considering these interactions, this thesis seeks to identify multivariate patterns of neurocognitive functions related to threat and deprivation, providing valuable information for interventions.

The primary objectives of this study are 1) to replicate and extend findings on the relationships between ELS and neurocognitive functions, focusing on the unique effects of threat and deprivation, and 2) to explore multivariate patterns of neurocognitive functions associated with threat and deprivation, using both cross-sectional and longitudinal analyses to capture developmental trajectories.

This thesis proposes the following hypotheses based on prior findings:

\textbf{Reward processing} Threat and deprivation are associated with decreased striatal activation during reward anticipation, especially in deprivation.

\textbf{Emotion processing} Positive emotional stimuli: Threat is associated with hyperactivation in emotion-related regions (e.g., amygdala, striatum), while deprivation is linked to reduced activation.\\
Negative emotional stimuli: threat and deprivation are linked to amygdala hyperactivation, with a stronger effect observed in threat.

\textbf{Working memory} Deprivation is associated with poorer working memory performance and reduced activation in key neural correlates of working memory.

\textbf{Impulsivity} Trait impulsivity: Threat and deprivation are associated with heightened trait impulsivity, particularly negative urgency. \\ 
Impulsive choice: Threat and deprivation are linked to increased impulsive choice.

While some hypotheses focus on well-established directional associations, others remain exploratory due to inconsistent findings in the literature. A summary of previous findings is presented in Table~\ref{tab:previous_results}.

% Please add the following required packages to your document preamble:
% \usepackage{booktabs}
% \usepackage{multirow}
% \usepackage{graphicx}
\begin{table}[]
\caption{Summary of previous results on the impact of early life stress and neurocognitive functions}
\label{tab:previous_results}
\resizebox{\columnwidth}{!}{%
\begin{tabular}{@{}lllcc@{}}
\toprule
\multicolumn{1}{c}{Domain} &
  \multicolumn{2}{c}{Task or measurement} &
  Threat &
  Deprivation \\ \midrule
\multirow{7}{*}{Reward processing} &
  \multirow{2}{*}{\raisebox{3.5em}{fMRI}} &
  \begin{tabular}[c]{@{}l@{}}Reward anticipation \\ (reward vs. neutral)\end{tabular} &
  \multicolumn{2}{c}{\begin{tabular}[c]{@{}c@{}}Striatal activation ↓ especially in deprivation \\ (ACC, insula, pallidum ↓)\end{tabular}} \\
 &
   &
  \begin{tabular}[c]{@{}l@{}}Reward feedback \\ positive vs. negative\end{tabular} &
  \multicolumn{2}{c}{Inconsistent} \\
 &
  \multirow{2}{*}{\begin{tabular}[c]{@{}l@{}}fMRI\\ Behavior\end{tabular}} &
  Accuracy (reward) &
  \multicolumn{2}{c}{N.S. (↓ in food insecurity)} \\
 &
   &
  Reaction Time (reward) &
  (↑ ↓ N.S.) &
  (↑ N.S.) \\
 &
  \multirow[t]{3}{*}{BAS} &
  Reward responsiveness &
  \multicolumn{2}{c}{\multirow{3}{*}{(N.S.)}} \\
 &
   &
  Drive &
  \multicolumn{2}{c}{} \\
 &
   &
  Fun seeking &
  \multicolumn{2}{c}{} \\ \midrule
\multirow{7}{*}{Emotion processing} &
  \multirow[t]{2}{*}{\raisebox{1em}{fMRI}} &
  \raisebox{1em}{Positive vs neutral} &
  \begin{tabular}[c]{@{}c@{}}Striatum, amygdala, pallidum ↑\\ N.S.\end{tabular} &
  \begin{tabular}[c]{@{}c@{}}N.S. \\ (ventral striatum ↓)\end{tabular} \\
 &
   &
  \raisebox{1em}{Negative vs neutral} &
  \begin{tabular}[c]{@{}c@{}}Amygdala ↑\\ (striatum, insula, pallidum, ACC ↑)\end{tabular} &
  \begin{tabular}[c]{@{}c@{}}Amygdala ↑ (some inconsistency)\\ (NAc ↓, hippocampus, ACC ↑)\end{tabular} \\
 &
\multirow{4}{*}{\raisebox{5em}{\begin{tabular}[c]{@{}l@{}}fMRI\\ Behavior\end{tabular}}}&
  Accuracy (positive) &
  \multicolumn{2}{c}{N.S.} \\
 &
   &
  Accuracy (negative) &
  \multicolumn{2}{c}{N.S. (↑ ↓)} \\
 &
   &
  Reaction Time (positive) &
  \multicolumn{2}{c}{N.S.} \\
 &
   &
  Reaction Time (negative) &
  \multicolumn{2}{c}{N.S. (↓)} \\ \midrule
\multirow{4}{*}{Working memory} &
  fMRI &
  2 back vs. 0 back &
  \multicolumn{2}{c}{Inconsistent} \\
 &
  \multirow{2}{*}{\raisebox{1em}{fMRI Behavior}} &
  Accuracy (2 back) &
  \multicolumn{2}{c}{↓ (N.S.)} \\
 &
   &
  Reaction Time (2 back) &
  \multicolumn{2}{c}{N.S.} \\
 &
  Behavioral tasks &
  Total score &
  \multicolumn{2}{c}{↓ especially in deprivation} \\ \midrule
\multirow{9}{*}{Impulsivity} &
  fMRI &
  Correct stop vs correct go &
  \multicolumn{2}{c}{Inconsistent} \\
 &
  \multirow{2}{*}{\raisebox{1em}{fMRI Behavior}}  &
  Reaction Time &
  \multicolumn{2}{c}{N.S. (↑ especially in deprivation)} \\
 &
   &
  Stop Accuracy &
  \multicolumn{2}{c}{N.S. (threat ↓)} \\
 &
  Choice impulsivity &
   &
  \multicolumn{2}{c}{↑ (N.S.)} \\
 &
  \multirow[t]{5}{*}{UPPS-P} &
  Positive urgency &
  \multicolumn{2}{c}{Inconsistent  (N.S. ↑)} \\
 &
   &
  Negative urgency &
  \multicolumn{2}{c}{↑} \\
 &
   &
  Lack of perseverance &
  \multicolumn{2}{c}{Inconsistent  (N.S. ↑)} \\
 &
   &
  Sensation seeking &
  \multicolumn{2}{c}{Inconsistent  (N.S. ↑)} \\
 &
   &
  Lack of planning &
  \multicolumn{2}{c}{Inconsistent  (N.S. ↑)} \\ \bottomrule
\end{tabular}%
}

    \renewcommand{\baselinestretch}{1}
    \small
    \parbox{\linewidth}{ % 테이블 너비에 맞춰 Note 설정
               \vspace{0.5em}
        \textit{Note.} Results in parentheses represent limited evidence. \\
        \textit{Abbreviations.} N.S., Not significant; BAS, Behavioral Activation System; UPPS-P, Urgency-Premeditation-Perseverance-Sensation Seeking-Positive Urgency; ACC, anterior cingulate cortex; NAc, nucleus accumbens.}
\end{table}

% \begin{table}[]
%     \centering
%     \caption{Summary of previous results on the impact of early life stress and neurocognitive functions}
%     \label{tab:previous_results}
%     \includegraphics[width=\linewidth]{previous_results.png}

%     \renewcommand{\baselinestretch}{1}
%     \small
%     \parbox{\linewidth}{ % 테이블 너비에 맞춰 Note 설정
%         \textit{Note.} Results in parentheses represent limited evidence. \\
%         \textit{Abbreviations.} N.S., Not significant; BAS, Behavioral Activation System; UPPS-P, Urgency-Premeditation-Perseverance-Sensation Seeking-Positive Urgency; ACC = anterior cingulate cortex; NAc, nucleus accumbens.
%     }
% \end{table}

\end{document}
