\makeatletter
\def\input@path{{../}}
\makeatother
\documentclass[../document-en.tex]{subfiles}

\begin{document}

\section{Methods}
\subsection{Participants}
The dataset used in this thesis originates from the Adolescent Brain Cognitive Development (ABCD) study, a large-scale longitudinal dataset. The ABCD study recruited 11,878 children aged 9–10 years and their parents or guardians across 21 research sites in the United States (\cite{garavanabcd2018}). Data collection began in September 2016, with annual follow-ups planned for ten years. Further details regarding the recruitment process and study design can be found in (\cite{garavanabcd2018}).

This thesis used ABCD release 5.0, which includes complete data from baseline through the three-year follow-up, as well as a subset of data from the four-year follow-up. Cross-sectional analyses used baseline data, while longitudinal analyses incorporated data from the baseline and two-year follow-up.

Of the 11,878 participants, individuals with missing data on any variables of interest were excluded. Additionally, fMRI data with poor imaging quality were removed based on the recommended image inclusion criteria (\cite{haglerabcd2019}). To minimize biases arising from shared genetic factors or familial environments, only one participant per family was randomly selected for inclusion. The final sample sizes were 3,568 for cross-sectional analyses at baseline (Y0; mean age = 9.96 years) and 2,222 for longitudinal analyses at the two-year follow-up (Y2; mean age = 11.94 years).


% ver not remove family
% Out of the 11,878 participants, individuals with missing data on any variables of interest were excluded. fMRI data with poor imaging quality were also removed following the recommended image inclusion criteria (\cite{haglerabcd2019}). The final sample size for the cross-sectional analyses is 3,907, and for the longitudinal analyses is 2,419.

\subsection{Measurement}
\subsubsection{Early life stress}

ELS measures were derived from baseline data and divided into two dimensions—threat and deprivation—based on the theoretical model of ELS (\cite{mclaughlinsheridanlambert2014}).

\textbf{Threat} Threat measures included sexual abuse, physical abuse, domestic violence, non-interpersonal trauma, and community violence/safety. These subdomains were assessed using items from parent and youth reported ABCD Family Environment Scale-Family Conflict Subscale Modified from PhenX, parent reported ABCD Neighborhood Safety/Crime Survey-Safety from Crime Subscale Modified from PhenX, parent reported ABCD KSADS - Post-Traumatic Stress Disorder. All items were binary coded (1 = ELS present, 0 = ELS absent).

Sexual abuse was assessed with three items, including: "A grown-up in the home touched your child in their privates, had your child touch their privates, or did other sexual things to your child." Physical abuse included five items, such as: "Shot, stabbed, or beaten brutally by a non-family member" and "A family member threatened to kill your child." Domestic violence, distinct from physical abuse that directly targets the child, was measured using both youth and parent-reported items, such as: "Family members sometimes hit each other."

Non-interpersonal trauma was assessed with six items, including experiencing a car accident, witnessing a natural disaster, or war. Community violence/safety was evaluated with four items, such as: "Witnessed someone shot or stabbed in the community" and "I feel safe walking in my neighborhood, day or night."


\textbf{Deprivation} Deprivation measures included emotional neglect, physical neglect, parental separation, previous institutionalization, food insecurity, utility insecurity, household poverty, and neighborhood poverty.

Emotional neglect was assessed using ten items from the ABCD Children’s Report of Parental Behavioral Inventory. These items evaluated relationships with the first and second caregivers, such as: "First caregiver makes me feel better after talking over my worries with him/her." Physical neglect was measured using five items related to parental monitoring from ABCD Parental Monitoring Survey, including: "In an average week, how many times do you and your parents/guardians eat dinner together?"

Parental separation was defined as parents reporting being "separated" or "divorced," or if the current partner was not the child’s biological or adoptive parent. Previous institutionalization was identified if the parent answered "yes" to the question: "Is your partner the child's adoptive parent?"

Food insecurity was assessed with an item such as: "In the past 12 months, has there been a time when you and your immediate family experienced any of the following: Needed food but couldn't afford to buy it or couldn't afford to go out to get it?" Utility insecurity was measured using six items about the affordability of housing or essential services, such as telephone or medical care, in the past year.

Neighborhood poverty was assessed using the Area Deprivation Index (ADI; \cite{kindadi2014}), with higher national percentiles indicating greater deprivation. Consistent with prior research, a national percentile of 85 or higher was classified as poverty (\cite{kindadi2014}). Household poverty was derived using the income-to-needs ratio (INR), as described in a previous study (\cite{kimpoverty2022}). Since income was reported in ranges, the median value of each range was used for calculations. The INR was calculated by dividing household income by the poverty line based on family size provided by the 2017 Poverty Guidelines (\cite{hhs_poverty_guidelines_2017}). An INR below 1 was defined as household poverty (\cite{kimpoverty2022}).

To calculate scores for each dimension, composite scores were created by counting the subdomains associated with each dimension as dichotomous variables. A subdomain was considered present if at least one item within it was marked. The threat score ranged from 0 to 5, and the deprivation score ranged from 0 to 8. Participants scoring within the top 10\% were assigned to the high-threat or high-deprivation groups. Specifically, the high-threat group included those with threat scores of 2 or higher (≥87th percentile), while the high-deprivation group included those with deprivation scores of 3 or higher (≥90th percentile). 

Alternative thresholds were tested but were found to be less suitable, leading to the selection of the 10\% threshold for group classification. For threat, the top 20\% cut-off included nearly half of the sample (up to the 50th percentile), making it unsuitable for meaningful group distinctions. Similarly, the top 5\% cut-off produced the same group classifications as the 10\% threshold, limiting its utility.

For a detailed description of the measures, refer to Appendix~\ref{appendix:A}.

\clearpage
\subsubsection{Neurocognitive functions}
Neurocognitive functions are divided into four domains: reward processing, emotion processing, working memory, and impulsivity.

These constructs were measured through a combination of self-report surveys, behavioral tasks, and behavioral/neural measures from fMRI tasks. Behavioral tasks, two for working memory and one for impulsivity, were available only at baseline and therefore limited to cross-sectional analyses.

For neural measures, spatial regions of interest (ROIs) for each task were primarily defined based on prior findings, which have reported inconsistent results. Additionally, core regions associated with the corresponding neurocognitive functions were included. The fMRI beta weights for each ROI, derived from first-level GLM analyses, were used as provided by the ABCD dataset. 

\textbf{Reward processing} The Monetary Incentive Delay (MID) task (\cite{knutonmid2001}; Figure~\ref{MID}) was used for reward processing. Participants saw one of five incentive cues: win \$5, win \$0.2, lose \$0.2, lose \$5, or \$0 (no win or lose). They then responded to a target and received feedback on winning a reward or avoiding a loss. The task consisted of 100 trials: 40 reward trials, 40 loss trials, and 20 no-money trials, with accuracy maintained at 60\% across all trials (\cite{caseyfmri2018}). 

Behavioral measures included the accuracy of reward trials and the average reaction time for successful reward trials. For fMRI analysis, two contrasts were used: the "reward versus neutral" contrast at cue onset, reflecting reward anticipation, and the "positive feedback versus negative feedback" contrast in reward trials, reflecting reward feedback. The ROIs included the rostral and caudal anterior cingulate cortex (rACC, cACC), the insula, the striatum (caudate nucleus, putamen, and nucleus accumbens [NAc]), and the pallidum.


\begin{figure}
    \centering
    \includegraphics[width=0.75\linewidth]{MID.jpg}
    \caption{Monetary Incentive Delay Task, Adapted from \textcite{caseyfmri2018}}
    \label{MID}
\end{figure}

\sloppy

The ABCD Youth Behavior Inhibition/Behavioral Activation System (BIS/BAS; \cite{carverbas1994}), modified from PhenX, was used to assess reward sensitivity and motivation for goal-directed behavior. Participants rated each item on a 4-point Likert scale, ranging from 0 (very true) to 3 (very false). The analysis focused on three BAS subscales: Reward Responsiveness, capturing positive reactions to anticipated or received rewards; Drive, reflecting motivation to pursue goals; and Fun Seeking, assessing the desire for new rewards.

\textbf{Emotion processing} The behavioral measures and fMRI beta weights from Emotional N-Back (EN-back) task (\cite{cohenenback2015}; Figure~\ref{ENBACK}) were used to assess emotion processing. In this task, participants responded when the current stimulus matched a target (0-back) or when it matched the stimulus from two previous trials (2-back). The task included two runs, each consisting of four 2-back blocks and four 0-back blocks. The stimuli were happy, fearful, and neutral faces, as well as places, presented in separate blocks. Each stimulus type included 40 trials, resulting in 160 trials in total (\cite{caseyfmri2018}). 

Behavioral measures included the accuracy for positive faces (positive accuracy), accuracy for negative faces (negative accuracy), average reaction time for correct responses to positive faces (positive RT), and average reaction time for correct responses to negative faces (negative RT). For fMRI analysis, beta weights were derived from two contrasts: "positive face versus neutral face' and "negative face versus neutral face". The ROIs included the rostral and caudal anterior cingulate cortex (rACC, cACC), insula, amygdala, striatum (caudate nucleus, putamen, and nucleus accumbens [NAc]), pallidum, and thalamic regions (thalamus, ventral diencephalon, and hippocampus). 

\begin{figure}
    \centering
    \includegraphics[width=1\linewidth]{ENBACK.jpg}
    \caption{Emotional N-Back Task, Adapted from \textcite{caseyfmri2018}}
    \label{ENBACK}
\end{figure}


\textbf{Working memory} The EN-Back task(\cite{cohenenback2015}; Figure~\ref{ENBACK}) was employed to assess working memory. For neural measures, the "2-back versus 0-back" contrast was utilized, as the 2-back condition required greater working memory capacity compared to the 0-back condition. The ROIs were defined based on the involvement of the Default Mode Network (DMN) and the Central Executive Network (CEN). DMN regions included the posterior cingulate cortex (PCC), precuneus, medial prefrontal cortex (mPFC; rostral anterior cingulate cortex [rACC] and medial orbitofrontal cortex [mOFC]), and inferior parietal lobule (IPL). CEN regions included dorsolateral prefrontal cortex (dlPFC; rostral and caudal middle frontal gyrus [rMFG, cMFG]) and the posterior parietal cortex (PPC; superior and inferior parietal lobule [SPL, IPL]). Behavioral measures included accuracy and average reaction time from the 2-back trials. 

List Sorting Working Memory Test (list sorting test; \cite{tulskywm2013}) and Dimensional Change Card Sort Test (card sort test; \cite{tulskywm2013}) from NIH Toolbox were also included as the behavioral measures of working memory. The list sorting test asked participants to sequence items in a specific order. Card sort test required participants to sort cards based on either their color or shape. Age-corrected scores for both tasks were included. However, these tasks were limited to cross-sectional analyses due to their lack of administration at the 2-year follow-up.


\textbf{Impulsivity} Impulsive action, trait impulsivity, and impulsive choice were assessed using the Stop Signal Task (SST), the modified UPPS-P Impulsive Behavior Scale for Children, and the Cash Choice Task, respectively.

Stop Signal Task (SST; \cite{logansst1984}; Figure~\ref{SST}) assessed neural and behavioral aspects of impulsive action. In this task, participants responded to a left or right arrow by pressing a corresponding button as quickly and accurately as possible during "Go" trials. In "Stop" trials, participants were required to inhibit their response when an unpredictable "Stop signal" (up-right arrow) appeared. Stop trials made up 16.67\% of all trials. The task consisted of 360 trials in total, divided equally into two runs. The stop signal delay (SSD) was dynamically adjusted to maintain approximately 50\% accuracy for stop trials (\cite{caseyfmri2018}).

Behavioral measures included Stop Signal Reaction Time (SSRT) and accuracy in stop trials (stop accuracy). For fMRI analysis, the "correct stop versus correct go" contrast was used to examine successful inhibition. The ROIs included the inferior frontal gyrus (IFG; pars triangularis and pars orbitalis), rostral and caudal anterior cingulate cortex (rACC, cACC), and the striatum (caudate nucleus, putamen, and nucleus accumbens [NAc]).

\begin{figure}
    \centering
    \includegraphics[width=1\linewidth]{SST.jpg}
    \caption{Stop Signal Task, Adapted from \textcite{caseyfmri2018}}
    \label{SST}
\end{figure}


To measure trait impulsivity, the modified Urgency, Premeditation, Perseverance, Sensation Seeking, and Positive Urgency Impulsive Behavior Scale for Children (UPPS-P; \cite{barchupps2018}) was used. This 20-item questionnaire, designed for youth, included five subscale scores: lack of planning, lack of perseverance, sensation seeking, positive urgency, and negative urgency. Participants rated each item on a 4-point Likert scale, ranging from 1 (not at all like me) to 4 (very much like me).

Impulsive choice was assessed using the Cash Choice Task (\cite{lucianacct2018}), which was included only in cross-sectional analyses. The task involved a single-item: "Let’s pretend a kind person wanted to give you some money. Would you rather have \$75 in 3 days or \$115 in months?” Choosing the smaller, immediate reward (\$75) reflects greater impulsivity, while selecting the larger, delayed reward (\$115) indicates lower impulsivity.

\subsubsection{Covariates}
\textbf{Demographics and Socioenocomic status (SES)}
Socio-demographic variables included the child’s age at interview, gender, race/ethnicity, parental education level, and maternal age at the child’s birth, all measured at baseline. Analyses included only children recorded as male or female. Race/ethnicity was categorized as "Hispanic" if the respondent identified as such. For those who did not identify as "Hispanic", the categories were "Black", "White", "Asian", and "Others". The "Others" category included participants reporting multiple races. Race/ethnicity was included based on prior findings suggesting that structural inequalities may influence variability in neurophysiological responses across racial/ethnic groups (\cite{harnettrace2023}). Parental education level was defined as the highest degree received, ranging from 0 (Never attended/Kindergarten only) to 21 (Doctoral degree). This measure reflects the education level of a single parent (mother or father), who participated in the study. Household income was excluded, because it was already part of the deprivation measure. When analyzing the effect of threat, the deprivation score was included in the model to indirectly account for the effects of income.

\textbf{Data acquisition site}
The 21 ABCD study sites were included as a control variable to account for site-related variability, such as differences in scanner hardware and data collection protocols. This inclusion helps reduce measurement bias.

\textbf{Adversity exposure before birth}
Adversity exposure before birth was included as a covariate due to its potential influence on neurocognitive function. Variables included birth weight (measured as a continuous variable), premature birth, cyanosis (blue at birth), and maternal smoking or alcohol consumption during pregnancy.


\textbf{Parental psychopathology and drug problem}
Parental psychopathology and substance use are often considered subcategories of ELS but were treated as covariates due to difficulties in clearly distinguishing between threat and deprivation.

Parental psychopathology was assessed at baseline using the Adult Self-Report from the Achenbach System of Empirically Based Assessment (ASEBA; \cite{achenbach2009asr}). The assessment included T-scores for scales oriented to the Diagnostic and Statistical Manual of Mental Disorders, Fifth Edition (DSM-5). Anxiety, somatic problems, depressive symptoms, avoidant personality problems, attention deficit/hyperactivity problems (inattention and hyperactivity/impulsivity subscales), and antisocial personality problems were included in the analyses. Note that only one parent (mother or father) completed the assessment, so these psychopathology measures may reflect data from a single parental source.

Parental substance use was assessed separately for mothers and fathers by determining whether they had problems with alcohol or drug use. Maternal and paternal substance use were analyzed separately to capture potential distinct pathways. For instance, maternal use may more directly impact daily caregiving if the mother is the primary caregiver, while paternal use may influence the broader home environment or family stability. Moreover, because data were collected separately for mothers and fathers, combining them into a single ‘parental use’ measure could conceal parent-specific differences and obscure potential differential effects on the child’s neurodevelopment.

\textbf{Children behavior checklist}
The parent-reported Child Behavior Checklist (CBCL; \cite{achenbach2009asr}) was used to assess the child's behavior over the past six months. T-scores for internalizing and externalizing problems were included as control variables due to their potential impact on neurocognitive functions. Baseline data were used for cross-sectional analyses, while 2-year follow-up data were utilized for longitudinal analyses.

\subsection{MRI acquisition}
The ABCD fMRI data (release 5.0) were used, which include pre-extracted task-related fMRI features. These features represent the averaged beta estimates for each brain region. Brain regions were parcellated using a surface-based method: cortical regions were defined by the Desikan–Killiany Atlas (\cite{desikanroi2006}), and subcortical regions by the Aseg Atlas (\cite{fischlroi2002}).

\subsection{Analysis}
\subsubsection{Generalized linear model (GLM)}
Generalized linear regression analyses were conducted to investigate the relationship between ELS and neurocognitive functions cross-sectionally and longitudinally. Threat and deprivation were treated as binary variables (e.g., low threat = 0, high threat = 1, based on a cut-off score) and entered as independent variables. This mass-univariate approach provides a clear interpretation by analyzing each neurocognitive outcome independently, clarifying how threat or deprivation relates to each specific domain. The analyses were performed using the "glm" function in R, comparing the following two models:

\hangindent=2em % 들여쓰기 간격 설정
\hangafter=1 % 첫 번째 줄 이후로 적용
\textbf{Model 1.} A model including only covariates

\hangindent=2em % 들여쓰기 간격 설정
\hangafter=1 % 첫 번째 줄 이후로 적용
\textbf{Model 2.} A model including covariates, threat, and deprivation

Results were considered significant only if the Akaike Information Criterion (AIC) of Model 2 was lower than that of Model 1, indicating better model fit. Baseline neurocognitive functions (N = 3,568) were used for cross-sectional analyses, while 2-year follow-up neurocognitive functions (N = 2,222) were used for longitudinal analyses.

Multicollinearity among predictors was evaluated using Variance Inflation Factor (VIF) and no variables exceeded the threshold of 5. Neurocognitive functions (reward processing, emotion processing, working memory, and impulsivity) were treated as dependent variables, resulting in 123 models for cross-sectional data and 120 models for longitudinal data. Because testing multiple outcomes can inflate the risk of false positives, p-values were adjusted using the false discovery rate (FDR) method, implemented in the stats R package (https://www.rdocumentation.org/packages/stats).

While the GLM approach is valuable for examining each outcome separately, this mass-univariate strategy may still heighten the risk of multiple comparisons and may fail to capture the complex, joint impacts of predictors. Therefore, to further mitigate these issues and explore multivariate patterns, LASSO regression was also employed.


\subsubsection{Binomial LASSO regression}
To explore multivariate patterns of neurocognitive functions associated with ELS, binomial LASSO (Least Absolute Shrinkage and Selection Operator; \cite{tibshiranilasso1996}) regression was applied. LASSO is a regression method that performs both variable selection and regularization to enhance the interpretability and accuracy of statistical models. Threat and deprivation were separately set as outcome variables, while covariates and neurocognitive functions were included as predictors. Four models were constructed:

\hangindent=2em % 들여쓰기 간격 설정
\hangafter=1 % 첫 번째 줄 이후로 적용
\textbf{Model 1.} Cross-sectional analysis with threat as the outcome variable: low threat group (coded as 0), high threat group (coded as 1)

\hangindent=2em
\hangafter=1
\textbf{Model 2.} Cross-sectional analysis with deprivation as the outcome variable: low deprivation group (coded as 0), high deprivation group (coded as 1)

\hangindent=2em
\hangafter=1
\textbf{Model 3.} Longitudinal analysis with threat as the outcome variable: low threat group (coded as 0), high threat group (coded as 1)

\hangindent=2em
\hangafter=1
\textbf{Model 4.} Longitudinal analysis with deprivation as the outcome variable: low deprivation group (coded as 0), high deprivation group (coded as 1)


To control for the influence of threat and deprivation, threat scores were included as continuous variables in models examining deprivation, and vice versa. Since ELS occurs before measuring neurocognitive functions, these analyses are not predictive in nature. However, this approach enables the identification of variables associated with ELS while accounting for the effects of other variables.

The analysis was performed using the "glmnet" and "easyml" (v0.1.1) R packages. The easyml package provides a framework for building and evaluating machine learning models with the "glmnet" package (\cite{ahn2017easyml}). The dataset was divided into 80\% training set and 20\% test set. The data was randomly split a hundred times, to ensure the robustness of the results. The model performance was averaged for every train-test split (\cite{ahn2017easyml}).

The integration of findings from the mass-univariate (GLM) and multivariate (LASSO) approaches provides complementary insights: GLM elucidates how threat or deprivation relates to each neurocognitive outcome individually, whereas LASSO highlights which combinations of variables best characterize ELS exposure. I focused on results that emerged consistently across both methods, enhancing the clarity and robustness of conclusions.

\end{document}
